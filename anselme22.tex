\documentclass[a4paper,12pt]{report}
%\usepackage[francais]{minitoc}
%\usepackage[latin1]{inputenc} 
\usepackage{ucs}
\usepackage[utf8x]{inputenc}
%\setcounter{minitocdepth}{1}
%\setcounter{parttocdepth}{2}
%\setlength{\ptcindent}{0pt}
%\renewcommand{\ptcfont}{\normalsize\rmfamily\upshape\mdseries}
%\renewcommand{\ptcCfont}{\normalsize\rmfamily\upshape\bfseries}
%\renewcommand{\ptcSfont}{\normalsize\rmfamily\upshape\mdseries}
%\setcounter{minitocdepth}{2} 
%\setlength{\mtcindent}{24pt} 
%\setlength{\mtcskipamount}{\bigskipamount}
%\renewcommand{\mtcfont}{\small\rmfamily\upshape\mdseries}
%\renewcommand{\mtcSfont}{\small\rmfamily\upshape\bfseries}
\newcommand{\mtcskipamount}{\bigskipamount}
\newcommand{\textlbrackdbl}[]
\setcounter{minitocdepth}{2} 
\setlength{\mtcindent}{24pt} 
\setlength{\mtcskipamount}{\bigskipamount}
%\renewcommand{\mtcfont}{\small\rmfamily\upshape\mdseries}
%\renewcommand{\mtcSfont}{\small\rmfamily\upshape\bfseries}
%\usepackage{french}
%\setcounter{parttocdepth}{2}
%\setlength{\ptcindent}{0pt}
%\renewcommand{\ptcfont}{\normalsize\rm}
%\renewcommand{\ptcCfont}{\normalsize\bf}
%\renewcommand{\ptcSfont}{\normalsize\rm}
\usepackage[french]{babel}
\usepackage{tikz}
\small
%%%%%%%%%%%%%% Definitions %%%%%%%%%%%%%%
\def\onehalf{\textstyle{\frac{1}{2}}}
 \def\D{{\mathcal D}{}}
 \def\Gammabol{{\stackrel{\circ}{\Gamma}}{}} 
 \def\gammabol{{\stackrel{\circ}{\Gamma}}{}}
 \def\Abol{{\stackrel{~\circ}{A}}{}}
 \def\Rbol{{\stackrel{\circ}{R}}{}}
 \def\Lbol{{\stackrel{\circ}{\mathcal L}}{}}
 \def\Tbol{{\stackrel{\circ}{T}}{}}
 \def\Dbol{{\stackrel{\circ}{\mathcal D}}{}}
 \def\nabol{{\stackrel{\circ}{\nabla}}{}}
 \def\jbol{{\stackrel{\circ}{\jmath}}{}}
 \def\Gammaw{{\stackrel{\bullet}{\Gamma}}{}}
 \def\Omegaw{{\stackrel{\bullet}{\Omega}}{}}
 \def\Aw{{\stackrel{~\bullet}{A}}{}}
 \def\Vw{{\stackrel{\bullet}{V}}{}}
 \def\Rw{{\stackrel{\bullet}{R}}{}}
 \def\Qw{{\stackrel{\bullet}{Q}}{}}
 \def\jw{{\stackrel{\bullet}{\jmath}}{}}
 \def\tw{{\stackrel{\bullet}{t}}{}}
 \def\L{{\mathcal L}{}}
 \def\Lw{{\stackrel{\bullet}{\mathcal L}}{}}
 \def\Tw{{\stackrel{\bullet}{T}}{}}
 \def\Kw{{\stackrel{\bullet}{K}}{}}
 \def\nablaw{{\stackrel{\bullet}{\nabla}}{}}
 \def\Dw{{\stackrel{\bullet}{\mathcal D}}{}}
 \def\dw{{\stackrel{\bullet}{D}}{}}
 \def\Sw{{\stackrel{\bullet}{S}}{}}
 \def\be{\begin{equation}}
 \def\ee{\end{equation}}
 \def\ba{\begin{eqnarray}}
 \def\ea{\end{eqnarray}}
 \def\nn{\nonumber}
 % d'Alembertian definition
 \def\qed{\hbox{${\vcenter{\vbox{%HOLLOW SQUARE
    \hrule height 0.4pt\hbox{\vrule width 0.4pt height 6pt
    \kern5pt\vrule width 0.4pt}\hrule height 0.4pt}}}$}}
 \def\qedw{{\stackrel{\bullet}{\qed}}{}}
%\usepackage[T1]{fontenc}
%\usepackage[pctex32]{graphics}
%%%%%%%%%%%%%%%%%fin d\'efinition%%%%%%%%%%%%%%%%%%%%%%%%%%%%%%%%%%%%%
%%%%%%%%%%%%%%%%%%%%%%%%%%%%%%%%%%%%%%%%%%%%%%%%%%%%%%%%%%%%%%%
%\usepackage[lmargin=4cm,rmargin=2.5cm]{geometry}
\def\eps{\epsilon}
\def\f{\frac}
\def\cN{{\mathcal N}}
\def\cC{{\mathcal C}}
\def\nn{\nonumber}
\newcommand{\rot}{\mathop{\rm rot}\nolimits}
\newcommand{\dive}{\mathop{\rm div}\nolimits}
%\newcommand{\\mbox{diam}}{\mathop{\rm \mbox{diam}}\nolimits}
\def\bv{\textbf{v}}
\def\bw{\textbf{w}}
\def\bu{\textbf{u}}
\def\bH{\textbf{H}}
\def\bn{\textbf{n}}
\def\bt{\textbf{t}}
\def\Om{\Omega}
\def\bff{\textbf{f}}
\def\cT{{\mathcal T}}
%\newcommand{\cT}{\mathcal{T}}
%-------------------------------------------------

%-------------------------------------------------


\usepackage{amstext,amsmath,amssymb,amsfonts}

\usepackage{eucal,rotating}
\usepackage{float}


\usepackage{setspace}
%------------------------------interligne verticale---------------------------
\onehalfspacing 
\makeatletter 
\newcommand{\LyX}{L\kern-.1667em\lower.25em\hbox{Y}\kern-.125emX\spacefactor1000} 
\makeatother
% \usepackage{hyperref}
% \usepackage{hyperref}
%\usepackage[top=2cm, bottom=2cm, left=2cm, right=2cm]{geometry}
%\usepackage[table]{xcolor}
%\usepackage{color}
\usepackage{multirow}
\usepackage{ctable}

% \usepackage[pdftex=true,
%   hyperindex=true,
%   colorlinks=true]{hyperref} 
% Couleurs des liens plus propre et plus joli!!!
% \hypersetup{
% 	colorlinks,%
% 	citecolor=blue,%
% 	filecolor=black,%
% 	linkcolor=black,%
% 	urlcolor=blue}

\addtolength{\hoffset}{-0.7cm}

\addtolength{\textwidth}{1.7cm}%1.6avant
\addtolength{\voffset}{-2.cm}
\addtolength{\textheight}{2cm}

%\addtolength{\textwidth}{1.7cm}%1.6avant
%\addtolength{\voffset}{-3.2cm}
%\addtolength{\textheight}{2cm}

%jusqu'ici
%avant il y avait ca
\usepackage{color}
%%\usepackage[all]{xy}
%\usepackage[centertags]{amsmath}
%\usepackage{latexsym}
% \usepackage{stmaryrd}
%jusqu icib

\usepackage{amsfonts}
\usepackage{amsmath}
\usepackage{amscd}
\usepackage{array}
\usepackage{multirow}
\usepackage{mathrsfs}
%\usepackage{multicolumn}
%avant il y avait ca

\usepackage{amssymb}
\usepackage{amsthm}

\frenchspacing  \linespread{1.5}%1.5avant
\usepackage{fancyhdr}
\pagestyle{fancy}
\usepackage[dvips]{epsfig}

%jusqu ici

\usepackage{graphicx}
\usepackage{t1enc}
\usepackage[latin1]{inputenc}
%\usepackage[french]{babel}

%\usepackage{amscd}
\usepackage{graphics}
\setlength{\unitlength}{1mm}

%%%%%%%%%%%%%%%%%%%%%%%%%%%%%%%%%%%%%%%%%%%%%%%%%%%%%%%%%%%%%%%%%%%%%%%%%%%%%%%%%%%%%%%%%%%%%%%%%%%%%%%%%%%%%%%%%%%%%%%%%%
%%%%%%%%%%%%%%%%%%%%%%%%%%%%%%%%%%%%%%%%%%%%%%%%%%%%%%%%%%%%%%%%%%%%%%%%%%%%%%%%%%%%%%%%%%%%%%%%%%%%%%%%%%%%%%%%%%%%%%%%%%%%?
%%%%%%%%%%%%%%%%%%%%%%%%%%%%%%%%%%%%%%%%%%%%%%%%%%%%%%%%%%%%%%%%%%%%%%%%%%%%%%%%%%%%%%%%%%%%%%%%%%%%%%%%%%%%%%%%%%%%%%%%%%%



%
% \setlength{\headheight}{14pt}  %%
% \setlength{\oddsidemargin}{0pt} %%{52pt} %%
% \setlength{\evensidemargin}{0pt}       %%{10pt} %%
% \setlength{\marginparwidth}     {72pt}%%{72pt}
% \pagestyle{headings}
% \linespread{1.13}
% \textheight 22cm
% \textwidth 18cm
% \hoffset -.9 cm
%  \voffset -0.9 cm
%  \def\dessous#1\sous#2{\mathrel{\mathop{\kern0pt#2}\limits_{#1}}}
% \normalbaselineskip=18pt
% \normalbaselines
%
% \parindent.5cm
% \parskip=5pt
%%%%%%%%%%%%%%%%%%%%%%%%%%%%%%%%%%%%%%%%%%%%%%%%%%%%%%%%%%%%%%%%%%%%%%%%%%%%%%%%%%%%%%%%%%%%%%%%%%%%%%%%%%%
%%%%%%%%%%%%%%%%%%%%%%%%%%%%%%%%%%%%%%%%%%%%%%%%%%%%%%%%%%%%%%%%%%%%%%%%%%%%%%%%%%%%%%%%%%%%%%%%%%%%%%%%%%%?
%%%%%%%%%%%%%%%%%%%%%%%%%%%%%%%%%%%%%%%%%%%%%%%%%%%%%%%%%%%%%%%%%%%%%%%%%%%%%%%%%%%%%%%%%%%%%%%%%%%%%%%%%%%%


%%%%%%%%%%%%%%%%%%  Sommaire   %%%%%%%%%%%%%%%%%%
%\usepackage[french]{minitoc}

%\dominitoc
%\dominilof
%\dominilot
%\usepackage[english]{minitoc}
%\setcounter{minitocdepth}{1}
\setcounter{tocdepth}{3}


    %\usepackage[francais]{minitoc}
%    \setcounter{minitocdepth}{4}
    \usepackage{setspace}
    \usepackage{float}


%%%%%%%%%%%%%%%%%%%%%%%%%%%%%%%%%%%%%%%%%%%%%%%%%%%%%%%%%


%\usepackage[english]{babel}
%\usepackage[english,french]{babel}
%\usepackage[french]{babel}

\makeatletter
\newcommand{\thechapterwords}
{ \ifcase \thechapter\or Un\or Deux\or Trois\or Quatre\or
Cinq\or Six\or Sept \or Huit\or Neuf\or Dix\or Onze\fi}
\def\thickhrulefill{\leavevmode \leaders \hrule height 1ex \hfill \kern \z@}
\def\@makechapterhead#1{%
  %\vspace*{50\p@}%
  \vspace*{15\p@}%
  {\parindent \z@ \centering \reset@font
        \thickhrulefill\quad
        \scshape \@chapapp{} \thechapterwords
        \quad \thickhrulefill
        \par\nobreak
        \vspace*{15\p@}%
        \interlinepenalty\@M
        \hrule
        \vspace*{15\p@}%
        \Huge \bfseries #1\par\nobreak
        \par
        \vspace*{15\p@}%
        \hrule
    \vskip 60\p@
    %\vskip 100\p@
  }}
\def\@makeschapterhead#1{%
  %\vspace*{50\p@}%
  \vspace*{15\p@}%
  {\parindent \z@ \centering \reset@font
        \thickhrulefill
        \par\nobreak
        \vspace*{15\p@}%
        \interlinepenalty\@M
        \hrule
        \vspace*{15\p@}%
        \Huge \bfseries #1\par\nobreak
        \par
        \vspace*{15\p@}%
        \hrule
    \vskip 60\p@
    %\vskip 100\p@
  }}
  \def\@makechapterhead#1{%
  %\vspace*{50\p@}%
  \vspace*{15\p@}%
  {\parindent \z@ \centering \reset@font
        \thickhrulefill\quad
        \scshape \@chapapp{} \thechapterwords
        \quad \thickhrulefill
        \par\nobreak
        \vspace*{15\p@}%
        \interlinepenalty\@M
        \hrule
        \vspace*{15\p@}%
        \Huge \bfseries #1\par\nobreak
        \par
        \vspace*{15\p@}%
        \hrule
    \vskip 60\p@
    %\vskip 100\p@
    }}
  \frenchspacing \pagestyle{headings}
%%avant c etait \frenchspacing  \linespread{1.1}%1.5avant
%%d?finition des ent?tes
\usepackage{fancyhdr}
\pagestyle{fancy}

%avant il y avait ca
 %\usepackage[dvips]{epsfig}
%jusqu ici

% Ceci permet d?avoir les noms de chapitre et de section
% en minuscules
%avant il y avait ca
\renewcommand{\sectionmark}[1]{\markboth{#1}{}}
\renewcommand{\sectionmark}[1]{\markright{\thechapter\ #1}}


\renewcommand{\sectionmark}[1]{\markboth{#1}{}}
%\renewcommand{\chaptermark}[1]{\markboth{#1}{}}
%\renewcommand{\sectionmark}[1]{\markright{\thesection\ #1}}
\fancyhf{} % supprime les en-t?tes et pieds pr?d?finis
\fancyhead[L,R]{\bfseries\thepage}% Left Even, Right Odd
\fancyhead[L]{\bfseries\rightmark} % Left Odd
\fancyhead[R]{\bfseries\leftmark} % Right Even
\renewcommand{\headrulewidth}{1pt}% filet en haut de page
\addtolength{\headheight}{14pt} % espace pour le filet (avant on avait 1pt
\renewcommand{\footrulewidth}{1.5pt}% filet en bas de page
%\addtolength{\footheight}{0.5pt} % espace pour le filet en bas
\fancypagestyle{plain}{ % pages de tetes de chapitre
\fancyhead{} % supprime l'entete
%\fancyfoot{} %supprime le pied de page
\renewcommand{\headrulewidth}{0pt}
}
\newcommand{\clearemptydoublepage}{%
\newpage{\pagestyle{plain}\cleardoublepage}}

\rhead{\textbf{\thepage}}
\lhead{\textsl{\leftmark}}

%\fancyfoot[LE, RO]{\tiny \textbf{Bakary MANGA \copyright URMPM/IMSP 2008}}
\fancyfoot[L, RO]{\tiny \textbf{ Anselme HADO   \copyright URPM/IMSP 2021
                                 \\ anselmehado85@yahoo.fr}}
% % \fancyfoot[L, RO]{\tiny \textbf{Andr\'e DEMBELE \copyright URMPM/IMSP 2010}}
\fancyfoot[LO]{\tiny \emph{  \textbf{
L'acc\'el\'eration de l'Univers et mod\`eles de gravit\'e
}   }}
\fancyfoot[RE]{\tiny \textsl{Math\'ematique}}

 \rhead{\textbf{\thepage}}
\lhead{\textsl{\leftmark}}
\usepackage{newlfont}

%\usepackage[active]{srcltx}
\hfuzz2pt
\newlength{\defbaselineskip}
\setlength{\defbaselineskip}{\baselineskip}
\newcommand{\setlinespacing}[1]%
           {\setlength{\baselineskip}{#1 \defbaselineskip}}
%\newcommand{\doublespacing}{\setlength{\baselineskip}%
                     %      {1.5 \defbaselineskip}}
%\newcommand{\singlespacing}{\setlength{\baselineskip}{\defbaselineskip}}
\newtheorem{remark}{Remark}[section]
\newenvironment{prof}[1][Preuve]{\textbf{#1.} }{\ \rule{0.5em}{0.5em}}
\theoremstyle{plain}
%serge
\newtheorem{thm}{Th\'eor\`eme}[chapter]
\newtheorem{cor}[thm]{Corollaire}
\newtheorem{lem}[thm]{Lemme}
\newtheorem{pro}[thm]{Proposition}
\newtheorem{hypo}[thm]{Hypoth\`ese}
\newtheorem{nota}[thm]{Notation}
%\theoremstyle{definition}
\newtheorem{dfn}[thm]{D\'efinition}
\newtheorem{rap}{Rappel}[chapter]
%\theoremstyle{remark}
\newtheorem{rmq}[thm]{Remarque}
%\newtheorem{nota}[thm]{Notations}
%\numberwithin{equation}{section}
%\theoremstyle{definition}
\newtheorem{expl}[thm]{Exemple}

%bakary
%\newtheorem{lemma}{Lemma}[section]
%\newtheorem{exercise}{Exercice}[section]
%\newtheorem{proposition}{Proposition}[section]
%\newtheorem{theorem}{Theorem}[section]
%\newtheorem{definition}{Definition}[section]
%\newtheorem{corollary}{Corollary}[section]
%\newtheorem{notation}{Notation}[section]
%\newtheorem{example}{Example}[section]
%\newtheorem{property}{Propri?©t?©}[section]
%\newtheorem{theodef}{Theorem and Definition}[section]
%\newtheorem{rap}{Rappel}[section]





%%%%%%%%%%%%%%%%%%%%%%%%%%%%%%%%%%%%%%%%%%%%%%%%%%%%%%%%%%%%%%%%%%%
\newcommand{\al}{\alpha}
%\newcommand{\be}{\beta}
\newcommand{\p}{\partial}
\newcommand{\pr}{\prime}
\newcommand{\mbb}{\mathbb}
\newcommand{\mbf}{\mathbf}
\newcommand{\mc}{\mathcal}
\newcommand{\mf}{\mathfrak}
\newcommand{\mi}{\mathit}
\newcommand{\nb}{\nonumber}
\newcommand{\ti}{\textit}
\newcommand{\T}{\mathcal{T}}
\newcommand{\ben}{\begin{enumerate}}
\newcommand{\een}{\end{enumerate}}
\newcommand{\brap}{\begin{rap}}
\newcommand{\erap}{\end{rap}}
\newcommand{\bnota}{\begin{nota}}
\newcommand{\enota}{\end{nota}}
%%%%%%%%%%%%%%%%%%%%%%%%%%%%%%%%%%%
\newcommand{\beqs}{\begin{eqnarray}}
\newcommand{\eeqs}{\end{eqnarray}}
\newcommand{\ms}[1]{\mathsf{#1}}
\newcommand{\bs}{\symbol{92}}
%%%%%%%%%%%%%%%%%%%%%%%%%%%%%%%%%%%%%%%%%%%%%%%%%%%%%%%%%%%%%%%%
\newcommand{\n}{\nabla}
\newcommand{\noi}{\noindent}
\newcommand{\beq}{\begin{eqnarray}}
\newcommand{\eeq}{\end{eqnarray}}
\newcommand{\bpro}{\begin{pro}}
\newcommand{\epro}{\end{pro}}
\newcommand{\blem}{\begin{lem}}
\newcommand{\elem}{\end{lem}}
\newcommand{\bdfn}{\begin{dfn}}
\newcommand{\edfn}{\end{dfn}}
\newcommand{\bcor}{\begin{cor}}
\newcommand{\ecor}{\end{cor}}
\newcommand{\bthm}{\begin{thm}}
\newcommand{\ethm}{\end{thm}}
\newcommand{\bex}{\begin{expl}}
\newcommand{\eex}{\end{expl}}
\newcommand{\brmq}{\begin{rmq}}
\newcommand{\ermq}{\end{rmq}}
\newcommand{\bhypo}{\begin{hypo}}
\newcommand{\ehypo}{\end{hypo}}
%\newcommand{\benum}{\begin{enumerate}}
%\newcommand{\eenum}{\end{enumerate}}
\newcommand{\bitem}{\begin{itemize}}
\newcommand{\eitem}{\end{itemize}}
\newcommand{\cE}{\mathcal{E}}
%\newcommand{\curl}{\mathop{ curl}\nolimits}
%\newcommand{\dive}{\mathop{ div}\nolimits}
\theoremstyle{plain}
%\frenchspacing
\linespread{1}

%\usepackage[latin1]{inputenc}



\usepackage{graphicx}
\usepackage{t1enc}
% \hypersetup{
%        backref=true,                           % Permet d'ajouter des liens dans
%        pagebackref=true,                       % les bibliographies
%        hyperindex=true,                        % Ajoute des liens dans les index.
%        colorlinks=true,                        % Colorise les liens.
%        breaklinks=true,                        % Permet le retour ?  la ligne dans les liens trop longs.
%        urlcolor= blue,                         % Couleur des hyperliens.
%        linkcolor= blue,                        % Couleur des liens internes.
%        citecolor=blue,
%        bookmarks=true,                         % Cr\'e\'e des signets pour Acrobat.
%        bookmarksopen=true,                     % Si les signets Acrobat sont cr\'e\'es,
%                                                % les afficher compl\`etement.
%        pdftitle={Syst\`eme de facilitation des transactions portuaires},  % Titre du document.
%      % Informations apparaissant dans
%        pdfauthor={Andre},                      % dans les informations du document
%        pdfsubject={M\'emoire de fin de formation}           % sous Acrobat.
%     }

%%%%%%%%%%%%%%%%%%%%%%%%%%%%%%%%%%%%%  pdf interactive links %%%%%%%%%%%%%%%%%%%%%%%%%%%%%%%%%%%%%%%%%%%%
%  \definecolor{couleurliens}{rgb}{0.,0.,0.}% % "couleurliens" will contain black color
% \definecolor{couleurliens}{cmyk}{0.,0.,0.,0.} %other color see colorfile
% % % activation for links colors  in the pdf
\definecolor{couleurliens}{rgb}{0.,0.,1} % for sections, subsections and equations
 \definecolor{couleurliensref}{rgb}{0.1,0.,0.9} % for book and paper
\definecolor{couleurliensurl}{rgb}{.3,.4,.3} % for internet references
% \usepackage[pdftex,colorlinks=true,urlcolor=couleurliensurl,linkcolor=coul
% eurliens,
% citecolor=couleurliensref]{hyperref}



 \usepackage[pdftex,pdfmenubar=true,bookmarks=true,pdftoolbar=true]{hyperref}
\hypersetup{colorlinks,citecolor=red,filecolor=red,linkcolor=blue,urlcolor=cyan,pdftex}%permet de choisir la couleur des equation, refrence ...


%\setcounter{tocdepth}{0} % pour indiquer la profondeur de la table des mati\`eres
%%%%%%%%%%%%%%%%%%%%%%%%%%%%%%%%%%%%%%%%%%%%%%%%% end preamble %%%%%%%%%%%%%%%%%%%%%%%%%%%%%%%%%%55%%%%%
\linespread{1.1}
%%%%%%%%%%%%%%%%%%%%%%%%%%%%%%%%%%%%%%%%%%%%%%%%%%%%%%%%%%%%%%%%%%%%%%%%%%%%%%%%%%%%%%%%

\title{
%{\textbf{THEORIES ALTERNATIVES:\\ Th\'eorie N\'eo-Newtonienne,  
 %Th\'{e}orie T\'{e}l\'{e}parall\`{e}le G\'{e}n\'{e}ralis\'{e}e}} 
 }
\author{ 
%In\`es Godonou SALAKO
}
\date{ 
%Novembre 2013
}

%%%%%%%%%%%%%%%%%%%%%%%%%%%%%%%%%%%%%%%%%%%%%%%%%%%%%%%%%%%%%%%%%%%%%%%%%%%%%%%%%%%%%%%%
\begin{document}
%\begin{normalsize}
\begin{titlepage} 
% \begin{figure}\includegraphics[height= 2 cm, width=2.1cm]
% {uac.pdf}
% %\hspace*{13cm}
% \ \includegraphics[height= 1.5 cm, width=11cm]{logoIMSP.pdf}
% \includegraphics[height= 2 cm, width=2cm]{ICTP.pdf}
% \end{figure}
\begin{center}
  Universit\'e d'Abomey-Calavi\\
{\bf Ecole Doctorale des Sciences Exactes et Appliqu\'ees }\\
Formation Doctorale Sciences des Mat\'eriaux\\[1cm]
\textbf{Memoire de Master II}\\
\underline{{\bf Option}}: Physiques Th\'eoriques et Math\'ematiques\\
\underline{{\bf Sp\'ecialit\'e}}: Cosmologie et gravitation
\end{center}
%\begin{center}
 %\textbf{TH\`ESE}
%\end{center}

\begin{center}
 %\rule{.8\textwidth}{.01pt}\\[0.3pt]
 \vspace{0.50cm}
 Th\`ese pr\'esent\'e et soutenu publiquement par:\\
 \vspace{0.5cm}
 \textbf{Anselme HADO }\\
 \begin{tabular}{|l|c|r|}
 \hline
 {\color{red}{\textbf{Titre}}}\\
 \hline
 \end{tabular}
\hrulefill
\begin{flushleft} 
\begin{center}
 \emph{{\color{black}\textbf{\small {L'acc\'el\'eration de l'Univers et mod\`eles de gravit\'e}}}}
 \end{center}
\end{flushleft} 
\hrulefill
 \vspace{0.20cm}
%\footnote{Support\'ee par le Centre International de la Physique Th\'eorique (Abdus Salam), Italie.
%by the Abdus Salam International Centre for Theoretical
%Physics (ICTP), Italy. %; %the 
%Japanese Government through the UNESCO-ICTP/IAEA/MORI and The Academy of Sciences for Developing World
%(TWAS)}
%\rule{.8\textwidth}{.01pt}\\[0.3pt]
\end{center}
%\vspace*{0.20cm}
%%%%%%%%%%%%%%%%%%%%%%%%%%%%%%%%%%%%%%%%%%%%%%%%%%%%%%%%%%%%%%%%%%%%%%%%%%%%%%%%%%%%%%%%%%%%%%%%%%%%%%%%%%%
\begin{center}
 \begin{tabular}{|l|c|r|}
 \hline
  {\color{red}{\textbf{Pr\'esident de jury}}} \\
 \hline
 \end{tabular}
\end{center}
\begin{center}
\small {\textbf{ Prof. },   \emph{ }}\\
\end{center}
%%%%%%%%%%%%%%%%%%%%%%%%%%%%%%%%%%%%%%%%%%%%%%%%%%%%%%%%%%%%%%%%%%%%%%%%%%%%%%%%%%%%%%%%%%%%%%%%%%%%%%%%%%%%
%%%%%%%%%%%%%%%%%%%%%%%%%%%%%%%%%%%%%%%%%%%%%%%%%%%%%%%%%%%%%%%%%%%%%%%%%%%%%%%%%%%%%%%%%%%%%%%%%%%%%%%%%%%
\begin{center}
 \begin{tabular}{|l|c|r|}
 \hline
{\color{red}{\textbf{Membres}}}  \\
 \hline
 \end{tabular}
\end{center}
\begin{center}
\small {\textbf{Prof. },   \emph{.}}\\
\small \textbf{Prof. }, \emph{.}
\\
\small \textbf{Prof. }, \emph{.}
\end{center}

\begin{center}
 \begin{tabular}{|l|c|r|}
 \hline
  {\color{red}{\textbf{Encadreur}}} \\
 \hline
 \end{tabular}
\end{center}
\begin{center}
\small {\textbf{ Dr. In\`es G.  S\textsc{alako} },   \emph{Ecole de G\'enie Rural, K\'etou}\\
\emph{Ecole Doctorale des Sciences Exactes et Appliqu\'ees.}}
% \\
% \small {\textbf{Dr. St\'ephane  H\textsc{oundjo}},   \emph{Facult\'e des Sciences et Techniques Natitingou, B\'enin .}\\
% \emph{Ecole Doctorale des Sciences Exactes et Appliqu\'ees}}
\end{center}

\begin{center}
 \begin{tabular}{|l|c|r|}
 \hline
  {\color{red}{\textbf{Superviseur}}} \\
 \hline
 \end{tabular}
\end{center}
\begin{center}
\small {\textbf{ Prof. St\'ephane  H\textsc{oundjo}},   \emph{Facult\'e des Sciences et Techniques Natitingou, B\'enin .}\\
\emph{Ecole Doctorale des Sciences Exactes et Appliqu\'ees}}
% \\
% \small {\textbf{ Prof. Antonin  K\textsc{anfon}},   
%  \emph{Facult\'e des Sciences et Techniques Natitingou, B\'enin .}\\
% \emph{Ecole Doctorale des Sciences Exactes et Appliqu\'ees}}\\
\end{center}


% 
% \begin{center}
%  \begin{tabular}{|l|c|r|}
%  \hline
%   {\color{red}{\textbf{Superviseur}}} \\
%  \hline
%  \end{tabular}
%  
% \end{center}
% \begin{center}
% %\small {\textbf{ Dr. Stephane  H\textsc{oundjo} },   \emph{Facult\'e des Sciences et Techniques Natitingou, B\'enin .}}\\
% \small {\textbf{ Prof. Jo\"{e}l  T\textsc{ossa}},   \emph{Institut de Math\'ematiques et de Sciences Physiques, B\'enin.}}
% \end{center}
\end{titlepage}
\onehalfspacing
\maketitle
%\doparttoc
%\dopartlof
%\dopartlot
%\dominitoc
%\doparttoc
%\dopartlof
%\dopartlot
%\dominilof
%\dominilot
%\dominitoc



\chapter*{D\'edicace}

\vspace*{5.cm}
\begin{center}
\begin{flushleft}
\hspace{0.25cm}{\itshape \`A 

 ma M\`ere KOUDOGBO Hou\'edassi\\
pour  son engagement face \`a ses responsabilit\'es dans mon \'education}
\end{flushleft}
\end{center}




\chapter*{Remerciements}
Je souhaite 
 
 
 

\chapter*{R\'esum\'e}
Da


 \textbf{Mots-cl\'es: }
  
 \emph{}

\chapter*{Abstract}
This 
 \textbf{Key Words: }
  
 \emph{Dark energy, generalized teleparallel theory of gravity $f(T)$, unimodular gravity,  ECHDE and ECNADE models}



%\listeofnotations
\tableofcontents
%\addstarredchapter{Liste des  publications}


\addstarredchapter{Table des figures}
\include{figure}
\listoffigures





%%%%%%%%%%%%%%%%%%%%%%%%%%%%%%%%%%%%%%%%%%%%%%%%%%%%%%%%%%%%%%%%%
% 
\chapter{Introduction G\'en\'erale} \label{introduction}
 \minitoc 
La th\'eorie de la Relativit\'e G\'en\'erale est certainement l'une des th\'eories les plus fas-
cinantes de la physique moderne : elle allie l'\'el\'egance d'une formulation g\'eom\'etrique des
lois de la gravit\'e \`a une capacit\'e in\'egal\'ee de description des ph\'enom\`enes gravitationnels du
monde physique qui nous entoure. S'appuyant sur elle, le mod\`ele cosmologique standard
parvient \`a expliquer, avec une dizaine de param\`etres, toutes les observations cosmologiques,
qu'il s'agisse de la nucl\'eosynth\`ese, du fond diffus cosmologique, de la forme des grandes
structures de l'Univers, ou encore de ph\'enom\`enes proprement relativistes comme les effets
de lentille gravitationnelle. Il parvient \`a donner une description coh\'erente de notre Univers,
depuis le Big-Bang et les premi\`eres ''graines'' des structures jusqu'aux galaxies que nous
observons aujourd'hui.\par
Plusieurs \'el\'ements laissent cependant penser que la Relativit\'e G\'en\'erale n'est peut-\^etre
pas la th\'eorie d\'efinitive de la gravit\'e. \'A tr\`es haute \'energie, on sait depuis l'av\`enement de
la th\'eorie quantique des champs que la Relativit\'e G\'en\'erale ne saurait \^etre valide pour des
\'echelles d'\'energies au-del\`a de l'\'echelle de Planck. La pr\'esence de ce cutoff ultra-violet nous
apprend donc qu'une autre th\'eorie doit prendre le relais \`a tr\`es haute \'energie, motivant
ainsi les recherches en th\'eories des cordes et en gravit\'e quantique. \'A l'autre bout du
spectre d'\'energie, c'est la cosmologie qui nous am\`ene \`a nous interroger sur la validit\'e
de la Relativit\'e G\'en\'erale. En effet, bien que le mod\`ele cosmologique standard r\'eussisse
brillamment \`a donner une description effective du monde physique, il doit pour cela faire
l'hypoth\`ese que $95\%$ du contenu de l'Univers nous est de nature inconnue. Ce secteur
sombre est alors suppos\'e \^etre constitu\'e pour  $70\%$ d'\'energie noire et pour  $25\%$ de mati\`ere
noire, la d\'enomination de ces composantes illustrant bien notre ignorance.
Au vu de ces difficult\'es, on ne peut s'emp\^echer de penser \`a d'autres th\'eories qui au
cours de l'histoire des Sciences sont parvenues \`a d\'ecrire les observations, certes correcte-
ment, mais au prix de raisonnements de plus en plus alambiqu\'es, jusqu'\`a finalement \^etre
d\'epass\'ees. Ce fut le cas de la th\'eorie des \'epicycles de Ptol\'em\'ee qui fut abandonn\'ee au
profit de la description copernicienne du syst\`eme solaire, ou encore de la th\'eorie de l'\'ether
rendue obsol\`ete par la relativit\'e restreinte. \'A chaque fois, il s'est agi d'interpr\'eter dans
un cadre th\'eorique renouvel\'e les observations qui posaient probl\`eme, afin de leur donner
une explication plus convaincante. La Relativit\'e G\'en\'erale conna\^itra-t-elle le m\^eme sort ?
Il est encore trop t\^ot pour le dire, mais il est sans nul doute l\'egitime de s'interroger sur
sa validit\'e et d'essayer de construire des th\'eories de la gravit\'e \'evitant de faire appel \`a des
sources d'\'energie inconnues. 
%%%%%%%%%%%%%%%%%%%%%%%%%%%%%%%%%%%%%%%%%%%%%%%%%%%%%%%%%%%%

Nous sommes encore loin d\'une th\'eorie de la gravit\'e qui serait plus satisfaisante d\'un
point de vue conceptuel que la Relativit\'e G\'en\'erale, tout en \'etant en accord avec les nom-
breuses contraintes exp\'erimentales qui existent sur les lois de la gravit\'e. Pour l\'instant,
aucune th\'eorie de gravit\'e modifi\'ee n\'a pu atteindre le statut de th\'eorie r\'ealiste de la gravit\'e. Face \`a la difficult\'e de la t\^ache que repr\'esente la construction d\'une th\'eorie de la gravit\'e
plus aboutie que la Relativit\'e G\'en\'erale, il est possible d\'adopter une d\'emarche l\'eg\`erement
diff\'erente, que l\'on peut qualifier de ph\'enom\'enologique. Cette d\'emarche consiste \`a proposer
et \`a \'etudier des mod\`eles de gravit\'e qui, sans avoir la pr\'etention d'\^etre enti\`erement r\'ealistes, ouvrent d\'int\'eressantes pistes d\'exploration qui pourront peut-\^etre mener in fine \`a
une th\'eorie coh\'erente de la gravit\'e.
C\'est dans ce cadre que se situe le travail de th\`ese pr\'esent\'e dans ce m\'emoire. Ce dernier
s\'articule en quatre grandes \'etapes.
 
 
\chapter{Introduction \`a la gravitation }\label{chapitre1}
\part{introduction \`a la RG}
\minitoc 
\section{Introduction}
La premi\`ere th\'eorie de la gravitation a \'et\'e formul\'ee par Isac Newton pendant plus de XVII Si\`ecle.
La th\'eorie de Newton con\c{c}oit la gravitation comme une force qui agit instantan\'ement \`a distance c'est \`a dire 
une attraction. Malgr\'e son succ\`es
\'evident, la th\'eorie Newtonienne contient deux inconv\'enients majeurs du point de vue th\'eorique:
\begin{enumerate}
 \item La propagation instantan\'ee de l'interaction gravitationnelle est en flagrante contradiction avec les principes
 de la Relativit\'e restreinte
 qui \'etablissent une vitesse limite dans la nature, celle de la lumi\`ere.
 \item Le principe d'\'equivalence n'est pas naturellement contenu dans la th\'eorie.
\end{enumerate}
Ces deux probl\`emes ont \'et\'e abord\'es par Einstein, qui a propos\'e une nouvelle vision de l'interpr\'etation de 
l'interaction gravitationnelle
d\'enomm\'ee {\it la th\'eorie de la gravitation}. Dans cette th\'eorie, la gravitation est vue comme une force attractive
mais comme une structure 
qui d\'eforme l'espace-temps quadri-dimensionnel. L'id\'ee centrale d'Einstein est de g\'eom\'etriser l'espace-temps en 
utilisant la notion de courbure. La relation fondamentale est du genre:\\  
{\it G\'eom\'etrie $\equiv$ Mati\`ere }.\\
Afin de mieux explorer la th\'eorie, il urge qu'on fasse un rappel des objets g\'eom\'etriques couramment utilis\'es.


le memoire ainsi redigé est 
le premier chapitre\ref{introduction} 

\section{La Gravitation comme g\'eom\'etrie de l'espace-temps}

La premi\`ere th\'eorie de la gravitation a \'et\'e formul\'ee par
Newton pendant le XVII si\`ecle.  Elle dit que la force
gravitationnelle entre deux corps ponctuels de masses $m_1$ et
$m_2$ est donn\'ee par l'expression
\begin{equation}
F = G\frac{m_1m_2}{r^2},
\end{equation}
o\`u $G = 6, 67\times10^{-11}\;N\, kg^{-2}\, m^{2}$ est la constante
gravitationnelle.  La th\'eorie Newtonienne con\c{c}oit la
gravitation comme une force qui agit instantan\'ement \`a distance. 
En plus,  si nous combinons la loi de force gravitationnelle
Newtonienne avec la deuxi\`eme loi de Newton,  nous obtenons pour
la force sur le corps de masse $m_2$,  par exemple, 
\begin{equation}
m_2a = G\frac{m_1m_2}{r^2} \quad \rightarrow \quad a =
G\frac{m_1}{r^2}.
\end{equation}
Cela veut dire que tous les corps subissent la m\^eme
acc\'el\'eration due au corps $m_1$,  ind\'ependamment de leur
masse. 
\par
Cette derni\`ere propri\'et\'e est due au fait qu'on a
consid\'er\'e que la masse qui appara\^{\i}t dans la deuxi\`eme
loi de Newton est la m\^eme qui appara\^{\i}t dans la loi de force
gravitationnelle.  Or,  ceci du point de vue th\'eorique n'est pas
vrai:  la masse qui appara\^{\i}t dans la deuxi\`eme loi est
reli\'ee aux propri\'et\'es inertielles du corps,  c'est-\`a-dire, 
\`a la tendance du corps \`a se maintenir dans un certain \'etat de
mouvement; la masse qui appara\^{\i}t dans la loi de
gravitation est une mesure de l'intensit\'e du champ
gravitationnel cr\'e\'e par le corps,  \'etant par cons\'equent une
sorte de "charge gravitationnelle".  La signification physique des
deux masses est donc compl\`etement diff\'erente,  et leur
\'egalit\'e ressort du fait exp\'erimental que tous les corps
subissent la m\^eme acc\'el\'eration dans un champ gravitationnel. 
% Il n'y a aucune raison th\'eorique pour que \c{c}a soit le cas. 
L'\'egalit\'e des masses inertielle et gravitationnelle est connue
comme le "principe d'\'equivalence". 
\par
D'une mani\`ere plus pr\'ecise,  dans le cas o\`u le contenu
mat\'eriel est vu comme un fluide,  la gravitation Newtonienne
conduit \`a un ensemble d'\'equations qui peuvent \^etre
appliqu\'ees \`a la description d'un mod\`ele cosmologique.  La
premi\`ere concerne la conservation de la masse.  Consid\'erons un
volume $V$ d\'efini par une surface ferm\'ee $S$.  La variation de
la masse \`a l'int\'erieur du volume est \'egale au courant de masse
\`a travers la surface: 
\begin{eqnarray}
\frac{dM}{dt} = - \int_S \vec j. d\vec S \quad &\rightarrow& \quad
\frac{d}{dt}\int\rho dV = - \int\nabla. \vec j dV \quad \rightarrow
\quad \nonumber\\
\frac{\partial\rho}{\partial t} + \nabla\vec j = 0, \quad & & \quad
\vec j = \rho\vec v. 
\end{eqnarray}
\par
Consid\'erons maintenant la deuxi\`eme loi de Newton pour un
\'el\'ement de ce fluide de densit\'e $\rho$.  Sur cet \'el\'ement, 
agissent le gradient de la pression et la force gravitationnelle: 
\begin{eqnarray}
\rho\frac{d\vec v}{dt} = - \nabla p - \rho\nabla\phi \rightarrow
\nonumber\\
\frac{\partial\vec v}{\partial t} + \vec v. \nabla \vec v = -
\frac{\nabla p}{\rho} - \nabla\phi, 
\end{eqnarray}
o\`u $\phi$ est le potentiel gravitationnel. 
\par
Finalement,  le potentiel gravitationnel d\^u \`a une distribution
de masse devient:
\begin{equation}
\phi(\vec r) = - G\int \frac{\rho(\vec r')}{|\vec r - \vec r'|}dV
.
\end{equation}
Utilisant le fait que
\begin{equation}
\nabla^2\frac{1}{|\vec r - \vec r'|} = - 4\pi\delta(\vec r - \vec
r'), 
\end{equation}
nous obtenons
\begin{equation}
\nabla^2\phi = 4\pi G\rho.
\end{equation}
\par
Ainsi,  le syst\`eme d'\'equations convenable pour d\'ecrire une
cosmologie Newtonienne est: 
\begin{eqnarray}
\label{cn1}
\frac{\partial\rho}{\partial t} + \nabla. (\rho\vec v) = 0, \\
\label{cn2} \frac{\vec v}{\partial t} + \vec v. \nabla \vec v =
- \frac{\nabla p}{\rho} - \nabla\phi, \\
\label{cn3} \nabla^2\phi = 4\pi G\rho.
\end{eqnarray}
L'eq. (\ref{cn1}) \'equation est l'\'equation de continuit\'e,  la
deuxi\`eme \'equation d'Euler et la troisi\`eme l'\'equation de
Poisson.  Dans ces expressions,  $\rho$ est la densit\'e d'un
fluide,  $\vec v$ le champ de vitesse du fluide,  $p$ la pression et
$\phi$ le potentiel gravitationnel. 
\par
Malgr\'e son succ\`es \'evident,  la th\'eorie de Newton contient
deux probl\`emes majeurs du point de vue th\'eorique: 
\begin{enumerate}
\item La propagation instantan\'ee de l'interaction
gravitationnelle est en flagrante contradiction avec les principes
de la Relativit\'e Restreinte qui \'etablissent une vitesse limite
dans la nature,  celle de la lumi\`ere. \item Le principe d'\'equivalence n'est pas naturellement contenu dans la th\'eorie. 
\end{enumerate}
\par
Ces probl\`emes ont \'et\'e abord\'es par Einstein,  qui a
propos\'e une nouvelle th\'eorie de la gravitation.  Dans cette
th\'eorie,  la gravitation n'est plus vue comme une force,  mais
comme la structure de l'espace-temps quadri-dimensionnel.  Ainsi, 
les principes relativistes sont inclus,  puisque nous avons
maintenant \`a faire \`a un espace-temps,  et aussi le principe d'\'equivalence,  puisque les corps se d\'eplacent tous de la m\^eme
mani\`ere dans une g\'eom\'etrie donn\'ee.  L'id\'ee centrale de la
th\'eorie Einsteinienne, c'est que la g\'eom\'etrie de
l'espace-temps n'est pas une donn\'ee a priori,  mais qu'elle est
d\'etermin\'ee par la distribution de mati\`ere.  La relation
fondamentale est du genre, 
\begin{equation}
\mbox{g\'eom\'etrie} = \mbox{mati\`ere}.
\end{equation}
 
 

\section{Equations d'Einstein \`a partir du formalisme lagrangien}
\subsection{Equations de champs}
La plus part des th\'eories \'elabor\'ees jusqu'\`a ce jour sont bas\'ees sur la th\'eorie de jauge, ceci afin de permettre que les \'equations qui d\'ecrivent les 
lois physiques soient invariantes par rapport \`a des changements de coordonn\'ees (Vitesse, Energie ...etc). Ces grandeurs physiques doivent
\^etre valables \`a tout point de l'univers. En therme plus technique, on dit que les lois de la physique sont invariantes par rapport
\`a un changement de variable (jauge locale) en laissant invariant le lagrangien par rapport au champ de jauge. Ainsi,
les \'equations d'Einstein peuvent \^etre obtenues \`a partir d'un formalisme lagrangien.   Ainsi l'action totale $S$ est la somme de 
l'action purement gravitationnelle et de l'action de la mati\`ere coupl\'ee \`a la gravitation \\
$$S = S_{g} + S_{m}.   $$ L'action $S_{g}$ est l'int\'egrale d'une densit\'e lagrangienne g\'eom\'etrique dans un domaine $\Omega$ et est  d\'efinie par:
 $$ S_{g} = \frac{1}{2 \kappa}\int_{\Omega} \sqrt{-g} R d^{4}x.   $$
L'action $S_{m}$ est l'int\'egrale d'une densit\'e lagrangienne li\'ee \`a la mati\`ere dans un domaine $\Omega$:
$$S_{m} = \int_{\Omega} \mathcal{L}_\mathrm{M} \sqrt{-g} d^{4}x $$                    
o\`u $ \mathcal{L}_\mathrm{M} $ est la densit\'e lagrangienne li\'ee \`a la mati\`ere et donc 
\begin{equation} 
   S = \int_{\Omega} \left[ {1 \over 2\kappa} \,  R + \mathcal{L}_\mathrm{M} \right] \sqrt{-g} \,  \mathrm{d}^4 x.   \label{aeq24}
\end{equation}
La variation de l'action par rapport \`a $g_{\mu\nu}$ est obtenue par une variation du champ de jauge 
$\delta g_{\mu\nu}$ qui s'annule sur le bord $\partial\Omega$ de $\Omega$ (principe variationnel)
ainsi que ses d\'eriv\'ees premi\`eres.   Alors on a
%\begin{equation}
%  0  = \delta S \label{aeq25}
%\end{equation}
\begin{equation}
 \delta S  = \int_{\Omega} \left[ {1 \over 
2\kappa} \frac{\delta (\sqrt{-g}R)}{\delta g^{\mu\nu}} + \frac{\delta 
(\sqrt{-g} \mathcal{L}_\mathrm{M})}{\delta g^{\mu\nu}} \right] \delta 
g^{\mu\nu}\mathrm{d}^4x \label{aeq26}
\end{equation}
\begin{equation}\label{aeq27} 
 \delta S   = \int_{\Omega} \left[ {1 \over 2\kappa} \left( 
\frac{\delta R}{\delta g^{\mu\nu}} + \frac{R}{\sqrt{-g}} \frac{\delta 
\sqrt{-g}}{\delta g^{\mu\nu} } \right) + \frac{1}{\sqrt{-g}} 
\frac{\delta (\sqrt{-g} \mathcal{L}_\mathrm{M})}{\delta g^{\mu\nu}} 
\right] \delta g^{\mu\nu} \sqrt{-g}\,  \mathrm{d}^4x.    
\end{equation}
En minimisant l'action, on obtient: 
 $$ 0  = \delta S $$ 
 Ce qui revient \`a:
\begin{equation}
   \int_{\Omega} \left(\frac{\delta R}{\delta g^{\mu\nu}} + \frac{R}{\sqrt{-g}} 
\frac{\delta \sqrt{-g}}{\delta g^{\mu\nu}}\right)\delta g^{\mu\nu} \sqrt{-g}\,  \mathrm{d}^4x = - 2 \kappa \int_{\Omega}
\frac{1}{\sqrt{-g}}\frac{\delta (\sqrt{-g} 
\mathcal{L}_\mathrm{M})}{\delta g^{\mu\nu}}\delta g^{\mu\nu} \sqrt{-g}\,  \mathrm{d}^4x \ \label{aeq28}
\end{equation}
et le second membre de (\ref{aeq28}) est par d\'efinition proportionnel au tenseur \'energie-impulsion
\begin{equation}
   T_{\mu\nu}:= \frac{-2}{\sqrt{-g}}\frac{\delta (\sqrt{-g} 
\mathcal{L}_\mathrm{M})}{\delta g^{\mu\nu}} = -2 \frac{\delta 
\mathcal{L}_\mathrm{M}}{\delta g^{\mu\nu}} + g_{\mu\nu} 
\mathcal{L}_\mathrm{M}.  \label{aeq29}
\end{equation}
Pour calculer le premier membre de (\ref{aeq28}) nous devons calculer les variations de la courbure scalaire et le d\'eterminant
de la m\'etrique.  \\
Pour calculer la variation de la courbure scalaire,  nous calculons d'abord la variation du tenseur de courbure de Riemann et
ensuite la variation 
du tenseur de Ricci.   Ainsi le tenseur de courbure de Riemann \'etant d\'efini en coordonn\'ees locales par:
\begin{equation}
   {R^\rho}_{\sigma\mu\nu} = \partial_\mu\Gamma^\rho_{\nu\sigma} - 
\partial_\nu\Gamma^\rho_{\mu\sigma} + 
\Gamma^\rho_{\mu\lambda}\Gamma^\lambda_{\nu\sigma} - 
\Gamma^\rho_{\nu\lambda}\Gamma^\lambda_{\mu\sigma}\label{aeq30}, 
\end{equation}
 on a 
\begin{equation}\label{a310}
   \delta ({R^\rho}_{\sigma\mu\nu}) = \partial_\mu \delta \Gamma^\rho_{\nu\sigma} - 
\partial_{\nu} \delta \Gamma^\rho_{\mu\sigma} + 
\delta \Gamma^\rho_{\mu\lambda}\Gamma^\lambda_{\nu\sigma} + \Gamma^\rho_{\mu\lambda}\delta \Gamma^\lambda_{\nu\sigma}-
\delta \Gamma^\rho_{\nu\lambda}\Gamma^\lambda_{\mu\sigma} - \Gamma^\rho_{\nu\lambda} \delta \Gamma^\lambda_{\mu\sigma}.     
\end{equation}
Consid\'erons $ \delta\Gamma^\rho_{\nu\mu}$ comme un $(1, 2)$-tenseur $\delta F$ et nous pouvons donc calculer sa d\'eriv\'ee covariante\\
en posant
$$   \delta F \equiv (\delta\Gamma^\rho_{\nu\mu}) \partial_{\rho}\otimes dx^{\nu}\otimes dx^{\mu}.     $$
On a
\begin{eqnarray*}
 \nabla_{\lambda}  \delta\Gamma^\rho_{\nu\mu}=\partial_{\lambda}( \delta\Gamma^\rho_{\nu\mu})\partial_{\rho}\otimes dx^{\nu}\otimes dx^{\mu} + \cr
(\delta\Gamma^\rho_{\nu\mu})\nabla_{\lambda}(\partial_{\rho})\otimes dx^{\nu}\otimes dx^{\mu}  +(\delta\Gamma^\rho_{\nu\mu}) \partial_{\rho}\otimes (\nabla_{\lambda}dx^{\nu})\otimes dx^{\mu} +\cr
(\delta\Gamma^\rho_{\nu\mu}) \partial_{\rho}\otimes dx^{\nu} (\nabla_{\lambda}dx^{\mu}) \\ 
\end{eqnarray*}

et comme  $$\nabla_{i} \partial_{j}= \Gamma_{ij}^{k} \partial_{k} \quad  et \quad  \nabla_{i} (dx^{k}) =-\Gamma_{ij}^{k} dx^{j},  $$ 
on  \'ecrit

$$
  \nabla_\lambda (\delta \Gamma^\rho_{\nu\mu} ) = \partial_\lambda 
(\delta \Gamma^\rho_{\nu\mu} ) + \Gamma^\rho_{\sigma\lambda} 
\delta\Gamma^\sigma_{\nu\mu} - \Gamma^\sigma_{\nu\lambda} \delta 
\Gamma^\rho_{\sigma\mu} - \Gamma^\sigma_{\mu\lambda} \delta 
\Gamma^\rho_{\nu\sigma},  $$
et on en d\'eduit que
 \begin{equation}
\partial_\lambda (\delta \Gamma^\rho_{\nu\mu} )= \nabla_\lambda (\delta \Gamma^\rho_{\nu\mu} )-[ \Gamma^\rho_{\sigma\lambda} 
\delta\Gamma^\sigma_{\nu\mu} - \Gamma^\sigma_{\nu\lambda} \delta 
\Gamma^\rho_{\sigma\mu} - \Gamma^\sigma_{\mu\lambda} \delta \Gamma^\rho_{\nu\sigma}].    \label{aeq31}
\end{equation}
  De    (\ref{aeq31}) on a aussi
\begin{equation} 
\partial_\mu (\delta \Gamma^\rho_{\nu\sigma} )= \nabla_\mu (\delta \Gamma^\rho_{\nu\sigma} )-[ \Gamma^\rho_{\lambda\mu} 
\delta\Gamma^\lambda_{\nu\sigma} - \Gamma^\lambda_{\nu\mu} \delta 
\Gamma^\rho_{\lambda\sigma} - \Gamma^\lambda_{\sigma\mu} \delta \Gamma^\rho_{\nu\lambda}] \label{a330}.  
\end{equation}
et
\begin{equation}
 \partial_\nu (\delta \Gamma^\rho_{\mu\sigma} )= \nabla_\nu (\delta \Gamma^\rho_{\mu\sigma} )-[ \Gamma^\rho_{\lambda\nu} 
\delta\Gamma^\lambda_{\mu\sigma} - \Gamma^\lambda_{\mu\nu} \delta 
\Gamma^\rho_{\lambda\sigma} - \Gamma^\lambda_{\sigma\nu} \delta \Gamma^\rho_{\mu\lambda}].    \label{a340}
\end{equation}

En introduisant (\ref{a330}) et (\ref{a340}) dans (\ref{a310}) on retrouve l'expression de la variation du tenseur de courbure de Riemann  qui
 est \'egale \`a la diff\'erence de deux termes:
\begin{equation}
   \delta R^\rho{}_{\sigma\mu\nu} = \nabla_\mu (\delta 
\Gamma^\rho_{\nu\sigma}) - \nabla_\nu (\delta \Gamma^\rho_{\mu\sigma}).  \label{aeq32}
\end{equation}

Nous pouvons maintenant obtenir la variation du tenseur de courbure de Ricci simplement en contractant deux indices
 de la variation du tenseur de courbure de Riemann :
\begin{equation}
 \delta R_{\mu\nu} \equiv \delta R^\rho{}_{\mu\rho\nu} = \nabla_\rho 
(\delta \Gamma^\rho_{\nu\mu}) - \nabla_\nu (\delta 
\Gamma^\rho_{\rho\mu}).  \label{aeq33}
\end{equation}
Le scalaire de Ricci est d\'efini comme
\begin{equation}
    R = g^{\mu\nu} R_{\mu\nu} \label{aeq34}
\end{equation}
par cons\'equent sa variation est donn\'ee par :
\begin{equation}
    \delta R = R_{\mu\nu} \delta g^{\mu\nu} + 
g^{\mu\nu} \delta R_{\mu\nu}\\ 
  = R_{\mu\nu} \delta g^{\mu\nu} + 
\nabla_\sigma \left( g^{\mu\nu} \delta\Gamma^\sigma_{\nu\mu} - 
g^{\mu\sigma}\delta\Gamma^\rho_{\rho\mu} \right) \label{aeq35}
\end{equation}
 La variation du d\'eterminant est donn\'ee par : 
\begin{equation}
 \delta g = \delta det(g_{\mu\nu}) = g \,  g^{\mu\nu} \delta g_{\mu\nu} \label{aeq36}
\end{equation}
En effet
\begin{equation}
  \delta \sqrt{-g} = -\frac{1}{2\sqrt{-g}}\delta g = -\frac{1}{2} \sqrt{-g} (g^{\mu\nu} \delta g_{\mu\nu}) = -\frac{1}{2} \sqrt{-g} (g_{\mu\nu} \delta g^{\mu\nu}, \label{aeq37}
\end{equation}
l'int\'egrale (\ref{aeq27}) sur $\Omega$ devient
 \begin{eqnarray}
&& \kappa \int_{\Omega}d^{4} T_{\mu\nu}\sqrt{-g}\delta g_{\mu\nu} =\int_{\Omega}d^{4} [\frac{-1}{2} g_{\mu\nu}R+R_{\mu\nu}]\sqrt{-g}\delta 
 g_{\mu\nu} \\ 
&& +\int_{\Omega}d^{4}x \sqrt{-g}[\nabla_\sigma \left( g^{\mu\nu} \delta\Gamma^\sigma_{\nu\mu} - 
g^{\mu\sigma}\delta\Gamma^\rho_{\rho\mu} \right)].   \label{aeq38}
\end{eqnarray}

\begin{eqnarray}
\kappa \int_{\Omega}d^{4}x T_{\mu\nu}\sqrt{-g}\delta g^{\mu\nu} &=&\int_{\Omega}d^{4}x [\frac{-1}{2} g_{\mu\nu}R+R_{\mu\nu}]\sqrt{-g}\delta 
  g^{\mu\nu}+ \cr
+\int_{\Omega}d^{4}x \sqrt{-g}[\nabla_\sigma \left( g^{\mu\nu} \delta\Gamma^\sigma_{\nu\mu} - 
 g^{\mu\sigma}\delta\Gamma^\rho_{\rho\mu} \right)]\label{aeq38}
\end{eqnarray}

  On peut re\'ecrire le second terme du second membre de (\ref{aeq38}) comme suit
\begin{equation}
 \int_{\Omega}d^{4}x \sqrt{-g}[\nabla_\sigma \left( g^{\mu\nu} \delta\Gamma^\sigma_{\nu\mu}-g^{\mu\sigma}\delta\Gamma^\rho_{\rho\mu} \right)]= 
\int_{\Omega}d^{4}x \sqrt{-g}[\partial_\sigma \left( g^{\mu\nu} \delta\Gamma^\sigma_{\nu\mu}-g^{\mu\sigma}\delta\Gamma^\rho_{\rho\mu} \right)]=0 \label{aeq39}
\end{equation}
car en vertu du th\'eor\`eme de Stokes l'int\'egrale se transforme en int\'egrale de surface sur le bord $\partial\Omega$ \quad et
 \quad les\quad $\delta\Gamma^\rho_{\rho\mu}$ sont nuls au bord.  \\
 L'\'egalit\'e (\ref{aeq39}) dans  (\ref{aeq38}),  on retrouve les \'equations d'Einstein comme
 \begin{eqnarray}
 \kappa T_{\mu\nu}= R_{\mu\nu}- \frac{1}{2} R g_{\mu\nu}.    
 \end{eqnarray}
En g\'en\'eral on pose
\begin{eqnarray}
 G_{\mu\nu}=  R_{\mu\nu}- \frac{1}{2} R g_{\mu\nu} \label{aeinstein1}
\end{eqnarray}
qui est appel\'e le tenseur d'Einstein.   


\subsection{Non-lin\'earit\'e des \'equations de champ}
Les \'equations d'Einstein donne lieu \`a $10$ \'equations ind\'ependantes aux d\'eriv\'ees partielles non-lin\'eaires
pour les composantes de la m\'etrique.   
Cette caract\'eristique de non-lin\'earit\'e distingue la relativit\'e g\'en\'erale de l'ensemble des autres th\'eories
physiques.   Par exemple 
les \'equations de
 Maxwell de l'\'electromagn\'etisme sont lin\'eaires par rapport aux champs \'electrique et magn\'etique 
(c'est-\`a-dire que la somme de deux solutions est aussi une
 solution).   Un autre exemple est celui de l'\'equation de Schr\"{o}dinger en m\'ecanique quantique o\`u l'\'equation est 
 lin\'eaire par rapport 
\`a la fonction d'onde.  
\subsection{Propri\'et\'es du tenseur d'Einstein}
En utilisant les relations d\'efinies en (\ref{aeq5}), (\ref{aeq6}), et (\ref{aeq7}) 
contractons la deuxi\`eme identit\'e de Bianchi (\ref{aeq9}) sur $k$ et $i$,  on a\\
\begin{equation}
 \nabla_{m}g^{ki}R_{klij} +\nabla_{i}g^{ki}R_{kljm} + \nabla_{j}g^{ki}R_{klmi} =0 \label{aeq14}
\end{equation}
\begin{equation}
 \nabla_{m}R_{lj} + \nabla_{i} g^{ki}R_{kljm} - \nabla_{j} R_{lm} =0 \label{aeq15}
\end{equation}
\begin{equation}
 \nabla_{i} g^{ki}R_{kljm} = \nabla_{j} R_{lm} -\nabla_{m}R_{lj} \label{aeq16}
\end{equation}
Contractons (\ref{aeq16}) \`a nouveau sur $j$ et $l$  pour obtenir\\
\begin{equation}
 \nabla_{i} g^{ki}g^{jl}R_{kljm} = \nabla_{j} g^{jl} R_{lm} -\nabla_{m} g^{jl}R_{lj} \label{aeq17}
\end{equation}
\begin{equation}
 -\nabla_{i} R^{i}_{m} - \nabla_{j}  R^{j}_{m} = \nabla_{m} R \label{aeq18}
\end{equation}
\begin{equation}
-2\nabla_{i} g^{ik}R_{km}=\nabla_{i} g^{ik}g_{km} R \label{aeq19}
\end{equation}
 \begin{equation}
 \nabla_{i} g^{ik} (  R_{km} - \frac{1}{2} g_{km} R ) =0 \label{aeq20}
 \end{equation}
\begin{equation}
 \nabla^{k} (  R_{km} - \frac{1}{2} g_{km} R) =0 \label{aeq21}
\end{equation}

L'\'egalit\'e (\ref{aeq21}) se traduit par l'identit\'e
\begin{eqnarray}
 \nabla^{k}G_{km}=0 \label{aeq}
\end{eqnarray}
avec
\begin{equation}
G_{km} =  R_{km} - \frac{1}{2}R g_{km}  \label{aeq22}
\end{equation}
Ainsi, le tenseur d'Einstein $G_{km}$ est donc un tenseur \`a divergence nulle.
\subsection{Conservation de l'\'energie et du moment}
On vient de d\'emontrer que  le tenseur d'Einstein est  \`a divergence nulle
\begin{eqnarray}
 \nabla^{\mu}G_{\mu\nu}=0
\end{eqnarray}
 et cela entra\^ine que
 \begin{eqnarray}
 \nabla^{\mu}T_{\mu\nu}=0
\end{eqnarray}
puisque 
\begin{eqnarray}
 G_{\mu\nu}=\kappa T_{\mu\nu}  
\end{eqnarray}
D'o\`u la Conservation de l'\'energie et du moment.

%%%%%%%%%%%%%%%%%%%%%%%%%%%%%%%%%%%%%%%%%%%%%%%%%%%%%%%%%%%%%%%%%%%%%%%%%%%%%%%%%%%%%%%%%%%%%%%%%%%%%%%%%%%%%%%%%
%%%%%%%%%%%%%%%%%%%%%%%%%%%%%%%%%%%%%%%%%%%%%%%%%%%%%%%%%%%%%%%%%%%%%%%%%%%%%%%%%%%%%%%%%%%%%%%%%%%%%%%%%%%%%%%%%
%%%%%%%%%%%%%%%%%%%%%%%%%%%%%%%%%%%%%%%%%%%%%%%%%%%%%%%%%%%%%%%%%%%%%%%%%%%%%%%%%%%%%%%%%%%%%%%%%%%%%%%%%%%%%%%%%
%%%%%%%%%%%%%%%%%%%%%%%%%%%%%%%%%%%%%%%%%%%%%%%%%%%%%%%%%%%%%%%%%%%%%%%%%%%%%%%%%%%%%%%%%%%%%%%%%%%%%%%%%%%%%%%%%

\part{Approximation des champs faibles}
\section{Equation de Continuit\'e}
La loi de conservation est exprim\'ee par le fait que la divergence du tenseur
d'\'energie-impulsion est nulle
 \begin{eqnarray}\label{b1conserve}
 \nabla^{\mu} T_{\mu\nu}=0.
\end{eqnarray}
Ainsi, \`a partir de (\ref{b1conserve}), nous pouvons d\'eduire la conservation de la masse qu'est l'\'equation de
continuit\'e. Pour cet, il faut exprimer le premier membre de (\ref{b1conserve}). \\
En effet, il faut rappeler que 
\begin{eqnarray}
 \nabla_{\mu} V^\nu = \partial_\mu V^\nu -\Gamma^\nu_{\mu\lambda}\, V^\lambda,
\end{eqnarray}
et
\begin{eqnarray}
 \nabla_{\mu} V_\nu = \partial_\mu V_\nu -\Gamma^\nu_{\mu\lambda}\, V_\lambda.
\end{eqnarray}
Ainsi,  nous d\'eduisons
\begin{eqnarray}
 \nabla_{\mu} T^\mu_\nu = \partial_\mu T^\mu_\nu  + \Gamma^\mu_{\mu\sigma}
 T^\sigma_\nu -  \Gamma^\sigma_{\mu\nu} T^\sigma_\mu
\end{eqnarray}
En faisant usage de (\ref{b1conserve}) et  pour $\nu=0$, on obtient:
\begin{eqnarray}
  \partial_\mu T^\mu_0  + \Gamma^\mu_{\mu\sigma}
 T^\sigma_0 -  \Gamma^\sigma_{\mu0} T^\sigma_\mu=0,
\end{eqnarray}
soit 
\begin{eqnarray}
  \partial_0 T^0_0  +\partial_i T^i_0+ \Gamma^i_{i0}
 T^0_0 +   \Gamma^i_{ij} T^j_0- \Gamma^i_{ii} T^i_0-\Gamma^i_{00} T^0_i -\Gamma^i_{i0} T^i_i=0
\end{eqnarray}
En consid\'erant que le contenu mat\'eriel de l'Univers est un fluide parfait de densit\'e 
$\rho$ et de pression $p$, on obtient:

\begin{eqnarray}
 \dot{\rho} + 3 \frac{\dot{a}}{a} \,(\rho + p)=0
\end{eqnarray}
o\`u $\vec{v}= \frac{\dot{a}}{a} \vec{r}$
\section{Equation de Poisson}\label{11poisson}
En se mettant dans l'hypoth\`ese de la {\it limite Newtonienne}, c'est \`a dire que les champs gravitationnels sont faibles, 
on peut \'ecrire la m\'etrique de l'espace-temps $g_{\mu\nu}$ sous la forme
\begin{equation}\label{b1gnunu}
  g_{\mu\nu}= \eta_{\mu\nu} + h_{\mu\nu} 
\end{equation}
 avec  $ |h_{\mu\nu} | << 1 $ et  $\eta_{\mu\nu}$ la m\'etrique de Minkowski. Cela revient \`a consid\'erer la m\'etrique 
 de l'espace-temps comme celle de Minkowski plus une petite perturbation.\\
 On montre que 
\begin{equation}\label{b1gum}
  g^{\mu\nu}= \eta^{\mu\nu} - h^{\mu\nu}  
\end{equation}
En effet,

\begin{eqnarray}
 \delta\alpha_\beta= g^{\alpha\mu}g_{\mu\beta}
\end{eqnarray}
Consid\'erons \; $g^{\mu\nu}= \eta^{\mu\nu} + k^{\mu\nu} $
\begin{eqnarray}
  \delta\alpha_\beta= \Big(\eta^{\alpha\mu} + k^{\alpha\mu}\Big)\Big(\eta_{\mu\beta} + k_{\mu\beta}\Big)
\end{eqnarray}
Ce qui conduit \`a
\begin{eqnarray}
 \eta^{\alpha\mu}  h_{\mu\beta} + k^{\alpha\mu}  \eta_{\mu\beta} + k^{\alpha\mu}  h_{\mu\beta} =0
\end{eqnarray}
Pour un d\'eveloppement du premier ordre en $ h_{\mu\beta}$ ou $k^{\alpha\mu}$, on peut n\'egliger le terme quadratique 
$k^{\alpha\mu}  h_{\mu\beta} $. Ainsi, on obtient  

\begin{eqnarray}
 k^{\alpha\mu}  \eta_{\mu\beta}&=& - \eta^{\alpha\mu} {\color{blue} h_{\mu\beta} } \cr
   &=&  - \eta^{\alpha\mu}  {\color{blue}   h_{\mu\nu}\eta^{\nu\beta}} \cr
  & =& - \eta^{\alpha\mu}    \eta^{\nu\beta}h_{\mu\nu}
\end{eqnarray}
or
\begin{eqnarray}
  k^{\alpha\beta}= \eta^{\alpha\mu}    \eta^{\beta\nu}h_{\mu\nu}
\end{eqnarray}
donc
\begin{eqnarray}
 k^{\alpha\mu}  \eta_{\mu\beta}&=&k^{\alpha\beta} \cr
 &=& - h_{\alpha\beta} 
\end{eqnarray}
D'o\`u 
\begin{eqnarray}
  g^{\mu\nu}= \eta^{\mu\nu} - h^{\mu\nu}
\end{eqnarray}
On retient contrairement \`a $g^{\mu\nu}$ et $\eta^{\mu\nu}$,  $h^{\mu\nu}$ n'est pas l'inverse de $h_{\mu\nu}$.\\

On aimerait v\'erifier que cela pr\'edit bien la loi de la gravitation de Newton dans les conditions de champs faibles.  
Dans ces conditions,  l'\'energie de la mati\`ere au repos $ \rho =T_{00} $  est tr\`es sup\'erieure aux autres
termes de  $ T_{\mu\nu}$ donc on va se focaliser sur la composante $ \mu=0,  \nu=0 $.   Dans la limite du
champ faible on \'ecrit conform\'ement  (\ref{b1gnunu}) et (\ref{b1gum}) \\

\begin{eqnarray}
 g_{00} &=& -1 + h_{00} \\ 
g^{00} &=& -1 - h^{00}.    
\end{eqnarray}
La trace du tenseur \'energie impulsion au premier ordre est:
$$   T=g^{00}T_{00}=-T_{00}.           $$
En la reportant  dans $$ R_{\mu\nu}=\kappa(T_{\mu\nu}-\frac{1}{2}T g_{\mu\nu})$$ on obtient:\\
\begin{equation}\label{b1taok}
  R_{00}= \frac{1}{2}\kappa T_{00}.   
\end{equation}        
Evaluons l'expression de $ R_{00} $ \`a partir de
$$R_{00}=R^{\alpha}_{0 \alpha 0} $$
 avec 
$$ R^{\alpha}_{0 \alpha 0}= \partial_{\alpha}\Gamma^{\alpha}_{00} -\partial_{0}\Gamma^{\alpha}_{\alpha 0}  + \Gamma^{\alpha}_{\alpha\beta}\Gamma^{\beta}_{00} -\Gamma^{\alpha}_{0 \beta}\Gamma^{\beta}_{\alpha 0}  $$
Le second terme est une d\'eriv\'ee par rapport au temps qui est nulle pour le champ statique.  
Le troisi\`eme et quatri\`eme sont d'ordre sup\'erieur et peuvent \^etre n\'eglig\'es.  
% Il nous reste  $ R^{\alpha}_{0 \alpha 0} =  \partial_{\alpha}\Gamma^{\alpha}_{00} $
\begin{eqnarray*}
 R_{00} &=& R^{\alpha}_{0 \alpha 0} \\
         & = & \partial_{\alpha}\Gamma^{\alpha}_{00} \\
        & = & \partial_{\alpha}[\frac{1}{2}g^{\alpha \lambda}(\partial_{0}g_{\lambda 0}+ \partial_{0}g_{0 \lambda }-\partial_{\lambda}g_{0 0} ) ]\\
         &= & -\frac{1}{2}\eta^{\alpha \lambda } \partial_{\alpha}\partial_{\lambda} h_{00} \\
         &  = &- \frac{1}{2} \nabla^{2}h_{00} \\
\end{eqnarray*}
En comparant \`a (\ref{b1taok}),  on voit que la composante $00$ de $ G_{\alpha\beta}= \kappa T_{\alpha\beta}$  en limite Newtonienne pr\'edit \\
$$  \nabla^{2}h_{00}= -\kappa T_{00} $$
En posant $ h_{00} = \Phi$ et  $\kappa = 8 \pi G $,  l'\'equation $ \nabla^{2}h_{00}= -\kappa T_{00} $ devient
$$  \nabla^{2} \Phi = -\kappa \rho.   $$
Ainsi, dans la limite Newtonienne, on constate que les \'equations d'Einstein se r\'eduisent \`a l'\'equation de poisson. 
\section{Deuxi\`eme loi de Newton}
Nous montrons dans cette partie que l'approximation des champs faibles permet d'obtenir la deuxi\`eme loi de Newton \`a partir de la 
r\'elativit\'e g\'en\'erale.  Pour y parvenir, nous rappelons les co\'efficients de la connexion(symboles de Christoffel) et l'\'equation des g\'eodesiques en r\'elativit\'e 
g\'en\'erale repectivement par:
\begin{equation}\label{b1chri}
 \Gamma^\lambda_{\mu\nu}=\frac{1}{2}g^{\lambda\sigma}(g_{\mu\sigma, \nu}+g_{\nu\sigma, \mu} -g_{\mu\nu, \sigma})
\end{equation}

\begin{equation}\label{b1geo}
 \frac{d^2x^\lambda}{d\tau^2} +\Gamma^\lambda_{\mu\nu}\frac{dx^\nu}{d\tau}\frac{dx^\mu}{d\tau}=0
\end{equation}
avec $d\tau^2=ds^2=g_{\mu\nu} dx^\nu dx^\mu$                         \\
Pour des champs de gravitation faibles on a $$\frac{v}{c}<<1 \quad ou \quad \frac{dx^i}{cdt}<<1 \quad(i=1, 2, 3),   \quad h_{\mu\nu}<<\eta_{\mu\nu}  
  \quad donc  \quad  ds^2=(dx^0)^2= c^2dt^2  $$ \\
En rempla\c{c}ant (\ref{b1gnunu}) et (\ref{b1gum}) dans (\ref{b1chri}) on a


\begin{equation}
 \Gamma^\lambda_{\mu\nu}=\frac{1}{2}( \eta^{\lambda\sigma} - h^{\lambda\sigma})(h_{\mu\sigma, \nu}+h_{\nu\sigma, \mu} -h_{\mu\nu, \sigma}) 
\end{equation}
A l'ordre $1$,  on peut n\'egliger $h^{\lambda\sigma}$ devant $\eta^{\lambda\sigma}$,  donc

\begin{equation}
 \Gamma^\lambda_{\mu\nu}=\frac{1}{2}( \eta^{\lambda\sigma})(h_{\mu\sigma, \nu}+h_{\nu\sigma, \mu} -h_{\mu\nu, \sigma}) 
\end{equation}
Pour $\lambda=0$ on a
\begin{equation}
 \Gamma^0_{\mu\nu}=\frac{1}{2}( \eta^{00})(h_{\mu0, \nu}+h_{\nu0, \mu} -h_{\mu\nu, 0}) 
\end{equation}
Le champ faible est donc statique et par consequent $ h_{\mu\nu, 0}=0$.  D'o\`u 

\begin{equation}\label{b1gamo}
 \Gamma^0_{\mu\nu}=\frac{1}{2}(h_{\mu0, \nu}+h_{\nu0, \mu}) 
\end{equation}
Pour$\lambda=i$,   on a
\begin{equation}
 \Gamma^i_{\mu\nu}=\frac{1}{2}( \eta^{ii})(h_{\mu i, \nu}+h_{\nu i, \mu} -h_{\mu\nu, i}) 
\end{equation}
\begin{equation}
 \Gamma^i_{\mu\nu}=-\frac{1}{2}(h_{\mu i, \nu}+h_{\nu i, \mu} -h_{\mu\nu, i}) 
\end{equation}
Revenons \`a l'\'equation de g\'eodesique.  \\
% Pour $\lambda=0$, en utlisant (\ref{b1gamo}), on obtient
% 
% \begin{equation}\label{b1geo2}
%  c^2\frac{d^2t}{ds^2} +\frac{1}{2}(h_{\mu0, \nu}+h_{\nu0, \mu})\frac{dx^\nu}{ds}\frac{dx^\mu}{ds}=0
% \end{equation}
% o\`u $x^0=ct$
% Lorsque nous d\'eveloppons le deuxi\`eme terme du premier membre de cette \'equation et en tenant compte des hypoth\`eses sur les champs faibles
% $(\frac{dx^i}{cdt}<<1)$,  cette derni\`ere se reduit en 
% \begin{equation}\label{b1geo3}
%  c^2\frac{d^2t}{ds^2} +\frac{1}{2}(h_{00, 0}+h_{00, 0})c^2\frac{dt}{ds}\frac{dt}{ds}=0
% \end{equation}
% $(h_{00, 0}+h_{00, 0})=0$ alors


 Pour $\lambda=i$,  puisque $ds=cdt $  alors (\ref{b1geo}) s'\'ecrit
\begin{equation}
 \frac{d^2x^i}{c^2dt^2}-\frac{1}{2}(h_{\mu i, \nu}+h_{\nu i, \mu} -h_{\mu\nu, i})\frac{1}{c^2}\frac{dx^\nu}{dt}\frac{dx^\mu}{dt}=0 
\end{equation}
D'apr\`es les approximations, on aura seulement $\mu=\nu=0 $ et donc$\frac{1}{c^2}\frac{dx^\nu}{dt}\frac{dx^\mu}{dt}=\frac{1}{c^2}c^2\frac{dt}{dt}\frac{dt}{dt}$, 
ce implique  

\begin{equation}
 \frac{d^2x^i}{c^2dt^2}-\frac{1}{2}(h_{0 i, 0}+h_{0 i, 0} -h_{00, i})=0 
\end{equation}
\begin{equation}
 \frac{d^2x^i}{c^2dt^2}=-\frac{c^2}{2}h_{00, i}
\end{equation}
\begin{equation}
 \begin{pmatrix}  \frac{d^2x^1}{dt^2} \\  \frac{d^2x^2}{dt^2}\\  \frac{d^2x^3}{dt^2}\end{pmatrix}= 
 -\frac{c^2}{2} \begin{pmatrix}  h_{00, 1} \\ h_{00, 2} \\ h_{00, 3} \end{pmatrix}
\end{equation}
Sous forme vectorielle on a 


\begin{equation}
 \frac{d^2 \vec r}{dt^2}=- \vec{grad}\phi  
 \end{equation}
o\`u $\phi=\frac{c^2}{2}h_{00}$

c'est la deuxi\`eme loi de Newton o\`u $\phi$ r\'epresente le potentiel \\
\section{Interpr\'etation physique}
En limite Newtonienne,  on consid\`ere la composante$(00)$ de $g$ devant $(ii)$ et 
\begin{eqnarray}
 g_{00}&=& \eta_{00}+ h_{00}\\
 &=&1+\frac{2\phi}{c^2}
\end{eqnarray}
Le potentiel $\phi$ est d\'efini  par 

\begin{equation}
 \phi=\frac{GM}{R}
\end{equation}
avec $M$et $R$ respectivement la masse et le rayon de la plan\`ete consider\'ee.  Apr\`es calcul on trouve:\\
$\bullet$ Sur la terre $\frac{2\phi}{c^2}\propto 10^{-9}$\\
$\bullet$sur le soleil $\frac{2\phi}{c^2} \propto 10^{-6}$\\
$\bullet$Naible  blanche $\frac{2\phi}{c^2}\propto 10^{-4}$\\
$\bullet$ Trou noir $\frac{2\phi}{c^2} \propto  1$\\
Ces resultats prouvent qu'on peut  n\'egliger  $ h_{00}$ devant $ g_{00}$ sur la terre  et non dans l'univers \`a grande \'echelle.  Cela 
montre que la deuxi\`eme loi de Newton n'est plus valable \`a grande \'echelle
 
 
 
 \chapter{Introduction \`a la cosmologie} \label{chapitre3}
 \minitoc
 \section{Introduction}

La cosmologie a pour but,  l'\'etude de l'\'evolution de notre Univers dans son enti\`eret\'e.  De cette mani\`ere,  elle
se distingue de l'astrophysique qui vise plut\^ot l'\'etude des objets particuliers qui
existent dans l'Univers,  comme les \'etoiles,  les galaxies...  etc.   La cosmologie essaye de
r\'epondre aux questions du genre: 
\begin{enumerate}
\item L'Univers,  est-il statique ou \'evolue-t-il avec le temps? \item
Est-il globalement homog\`ene ? 
Ses propri\'et\'es d\'ependent-elles de la
position dans l'espace? \item Comment peut on expliquer la
formation des galaxies,  les amas de galaxies,  et toutes les autres
structures que l'on observe? \item Comment se forment les
\'el\'ements chimiques qui existent dans la nature? \item Quelle est
la nature des composants mat\'eriels qui remplissent l'Univers?
\end{enumerate}
Evidemment,  cette liste n'\'epuise pas l'ensemble de probl\`emes
abord\'es par la cosmologie. 
\par
La force dominante dans l'Univers \`a grande \'echelle est la
gravitation.  Ainsi,  on doit avoir une th\'eorie de la gravitation
pour essayer de d\'ecrire la structure et l'\'evolution de
l'Univers.  Nous avons actuellement une th\'eorie de la gravitation
qui r\'epond de mani\`ere positive aux impositions th\'eoriques et
observationnelles:  c'est la Relativit\'e G\'en\'erale.  Du point de
vue th\'eorique elle incorpore les principes relativistes,  ainsi
comme le principe d'\'equivalence; du point de vue
observationnel,  la Relativit\'e G\'en\'erale a r\'esist\'e aux
tests locaux auxquels elle a \'et\'e soumise.  La Relativit\'e
G\'en\'erale d\'ecrit la gravitation comme la structure de
l'espace-temps cr\'e\'ee par la distribution de mati\`ere. 
\par
Nous d\'ebuterons ce chapitre sur la cosmologie en rappelant les
grandes lignes de la Relativit\'e G\'en\'erale comme dans \cite{julio}-\cite{gasperini}. 
L'Univers semble \^etre \`a grande \'echelle homog\`ene et
isotrope,  ainsi donc,   nous obtiendrons la structure
g\'eom\'etrique qui doit servir \`a la description de cet Univers
homog\`ene et isotrope.  Les possibles solutions dynamiques seront
d\'etermin\'ees et les principaux param\`etres observationnels seront
d\'ecrits.
 \section{Espace homog\`ene et isotrope - M\'etrique de Robertson-Walker} 
 $\qquad$ 	 En cosmologie et en astrophysique apparaissent diff\'erentes \'echelles. Donnons quelques ordres de grandeurs :
 \begin{align}
 \text{Rayon de la terre} &\simeq \;6,4.10^8 \text{cm} \\
 \text{Rayon du soleil} & \simeq \;7.10^{10} \text{cm} \\
 \text{Distance terre-soleil} &\simeq \; 1,5.10^{13}  =1\text{UA} 
 \end{align}
 Une unit\'e courante est le $parsec$, d\'efini comme \'etant la distance \`a laquelle la distance terre-soleil est vue comme sous un angle de 1 arcsec, i.e 
%  \begin{center}
%  	\[\includegraphics[scale=0.5]{0.PNG}\]
%  \end{center}
 
%  
%  \begin{figure}[h]
% \includegraphics[width=25pc]{0.PNG}
% \caption{}
% \end{figure}

\begin{figure}[h]
%\begin{minipage}{14pc}
\includegraphics[width=15pc]{t1.png}
\caption{}
% \includegraphics[width=15pc]{w_t2.eps}
% \caption{}
\end{figure}
 
 
 \begin{align*}
 pc=\frac{1\text{UA}}{1\text{arcsec}} &\simeq \; 3,26 \; \text{ ly}\\
 & \simeq \; 3.10^{18}\text{cm}
 \end{align*} 
 
 $\qquad$ Notre Univers est homog\`ene et isotrope \`a grande \'echelle (i.e pour des \'echelles plus grandes que l'\'echelle des amas de galaxie). Homog\`ene signifie qu'il n'existe pas de point pr\'ef\'er\'e dans l'espace. Isotrope signifie qu'il n'existe pas de direction pr\'ef\'er\'ee. Pour montrer que l'Univers est isotrope, on peut observer le ciel dans diff\'erentes directions et compter le nombre de galaxies pour voir s'il est plus ou moins le m\^eme dans chaque direction. Une autre preuve pour l'isotropie de l'Univers vient du fond de rayonnement cosmique (CMB -" Cosmic Microwave Brackground") qu'on discutera plus en d\'etail dans ce cours. Montrer que l'Univers est homog\`ene s'av\`ere plus difficile, vu qu'on ne peut pas l'\'etudier d'un autre point de l'espace. pour \'etudier l'homog\'en\'eit\'e on essaie de mesurer les distances entre les galaxies et de reconstruire une image en 3D de notre Univers. \\
 
 $\qquad$ On aimerait maintenant comprendre comment d\'ecrire math\'ematiquement un espace homog\`ene et isotrope. Essayons donc de trouver quelles sont les m\'etriques qui d\'ecrivent un espace homog\`ene et isotrope. Pour r\'epondre \`a cette question, oublions pour l'instant le temps et concentrons-nous sur la partie spatiale de la m\'etrique :
 \begin{equation*}
 ds^2=\gamma_{ij}dx^idx^j, \;  \text{avec signature}\left(\gamma \right) =\left(+,+,+ \right) 
 \end{equation*} \\
 
 $\qquad$ la m\'etrique $\gamma_{ij}$ d\'etermine compl\`etement la g\'eom\'etrie de l'espace courbe et entre autre le tenseur de courbure de Riemann $R_{ijl}$. Rappelons les propri\'et\'es sym\'etrie de Riemann 
 \begin{eqnarray}
  R_{ijkl} &=& -R_{jikl}, \cr 
  R_{ijkl} &=& -R_{ijlk},\cr
  R_{ijkl} &=& R_{klij}.
 \end{eqnarray} 
 On va utiliser ces propri\'et\'es et exiger qu'il n'existe pas de point ou de direction pr\'ef\'er\'e pour d\'eterminer la structure de $R_{ijkl}$ dans un espace homog\`ene et isotrope.\\
 
 $\qquad$ Choisissons un syst\`eme de coordonn\'ees localement plat autour d'un point $\bar{x}$, i.e. $\gamma	_{ij}=\delta_{ij}$ et $\Gamma_{jk}^i = 0$. Alors, par les propri\'et\'es de sym\'etrie et pour avoir un espace sans direction pr\'ef\'er\'ee, on aura :
 \begin{equation}
 R_{ijkl}=\zeta \left[ \delta_{ik} \delta_{jl} - \delta_{il} \ {jk}\right],
 \end{equation} 
 Par contraction des indices on trouve successivement le tenseur de Ricci et la courbure scalaire 
 \begin{eqnarray}
 R_{ij}&=&2\gamma_{ij} \zeta \cr
 R&=&6\zeta
 \end{eqnarray}
 On distingue 3 types d'espace 
 $	\begin{cases}
 \zeta >0 :\text{coubure constante positive},\\
 \zeta =0 : \text{espace plat}, \\
 \zeta <0 : \text{coubure constante n\'egative}.
 \end{cases}$ 
  \subsection*{Cas $\zeta =0$} 
  On peut  choisir les coordonn\'ees cart\'esiennes : $\gamma_{ij}= \delta_{ij}$. Donc 
  \begin{equation}
  dl^2=dx^2+dy^2+dz^2
  \end{equation}
  \subsection*{Cas $\zeta>0$} 
  En trois dimensions, on conna\^{\i}t un exemple d'un  espace homog\`ene et isotrope \`a courbure constante positive: la $3$-sph\`ere, $S^3$ d\'efinie par $ \sum_{i=1}^{4}x_i^2=a^2$, o\`{u} $a$ est le rayon de la sph\`ere. L'\'el\'ement de distance est le m\^eme qu'en espace plat quadri-dimensionnel, \`a savoir:
  
   \begin{eqnarray}
   dl^2= dx_1^2+dx_2^2+dx_3^2+dx_4^2
   \end{eqnarray}
   Toutefois sur la $3$-sph\`ere on peut exprimer $dx_4$ en fonction des autres coordonn\'ees, 
   \begin{eqnarray}
   x_1dx_1+x_2dx_2+x_3dx_3+x_4dx_4 =0,
   \end{eqnarray}
   d'o\`{u} 
   \begin{eqnarray}
   dl^2= dx_1^2+dx_2^2+dx_3^2+ \frac{(x_1dx_1+x_2dx_2+x_3dx_3)^2}{a^2-x_1^2+x_2^2+x_3^2}. 
   \end{eqnarray}
   On aimerait encore \'ecrire la m\'etrique sous une forme simplifi\'ee. Introduisons des coordonn\'ees sph\'eriques $(r,\theta,\varphi) $ d\'efinies par 
  \begin{eqnarray}
   x_1&=&r\sin \theta \cos \varphi \cr
   x_2&=&r\sin \theta \sin \varphi \cr
   x_3=r\cos \theta 
   \end{eqnarray}
   On trouve alors que 
    \begin{eqnarray}
    dx_1^2+dx_2^2+dx_3^2=dr^2+r^2\left[ d\theta ^2 + \sin ^2 \theta d\varphi^2  \right],
    \end{eqnarray}
    et 
    \begin{eqnarray}
    x_1dx_1+x_2dx_2+x_3dx_3=rdr.
    \end{eqnarray}
    L'\'el\'ement de longueur est donn\'e par 
    \begin{eqnarray}
    dl^2&=& dr^2+r^2\left[ d\theta ^2 + \sin ^2 \theta d\varphi ^2 \right] + \frac{r^2dr^2}{a^2-r^2} \cr
    &=& \frac{a^2}{a^2-r^2} dr^2+ r^2 \left[ d \theta^2 + \sin^2 \theta^2 d  d\varphi ^2 \right] 
    \end{eqnarray}
 Posons $d\Omega ^2 := d \theta ^2 + \sin ^2 \theta d\varphi ^2$ et $\bar{r}:= \frac{r}{a}, $ 
 \begin{equation}
dl^2 =a^2 \left[ \frac{d \bar{r}^2}{1-\bar{r}^2} + \bar{r}^2 d \Omega^2 \right] . 
\end{equation}
 Le domaine de d\'efinition des variables est 
  \begin{eqnarray}
  &&0 \leq \bar{r} \leq 1, \cr
 && 0\leq \theta \leq \pi , \cr
 && 0 \leq \varphi \leq 2\pi.
  \end{eqnarray}
Remarquons encore que la courbure scalaire est donn\'ee par $\frac{1}{a^2}>0$.
  
  \subsection*{Cas $\zeta <0$} 
  On s'attend \`a ce que le r\'esultat soit identique \`a celui de la sph\`ere, mais o\`{u} on a remplac\'e $a^2 \mapsto -a^2$ pour avoir une courbure scalaire n\'egative. Ainsi, 
  \begin{eqnarray}
   dl^2&=& \frac{-a^2}{-a^2-r^2} dr^2 +r^2d\Omega ^2 \nonumber \cr
  &=& a^2\left[ \frac{d \bar{r}^2}{1+\bar{r}^2} + \bar{r}^2 d\Omega^2 \right] 
  \end{eqnarray}
 Le domaine de d\'efinition des variables variables est donn\'e par 
 \begin{eqnarray}
 &&0 \leq \bar{r} \leq \infty, \cr
 &&0 \leq \theta \leq \pi , \cr
 && 0 \leq \varphi <2\pi .
 \end{eqnarray}
 
  On peut r\'esumer les trois cas par la formule g\'en\'erale 
  \begin{equation}
  dl^2=a^2\left[ \frac{d\bar{r}^2}{1-k\bar{r}^2} + \bar{r}^2d\Omega^2\right], 
  \end{equation} 
  o\`{u} 
  $$ k= \begin{cases}
   +1& \qquad  \text{, Univers ferm\'e,} \\
   0 &\qquad \text{, espace plat} \\
   -1 &\qquad \text{, Univers ouvert.} 
   \end{cases}$$
   Notons qu'avec une red\'efinition des coordonn\'ees, on peut toujours se ramener \`a\\ $k=-1,0,1$. Ces trois valeurs de $k$ correspondent aux diff\'erentes g\'eom\'etries possibles. La terminologie d'un Univers ferm\'e ou ouvert se r\'ef\`ere \`a la finitude ou infinitude du volume de l'Univers. \\
   
   Finalement, on doit rajouter la composante temporelle. Si l'espace est homog\`ene et isotrope, le param\`etre d'\'echelle $a$ peut au plus d\'ependre du temps. Ainsi, 
   \begin{equation}
    ds^2 = dt^2 -a(t)^2 \left[ \frac{d\bar{r}^2}{1-k\bar{r}^2} +\bar{r}^2 d \Omega ^2\right] .
   \end{equation}
  Cette m\'etrique est appel\'ee m\'etrique de Robertson-Walker (RW). Ci-dessus nous donnons les composantes non nulles des symboles de Christofell et du tenseur de Ricci, ainsi que la courbure scalaire.  
  \begin{eqnarray}
   \Gamma _{ij}^0&= & \frac{\dot{a}}{a} g_{ij} \cr
   \Gamma_{0j}^i &=& \frac{\dot{a}}{a} \delta_j^i, \cr
   \Gamma_{jk}^i &=& \frac{1}{2} g^{il} \left[ \partial_k g_{lj} + \partial_j g_{lk} - \partial_l g_{jk}  \right] \cr
   R_{00}&=& -3 \frac{\ddot{a}}{a}, \cr
   R_{ij}&=& - \left[ \frac{\ddot{a}}{a}+ 2 \frac{\dot{a}^2}{a^2} +2ka^2 \right] g_{ij} , \cr
   R&=& -6 \left[ \frac{\ddot{a}}{a} + \frac{\dot{a}^2}{2} + \frac{k}{a^2} \right]  .   
 \end{eqnarray}
  \section{Equations de Friedman} 
  Nous allons maintenant d\'eriver les \'equations r\'egissant l'Univers lorsque celui-ci est d\'ecrit par la m\'etrique de Robertson-Walker. Pour ce faire nous utilisons les \'equtions d'Einstein en prenant comme mati\`ere un fluide parfait qui d\'ecrira notre Univers homog\`ene et isotrope. Le tenseur d'\'energie-implusion d'un fluide parfait est: 
  \begin{eqnarray}
  T_{\mu \nu }= (p+\rho )u_\mu u _\nu -pg_{\mu \nu }
  \end{eqnarray}
  o\`{u} $p$ est la pression, $\rho $ la densit\'e d'\'energie et $u^\mu $ le quadri-vecteur vitesse. Si le fluide est au repos $u^\mu=\{1, \overrightarrow{0}\} $, alors $T_{OO}= \rho $ et $T_{ij}= -pg_{ij}$. \\
  
  
  En injectant ceci dan les \'equations d'Einstein
  \begin{eqnarray}
  R_{\mu \nu} - \frac{1}{2} g_{\mu \nu }
   R- \lambda g_{\mu \nu } = 8 \pi G T_{\mu \nu }, 
   \end{eqnarray}
   on trouve pour la composante $\mu =\nu =0$: 
   \begin{eqnarray}
   -3 \frac{\ddot{a}}{a} -\frac{1}{2} (-6)
 \left[\frac{\ddot{a}}{a} + \frac{\dot{a}^2}{a^2} + \frac{k}{a^2} \right] - \lambda = 8 \pi G \rho,
 \end{eqnarray}
ce qui nous donne la premi\`ere \'equation de Friedmann
\begin{eqnarray}
\frac{\dot{a}^2}{a^2} + \frac{k}{a^2} - \frac{\lambda}{3} = \frac{8\pi G}{3} \rho .
\end{eqnarray}
Pour la composante $ij$ on trouve 

\begin{eqnarray}
- \left[ \frac{\ddot{a}}{a} +2 \frac{\dot{a}^2}{a^2} +2 \frac{k}{a^2} \right]  g_{ij}  - \frac{1}{2} (-6) \left[ \frac{\ddot{a}}{a} + \frac{\dot{a}^2}{a^2} + \frac{k}{a^2} \right] g_{ij} - \lambda g_{ij} =8 \pi G(-p) g_{ij} ,
\end{eqnarray}
ce qui nous donne la deuxi\`eme \'equation de Friedmann 
\begin{eqnarray} 
2 \frac{\ddot{a}}{a} + \frac{\dot{a}^2}{a^2} + \frac{k}{a^2}- \lambda = -8\pi Gp.
\end{eqnarray}
 
 Toutes  les autres composantes des \'equations d'Einstein sont identiquement nulles. Voici en r\'esum\'e les \'equations r\'egissant l'Univers homog\`ene et isotrope d\'ecrit par la m\'etrique de Robertson-Walker, que l'on appelle commun\'ement \'equations de Friedmann 
 \begin{eqnarray}
 \frac{\dot{a}^2}{\dot{a}^2} + \frac{k}{a^2} - \frac{\lambda }{3} & = \frac{8 \pi G }{3} \rho \\
 2 \frac{\ddot{a}}{a} + \frac{\dot{a}^2}{a^2} + \frac{k}{a^2} - \lambda &= -8 \pi G p.
 \end{eqnarray} 
\textbf{Conservation du tenseur d'\'energie-impulsion} 

Par ailleurs, on aura la conservation du tenseur d'\'energie-impulsion 
\begin{equation}
\nabla_\mu T^{\mu \nu } = \partial_\mu T^{\mu\nu } + \Gamma_{\mu \alpha } ^\mu T^{\alpha \nu } + \Gamma_{\mu \alpha } ^\nu T^{\mu \alpha } =0
\end{equation}
Un calcul explicite de la composante $\nu =0$ donne 
\begin{eqnarray}
\nabla_\mu T^{\mu 0} &= \partial_\mu T^{\mu 0} + \Gamma_{\mu \alpha } ^\mu T^{\alpha 0} + \Gamma_{\mu \alpha }^0 T^{\mu \alpha } = \partial_0 T^{00} + \Gamma_{\mu 0}^\mu T^{00} + \Gamma_{ij}^0T^{ij}  \nonumber\\
&= \dot{\rho} +3 \frac{\dot{a}}{a} \rho + \dot{a}a \delta_{ij} p \frac{1}{a^2} \delta{ij} = \dot{\rho} + 3 \frac{\dot{a}}{a}\left( \rho +p \right) \nonumber ,  
\end{eqnarray}
d'o\`{u} 
\begin{equation}
\dot{\rho} + 3 \frac{\dot{a}}{a} \left(\rho +p \right) =0
\end{equation} 
ce qui peut \^etre mis sous la forme suivante:
\begin{equation}
   \frac{\partial}{\partial t} \left[\rho a^3 \right] +p \frac{\partial }{\partial t} a^3 =0,
\end{equation} 
c'est \`a dire 
\begin{equation}
dE+pdV=0.
\end{equation}
Ceci n'est rien d'autre que la premi\`ere loi de thermodynamique. 
\textbf{Forme \'equivalente des \'equations de Friedmann} 
Introduisons les notations suivantes:
\begin{eqnarray}
  \rho_\lambda := \frac{\lambda}{8\pi G} , \qquad \qquad \rho_{tot} := \rho + \rho_\lambda , \nonumber\\
  p_\lambda := - \frac{\lambda }{8\pi G} , \qquad \qquad p_{tot} := p+p_\lambda \nonumber
\end{eqnarray}

Ce sont la densit\'e et la pression associ\'ees \`a la constante cosmologique ainsi que la densit\'e et pression  totale provenant de la mati\`ere et de la constante cosmologique. Avec ces notations, les \'equations de Friedmann peuvent \^etre r\'e\'ecrites de la mani\`ere suivante :
\begin{eqnarray}
\frac{\dot{a}^2}{a^2} + \frac{k}{a^2}&=\frac{8\pi G}{3} \rho_{tot},\\
2\frac{\ddot{a}}{a} + \frac{\dot{a}}{a^2} + \frac{k}{a^2}&=-8\pi G p_{{tot}}.
\end{eqnarray}
On peut \'eliminer k en prenant la diff\'erence de ces deux \'equations 
\begin{equation}
\frac{\ddot{a}}{a}=-\frac{4\pi G}{3}\left(\rho_{tot}+3p_{tot} \right) 
\end{equation} 
En tout on d\'eriv\'e quatre \'equations : deux \'equations de Friedmann, une \'equation provenant de la conservation de l'\'energie-impulsion et l'\'equation qu'on vient d'obtenir. Parmi ces \'equations, il n'y a que deux qui sont lin\'eairement ind\'ependantes. Selon la nature du probl\`eme il peut \^etre plus utile de travailler avec une \'equation plut\^ot qu'avec une autre.\\

\textbf{Param\`etre de d\'ec\'el\'eration } \\
On d\'efinit le param\`etre de d\'ec\'el\'eration $q$ par 
\begin{equation}
q\left(t \right) : = -\frac{\ddot{a}}{aH^2}=-\frac{\ddot{a}a}{\dot{a}^2},
\end{equation}
qu'on peut exprimer en fonction des param\`etres de densit\'e observables :
\begin{equation}
q\left(t \right) =-\frac{\ddot{a}}{aH^2}=\frac{4\pi G}{3} \frac{\rho _{tot} + 3 p_{tot}}{\frac{8 \pi G}{3} \rho_{tot} - \frac{k}{a^2}}.
\end{equation}
\text{	Dans le cas d'un Univers plat $k=0$, on obtient} 
\begin{eqnarray}
\Rightarrow q\left(t \right) = \frac{\rho_{tot}+3p_{tot}}{2\rho_{tot}} 
\end{eqnarray}
En explicitant les diff\'erentes contributions \`a l'\'energie et \`a la pression \\
\begin{eqnarray}
\rho_{tot}=\rho_{mat}+\rho_{rad}+\rho_\lambda , \qquad p_{tot} =p_{rad}+p_{\lambda} ,
\end{eqnarray}
on peut \'ecrire
\begin{equation}
q\left(t \right)=\frac{\Omega_{mat}}{2} + \frac{1+3w_\lambda}{2} \Omega_\lambda ,
\end{equation}
o\`{u} on a introduit $w_\lambda=\frac{p_\lambda}{\rho_\lambda }$ et les fractions critiques $\Omega$. Le premier terme du membre de droite est toujours positif, mais le second peut \^etre n\'egatif si $ 1+3 w_\lambda < 0$ .Donc si $w_\lambda<-1/3$ et si le deuxi\`eme terme est assez grand, le param\`etre  de d\'ec\'el\'eration peut prendre une valeur n\'egative. Il d\'ecrit alors un Univers en expansion acc\'el\'er\'ee, et on l'appelle alors parfois param\`etre d'acc\'el\'eration. Les observations de Supernovae de type la sont en faveur d'un $w_\lambda < - \frac{1}{3}$ et notre Univers se trouve actuellement en expansion acc\'el\'er\'ee.
\section{Diff\'erentes solutions des \'equations de Friedmann}

\textbf{Univers statique sans constante cosmologique $ \dot{a}=0, \lambda =0$} \\

Dans ce cas les \'equations de Friedmann se r\'eduisent \`a 
\begin{eqnarray}
 \frac{k}{a^2} & =& \frac{8\pi G}{3} \rho ,  \cr
 \frac{k}{a^2}&=& - 8 \pi G p.
\end{eqnarray}

 Comme l'Univers n'est pas vide, on a $\rho >0$, et la premi\`ere \'equation impliquerait alors que $k>0$. Mais pour $k=1$, la deuxi\`eme \'equation impliquerait que la pression serait n\'egative $p<0$, ce qui n'a pas de sens. On doit donc conclure qu'il n'existe pas de solution consistante pour un Univers statique sans constante cosmologique. 
 
 
 Historiquement, Einstein avait d\'ej\`a r\'ealis\'e ce probl\`eme. Comme il croyait en l'existence d'un Univers statique il d\'ecida en 1917 de rajouter une constante cosmologique non-nulle aux \'equations. En 1929, quand Hubble mettais en \'evidence l'expansion de l'Univers, Einstein revenait sur l'introduction de la constante cosmologique, la qualifiant de "plus grande b\^etise de sa vie" . Des observations r\'ecentes sugg\`erent qu'il existe n\'eanmoins une constante cosmologique non-nulle, mais tr\`es petite. A l'heure actuelle on ne sait pas expliquer pourquoi elle est si faible; c'est le probl\`eme de la constante cosmologique.  \\
 
 \textbf{Univers statique avec constante cosmologique : $\dot{a}=0 , \lambda \neq 0$} \\
 Les \'equations de Friedmann s'\'ecrivent 
 \begin{eqnarray}
  \frac{k}{a^2}- \frac{\lambda}{3} & = &\frac{8\pi G}{3} \rho . \cr
  \frac{k}{a^2} -\lambda &=& -8 \pi Gp. \nonumber
 \end{eqnarray}
  Si les vitesses des \'etoiles sont faibles, on peut supposer que $p\simeq 0$. La deuxi\`eme \'equation implique alors 
  \begin{equation}
  \frac{k}{a^2}= \lambda ,
  \end{equation} 
  ce qui, inject\'e dans la premi\`ere \'equation, nous donne 
  \begin{equation}
  \lambda = 4 \pi G \rho 
  \end{equation}
   Comme $\rho >0$ on doit aussi avoir $\lambda >0$ et donc $k>0$. Pour $k=1$ on trouve pour $a$: 
   \begin{equation}
   a=\frac{1}{\sqrt{\lambda}} = \frac{1}{\sqrt{4\pi G \rho}}
   \end{equation} 
  Cette relation entre la densit\'e $\rho $ et la taille de l'Univers f\^ut d\'ej\`a d\'eriv\'ee par Einstein en 1917. Ce mod\`ele statique d'Einstein a malheureusement un probl\`eme conceptuel : si on suppose $\rho =0$, alors comme $\lambda \neq 0$, l'espace vide lui-m\^eme engendrerai une force de gravitation.\\
  \textbf{Univers vide statique: $\dot{a}=0 , \rho = p=0$} \\
  Les \'equations de Friedmann se r\'eduisent \`a : 
  \begin{eqnarray}
  \frac{k}{a^2} - \frac{\lambda}{3}& =&0 , \cr
  \frac{k}{a^2} - \lambda &=0&. 
  \end{eqnarray}
  On est alors confront\'e  au paradoxe suivant : si $\lambda =0$, alors $k=0$ et un Univers plat est donc solution. Mais on sait que dans ce cas il n'existe pas de solution statique. Pour avoir une solution statique il faut choisir $\lambda \neq 0$, donc $k \neq 0$ mais alors l'espace plat n'est plus une solution. Ce paradoxe f\^ut r\'esolu en 1922 quand Alexander Friedmann proposait un mod\`ele dans lequel l'Univers n'est plus statique.
  
  \textbf{Univers non-statique: $ \dot{a} \neq 0$} 
  
  Consid\'erons, pour simplifier, un Univers plat sans constante cosmologique:  $\lambda = k=0$.
  Les \'equations de Friedmann sont alors 
  \begin{eqnarray}
  \frac{\dot{a}^2}{a^2} & = &\frac{8\pi G}{3} \rho , \cr
  2 \frac{\ddot{a}}{a} + \frac{\dot{a}^2}{a^2} & =&-8 \pi G p\simeq 0. 
  \end{eqnarray}
  Multiplions la deuxi\`eme \'equation par $\frac{a}{\dot{a}}$ 
  \begin{eqnarray}
  && \implies 2 \frac{\ddot{a}}{\dot{a}} + \frac{\dot{a}}{a} =0 \implies \frac{d}{dt}
  \left( 2 \ln \dot{a} + \ln a \right)  =0 \cr
  && \implies \ln \left( \dot{a}^2a \right) = \text{ const } \implies \dot{a}^2a = \text{ const } \cr
  &&\implies \sqrt{a}da = \text{ const } . \; dt \implies a^{3/2} =\text{ const }.t.    
  \end{eqnarray} 
   D'o\`{u} 
   \begin{equation}
   a(t) = a_0 \left( \frac{t}{t_0} \right) ^{2/3}
   \end{equation} 
   On voit que $a$ augmente en fonction du temps- l'Univers est en expansion. On peut ensuite injecter cette solution dans la premi\`ere \'equation de Friedmann : 
   \begin{eqnarray}
   &  \implies \left( \frac{2}{3} \frac{1}{t} \right) = \frac{8\pi G}{3} \rho , \nonumber \\
   & \implies \rho = \frac{4}{9} \frac{3}{8 \pi G} \frac{1}{t^2}.
   \end{eqnarray}
   D'o\`{u} 
   \begin{equation}
   \rho (t) = \frac{1}{6 \pi G} \frac{1}{t^2}.
   \end{equation}
   
   En particulier, connaissant la densit\'e $\rho$, on peut d\'eterminer l'\^{a}ge de l'Univers. Constatons aussi que pour $t \rightarrow 0 $, la densit\'e diverge $\rho \rightarrow \infty $ (Big Bang). 
  
 \section{Equations de Friedmann en m\'ecanique Newtonienne} 
 
 Il s'av\`ere que l'on peut d\'eriver l'essentiel des \'equations de Friedmann d\'ej\`a en m\'ecanique Newtonienne et sans avoir recours \`a la relativit\'e g\'en\'erale. Consid\'erons le cas suivant 
  	$ k=0$, espace plat , 
  	$\lambda =0 $ , le vide ne produit pas de force gravitationnelle, 
  	$  p=0$, mouvement non-relativiste $(p<< \rho ). $ 
Consid\'erons une sph\`ere de rayon variable $a(t)$ remplie d'un gaz de particules sans interaction distribu\'ees de fa\c con homog\`ene et isotrope. D'une part on peut \'ecrire la conservation de l'\'energie totale \`a l'int\'erieur de cette sph\`ere :
 \begin{equation}
\frac{d}{dt}\left(  \frac{4}{3} \pi a^3 . \rho  \right) =0 
 \end{equation} 
 En d\'eveloppant on retrouve une des \'equations de Friedmann : 
 \begin{equation}
 \dot{\rho} + 3 \frac{\dot{a}}{a} \rho =0. 
 \end{equation}
 D'autre part on peut \'ecrire la loi de Newton pour une particule de gaz de masse $m$ situ\'ee sur la sph\`ere :
 \begin{equation}
 m \ddot{a} = -G . \frac{4}{3} \pi a^3 \rho .m. \frac{1}{a^2}
 \end{equation}
  Ce qui implique 
  \begin{equation}
  \frac{\ddot{a}}{a} = - \frac{4 \pi G}{3} \rho 
  \end{equation} 
  On a retrouv\'e l'autre \'equation de Friedmann dans le cas particulier $p_{tot} =0$ et $\rho_{tot}=\rho $. Pour trouver le terme $+3 p_{tot}$ dans cette \'equation il faut utiliser la relativit\'e g\'en\'erale.
  \section{Propagation de la lumi\`ere dans l'Univers- D\'ecalage vers le rouge}  
  On aimerait \'etudier la propagation de la lumi\`ere dans un Univers d\'ecrit par les \'equations de Friedmann. Le moyen direct serait d'\'etudier les \'equations de Maxwell dans l'espace courbe de Robertson-Walker. Mais il y a en fait un moyen plus facile qui est de faire usage d'un syst\`eme de coordonn\'ees particulier, appelle coordonn\'ees conformes. On r\'e\'ecrit la m\'etrique de Robertson-Walker comme suit 
  \begin{eqnarray}
  ds^2 & =& dt^2 + a^2(t) dl^2 \cr
  &= &a^2(t) \left[ \frac{dt^2}{a^2(t)} +dl^2 \right]   
  \end{eqnarray} 
  On introduit le temps conforme $\eta $ d\'efinit par 
  \begin{equation}
  d\eta := \frac{dt}{a(t)},
  \end{equation} 
  donc 
  \begin{equation}
  \eta - \eta_0 = \int_{0}^{t} dt^\prime \frac{1}{a(t^\prime)} 
  \end{equation}
  La m\'etrique s'\'ecrit alors 
  \begin{equation}
  ds^2 = a^2(\eta) \left[ d\eta ^2 +dl^2 \right] \equiv a^2(\eta) \bar{g}_{\mu \nu } dx^\mu dx^ \nu.
  \end{equation}
  La m\'etrique $\bar{g}_{\mu \nu} $ ainsi d\'efinie est ind\'ependante du temps. On peut maintenant \'etudier les \'equations de Maxwell dans ce syst\`eme de coordonn\'ees. Nous allons partir de l'action de l'\'electromagn\'etisme dans un espace courbe 
  \begin{eqnarray}
  S_{EM} = \displaystyle \int d^4 x \sqrt{|g|}
  \left[ - \frac{1}{4} g^{\mu \nu } g^{\rho \sigma } F_{\mu  \rho } F_{\nu \sigma } \right]. 
  \end{eqnarray}
  Pour le syst\`eme de coordonn\'ees de temps conforme on a 
  \begin{eqnarray}
   \sqrt{|g|} = \sqrt{\left[ a^2(\eta) \right]^4  } \sqrt{|\bar{g}|} =a^4 (\eta ) \sqrt{|\bar{g}|}, 
   \end{eqnarray}
   et 
   \begin{eqnarray}
   g^{\mu \nu } = \frac{1}{a^2(\eta )} \bar{g}^{\mu \nu } 
   \end{eqnarray}  
   D'o\`{u} 
   \begin{eqnarray}
   S_{EM} - \frac{1}{4} \displaystyle \int d^4x a^4(\eta) \sqrt{|\bar{g}|}. \frac{1}{a^2(\eta)} \bar{g}^{\mu \nu} . \frac{1}{a^2(\eta)} \bar{g}^{\rho \sigma } . F_{\mu \rho } F_{\nu \sigma}, 
   \end{eqnarray}
   ou encore 
   \begin{equation}
   S_{EM} = - \frac{1}{4} \displaystyle \int d^4x \sqrt{|\bar{g}|} \bar{g}^{\mu \nu } \bar{g}^{\rho \sigma } F_{\mu \rho } F_{\nu \sigma } 
   \end{equation}
   Comme $\bar{g}_{\mu \nu }$ est statique, cette action ne d\'epend pas explicitement du temps. De plus, si on consid\`ere des "petites" distances $(\bar{r}<<1)$, on peut supposer que l'espace est pratiquement plat $(k=0)$ et utiliser pour $\bar{g}_{\mu \nu }$ la m\'etrique de Minkowski: 
   \begin{eqnarray}
   ds^2 \simeq a^2 ( \eta) \eta_{\mu \nu } dx^\mu dx^\nu.
   \end{eqnarray}
   Dans ce cas, les solutions des \'equations de Maxwell sont simplement des ondes planes 
   \begin{equation}
   A_\mu \propto \mathrm{e}^{i \omega \eta +ik \bar{x}},
   \end{equation} 
   o\`{u} $\omega , k = \text{const} $ et $ \bar{x}, \eta , \omega , k $ sont sans dimension. 
   
   Ayant trouv\'e la solution des \'equations de Maxwell, on doit encore l'interpr\'eter physiquement. Rappelons qu'on a utilis\'e les coordonn\'ees suivantes 
   \begin{eqnarray}
   ds^2 =a^2(\eta ) \left[ d \eta ^2 -d\bar{x}^2 \right]   
   \end{eqnarray}
   
      	Un observateur de cette onde \'electromagn\'etique utilisera simplement la m\'etrique de Minkowski
      	\begin{eqnarray}
      	ds^2=dt^2-dx^2
      	\end{eqnarray}
      	On obtient ainsi la relation 
      	\begin{eqnarray}
      	a^2\left(\eta \right)d\bar{x}^2=dx^2. 
      	\end{eqnarray}
      	Ainsi l'observateur va en fait observer une onde plane donn\'ee par
      	\begin{eqnarray}
      	A_\mu \propto e^{iw\eta +ik\frac{x}{a\left(\eta \right) }}
      	\end{eqnarray}
      	Cet observateur va donc mesurer comme longueur d'onde 
      	\begin{equation}
      	\lambda = \frac{2\pi a\left(t \right) }{k}
      	\end{equation}
      	On conclut que $\lambda\propto a\left( t \right) $. Dans un Univers en expansion, le facteur d'\'echelle $a \left(t \right)$ croit, et donc aussi la longueur d'onde de la lumi\`ere observ\'ee. On peut illustrer cette augmentation de la longueur d'onde en s'imaginant de dessiner une onde sur un ballon et de le gonfler; la longueur va augmenter  avec le rayon du ballon.	
\begin{figure}[h]
%\begin{minipage}{14pc}
\includegraphics[width=15pc]{t2.png}
\caption{Augmentation de la longueur d'onde \`a cause de l'expansion de l'Univers}
% \includegraphics[width=15pc]{w_t2.eps}
% \caption{}
\end{figure}
      	\qquad Consid\'erons une source lumineuse (e.g.\'etoile) \'emettant de la lumi\`ere de longueur d'onde $\lambda_0$ \`a une distance $l$ de la terre. Calculons la longueur d'onde $\lambda$ qu'on observerait sur la terre. Soit $t$ le temps de r\'eception du signal sur terre et soit $t_0\simeq t-\frac{l}{c}$ le temps d'\'emission du signal lumineux (cette relation n'est qu'approximative car l'Univers est en expansion). Comme $\lambda \propto a$ on a 
      	\begin{eqnarray}
      	\frac{\lambda_0}{\lambda}=\frac{a\left( t_0\right) }{a\left(t \right) }\Rightarrow \lambda\left( t\right) =\lambda_0 \frac{a\left( t\right) }{a\left( t-\frac{l}{c}\right) }
      	\end{eqnarray}
      	Comme $\frac{l}{c}$ est petit, on peut d\'evelopper en s\'erie 
      	\begin{eqnarray}
      	\lambda\left( t\right) \simeq\lambda_0 \frac{a\left( t\right) }{a\left( t\right) \left( t-\frac{l}{c}\right) } \simeq \lambda_0\left(1+\frac{\dot{a}}{a} \frac{l}{c} \right) 
      	\end{eqnarray}
      	
      	On appelle d\'ecalage vers le rouge ("redshift"), not\'e $z$, le rapport 
      	\begin{equation}
      	z:= \frac{\lambda- \lambda_0}{\lambda_0} \\
      	\end{equation}
      	D'o\`{u} finalement (avec $c=1$)
      	\begin{equation}
      	z=\frac{\dot{a}}{a}l= Hl,
      	\end{equation}
      	o\`{u} $H:=\frac{\dot{a}}{a}$ est appel\'e constante de Hubble, m\^{e}me si elle n'est  pas vraiment une $constante $ vu qu'elle d\'epend du temps. On note par $H_0$ la constante de Hubble \`a notre \'epoque, donc $H_0=H\left(t_0 \right) $. La relation entre le d\'ecalage vers le rouge et la constante de Hubble est commun\'ement appel\'ee la loi de Hubble. Elle montre que, dans un Univers en expansion, les longueurs d'ondes sont d\'eplac\'ees vers le rouge . En 1929, Edwin Hubble a observ\'e en premier cette relation lin\'eaire entre la distance $l$ et le d\'ecalage vers le rouge $z$. C'\'etait une premi\`ere indication exp\'erimentale pour l'expansion de l'Univers. Ci- apr\`es se trouvent deux diagrammes de Hubble. Ils montrent la vitesse de la source en fonction de sa distance. Surtout le diagramme r\'ecent de 2005 met bien en \'evidence une d\'ependance lin\'eaire. Actuellement, les mesures du param\`etre de Hubble donnent  :
      	\begin{equation}
      	H_0= 71 \pm \frac{km/s}{Mpc}
      	\end{equation}
      	
%       \begin{figure}[h]
% %\begin{minipage}{14pc}
% \includegraphics[width=15pc]{t3.png}
% \caption{Diagramme publi\'e par Edwin
% Hubble dans son article de $1929$ \cite{1}}
%  \includegraphics[width=15pc]{t4.png}
%  \caption{Diagramme de Hubble (pour
% les supernovae) de $2005$ \cite{2}}
% \end{figure}
      	
      	 \begin{figure}[h]
\begin{minipage}{14pc}
\includegraphics[width=15pc]{t3.png}
\caption{Diagramme publi\'e par Edwin
Hubble dans son article de $1929$ \cite{1}}
\end{minipage}\hspace{3pc}%
\begin{minipage}{14pc}
\includegraphics[width=15pc]{t4.png}
 \caption{Diagramme de Hubble (pour
les supernovae) de $2005$ \cite{2}}
 \end{minipage}\hspace{3pc}%
\end{figure}
      	
      	
      	On peut constater dans le diagramme de Hubble de 1929 que l'unit\'e pour la vitesse est fausse (km au lieu km/s). En plus, la valeur pour la constante $H_0$ d\'etermin\'ee par Hubble lui-m\^{e}me \'etait fausse d'un facteur $\sim 10$.
      	
      	
      	\qquad Exp\'erimentalement il est relativement ais\'e de mesurer le d\'ecalage vers le rouge des \'etoiles. Pour cela il suffit de mesurer le d\'eplacement des lignes spectrales dans le spectre de la lumi\`ere re\c{c}ue. La mesure des distances par contre est plus d\'elicates. L'id\'ee est de d\'eduire la distance de l'objet \`a partir de la luminosit\'e. Mais pour cela il faut faire usage de ce qu'on appelle des chandelles standards, qui est un d'objet astrophysique \`a luminosit\'e connue. Il est alors facile de relier la luminosit\'e de ces chandelles standards et la luminosit\'e observ\'ee sur terre \`a leur distance.\\
      	
      	$\qquad$ Il existe une interpr\'etation \'equivalente de la loi de Hubble \`a l'aide de l'effet Doppler. On consid\`ere que l'observateur est au repos, mais que la source s'\'eloigne de lui \`a ue vitesse $v$. Alors, par effet Doppler, la longueur d'onde observ\'ee sera diff\'erente de celle \'emise  plus particuli\`erement on aura :
      	\begin{eqnarray}
z=\sqrt{\frac{1+v}{1-v}}-1 \overset{v<<c}{\simeq} v = \dot{l} = \frac{\dot{a}}{a} l, 
\end{eqnarray}
      	c'est \`a dire
     	\begin{eqnarray} 
     	\dot{\overrightarrow{l}} =H \overrightarrow{l}.
     	\end{eqnarray}
      	Cette relation motive aussi le choix d'unit\'es pour la constante de Hubble :
      	\begin{equation}
      	[H]=\frac{[\dot{l}]}{[l]}= \frac{km/s}{Mpc}
      	\end{equation} 
      	$\qquad$ On a vu que l'homog\'en\'eit\'e et l'isotropie de l'Univers impliquent la loi de Hubble. \'{E}tudions si la r\'eciproque est vraie aussi. Il est clair que la loi de Hubble implique que l'Univers est isotrope, vu que $H$ ne d\'epend pas de la direction d'observation. Il s'av\`ere que la loi de Hubble implique aussi que l'Univers est homog\`ene.
      	En effet, on a 
      	
      	     \begin{figure}[h]
%\begin{minipage}{14pc}
\includegraphics[width=15pc]{t5.png}
\caption{}
%  \includegraphics[width=15pc]{t4.png}
%  \caption{Diagramme de Hubble (pour
% les supernovae) de 2005 [2]}
\end{figure}
      	
      	mais aussi pour la vitesse du point $B$ par rapport au point $A$
      	\begin{eqnarray}
      	v_B ) _A &=&H\vec{r}_{AB} ,\cr
      	v_B ) _A &= &\vec{v}_B - \vec{v}_A =H\left( \vec{r}_B - \vec{r}_A\right). 
      	\end{eqnarray}
      	Les deux derni\`eres \'etant \'egales, on conclut qu'il y a homog\'en\'eit\'e.
    \section{Horizons} 
    
    L'Univers ayant un \^{a}ge fini, la lumi\`ere dans l'univers n'a p\^u parcourir qu'une distance finie. Il s'en suit que tr\`es probablement nous ne pouvons observer qu'une partie de notre Univers. De plus, si l'expansion de l'Univers est trop grande, cette partie visible d\'evient de plus en plus petite, vu que la lumi\`ere des r\'egions les plus lointaines ne peut plus nous atteindre. Pour discuter ces ph\'enom\`enes nous allons maintenant introduire la notion d'horizon. 

    \textbf{Horizon d'\'ev\`enement} 
    
     Le premier horizon  que nous introduisons est l'horizon d'\'ev\`enement ("event horizon"). Il correspond au rayon de la r\'egion de l'Univers dans le pass\'e qui peut nous influencer causalement. Tout \'ev\`enement en dehors de cet horizon ne peut pas nous influencer vu que seuls les signaux dans notre horizon d'\'ev\`enement peuvent nous atteindre. 
     
     
     Nous voulons calculer la distance qu'un photon peut parcourir s'il est \'emis \`a l'instant $t$. Supposons que le mouvement du photon se situe dans un plan et posons en cons\'equence dans la m\'etrique de Robertson-Walker $ds^2 =0 , d\theta =d \phi =0$, ce qui donne 
     \begin{eqnarray}
     \frac{dt}{a} = \frac{dr}{\sqrt{1-kr^2}} 
     \end{eqnarray}
     La distance en coordonn\'ees comobiles est alors donn\'ee par 
     \begin{eqnarray}
     D_e(t) = \displaystyle \int_{t}^{t_0} \frac{dt^\prime }{a(t^\prime )}. 
     \end{eqnarray}
 Pour passer \`a une distance physique il suffit de multiplier la distance exprim\'ee en coordonn\'ees comobiles par le facteur d'\'echelle : 
 \begin{eqnarray}
 d_e(t) =a(t) \displaystyle \int_{t}^{t_0} \frac{dt^\prime }{a(t^\prime )}.
 \end{eqnarray}
 Remarquons encore que toutes ces formules n'ont de sens que si les int\'egrales convergent. Si tel n'est pas le cas, on dit que l'horizon en question n'existe pas. 
 
 \textbf{Horizon de particule}  
 
 Le second horizon est appel\'e horizon de particule ("particle horizon"). Il s'agit de conna\^{\i}tre l'\'etendue de la r\'egion \`a laquelle nous sommes causalement reli\'es \`a l'instant $t_0$. En coordonn\'ees comobiles nous aurons 
 
 \begin{eqnarray}
 D_p(t_0) = \displaystyle \int_{t_{min}}^{t_0} \frac{dt}{a(t) }, 
 \end{eqnarray}

 et donc la distance physique est 
 \begin{eqnarray}
 d_p (t_0) =a(t_0) \displaystyle \int_{t_{min}}^{t_0} \frac{dt}{a(t)}.
 \end{eqnarray}

%      \begin{figure}[h]
% %\begin{minipage}{14pc}
% \includegraphics[width=15pc]{t6.png}
% \caption{Horizon d'\'ev\'enement}
%  \includegraphics[width=15pc]{t8.png}
%  \caption{Horizon de particule}
% \end{figure}


\begin{figure}[h]
\begin{minipage}{14pc}
\includegraphics[width=15pc]{t6.png}
\caption{Horizon d'\'ev\'enement}
\end{minipage}\hspace{3pc}%
\begin{minipage}{14pc}
\includegraphics[width=15pc]{t8.png}
\caption{Horizon de particule}
\end{minipage}\hspace{3pc}%
\end{figure}

\section{Densit\'e critique de l'Univers} 


  Jusqu'\`a pr\'esent notre \'etude s'est limit\'ee au cas particulier $ \lambda =0 , p=0 , k=0$.  Afin de pouvoir faire une \'etude plus g\'en\'erale, commen\c cons par r\'e\'ecrire une des \'equations de Friedmann: 
  \begin{eqnarray}
  H^2 + \frac{k}{a^2} &=& \frac{8 \pi G}{3} \rho_{tot} \cr
  \implies \frac{k}{H^2a^2} &= &\frac{8\pi G \rho_{tot}}{3H^2} -1 \equiv \frac{\rho_{tot}-\rho_c}{\rho_c} 
  \end{eqnarray}   
  o\`{u} on a introduit la densit\'e critique d\'efinie par 
  \begin{equation}
  \rho_c := \frac{3H^2}{8\pi G} .
  \end{equation}
  La densit\'e critique peut \^etre d\'eduite de la constante de Hubble: 
  
  \begin{eqnarray}
  \rho_c \approx 1,88.h^2.10^{-29} \frac{g}{cm^3},
  \end{eqnarray}

  o\`{u} on a \'ecrit la constante de Hubble de la mani\`ere suivante 
  \begin{eqnarray} 
  H =100.h. \frac{km/s}{Mpc}
  \end{eqnarray}

  De plus, l'\'equation de Friedmann implique 
  \begin{equation}\label{31}
  sign(k)= sign(\rho_{tot}-\rho_c),
  \end{equation}
  ainsi 
 \begin{eqnarray}
  &&\rho_{tot} > \rho_c \implies k=1 \qquad \text{ Univers ferm\'e, } \cr
  &&\rho_{tot} = \rho_c \implies k=0 \qquad \text{ Univers plat, } \cr
  &&\rho_{tot} < \rho_c \implies k=-1 \qquad  \text{ Univers ouvert } 
  \end{eqnarray}
  La densit\'e critique n'est pas forc\'ement constante et peut d\'ependre du temps. Mais d'apr\`es  \ref{31} le signe de $\rho_{tot}-\rho_c$ est ind\'ependant du temps>. Les observations r\'ecentes sugg\`erent que 
  \begin{eqnarray}
  \rho_{tot} \simeq \rho_c \pm (2-3)  \% 
  \end{eqnarray}
  notre Univers est donc extr\^emement proche d'un Univers plat. 
  
  Il est utile d'introduire encore une autre notation ; les abondances $\Omega $, aussi appel\'ees les fractions critiques: 
  \begin{equation}
  \Omega_{mat} := \frac{\rho_{mat}}{\rho_c} , \; \Omega_\lambda 
  := \frac{\rho_\lambda }{\rho_c} , \; \Omega_k := - \frac{k^2}{a_0^2H_0^2}
  \end{equation}  
  
  Dans ces notations, l'\'equation de Friedmann prend une forme particuli\`erement simple 
  \begin{equation}
  \Omega_{mat} + \Omega_\lambda +\Omega_k =1.
  \end{equation} 
  \section{Le futur de l'Univers}
  On a vu que notre Univers se trouve dans un \'etat d'expansion. Mais que peut-on dire sur le futur de l'Univers? Consid\'erons la situation $ \lambda =0 , p=0$. La valeur de $k$ n'\'etant pas fix\'ee, prenons la forme des \'equations de Friedmann qui est ind\'ependante de $k$ 
  \begin{eqnarray} 
  \frac{\ddot{a}}{a} = - \frac{4 \pi G}{3} \rho ,
  \end{eqnarray}

  et multiplions par $a \dot{a} $ 
  \begin{equation}\label{34}
  \dot{a} \ddot{a} = \frac{1}{2} \frac{d}{dt} (\dot{a}^2) = - \frac{4\pi G}{3} \rho a \dot{a} 
  \end{equation}
 On aimerait r\'e\'ecrire $\rho a \dot{a}$ comme d\'eriv\'ee totale par rapport au temps. Pour cela, constatons d\'ej\`a qu'on a trivialement 
 \begin{equation}
 \frac{d}{dt} (\rho a^2)  = -3\rho a \dot{a} +2 \rho a \dot{a} =-\rho a\dot{a}
 \end{equation} 
 On peut \`a pr\'esent r\'e\'ecrire l'\'equation  \ref{34} comme d\'eriv\'ee totale et l'int\'egrer par rapport au temps:
 \begin{eqnarray}
 \frac{1}{2} \frac{d}{dt} (\dot{a}^2) = \frac{4\pi G}{3} \frac{d}{dt} (\rho a^2)
 \end{eqnarray}
  
 \begin{eqnarray}
 && \implies \frac{d}{dt} \left( \frac{1}{2} \dot{a}^2 - \frac{4 \pi G}{3} \rho a^2 \right)=0 \cr
 & &\implies \frac{1}{2} \dot{a}^2 - \frac{4\pi G}{3} \rho a^2 = \text{ const } \overset{t=t_0}{=} \frac{1}{2} \ddot{a}_0^2 - \frac{4 \pi G}{3} \rho_0 a_0^2  
 \end{eqnarray}
 De plus 
 \begin{eqnarray}
 \rho a^3 = \text{ const } &= &\rho_0 a_0^3 \implies \rho a^2 = \frac{\rho_0 a_0^3}{a} \cr
  H_0& =& \frac{\dot{a_0}}{a_0} \implies \dot{a}_0^2 H_0^2a_0^2,
 \end{eqnarray} 
 ce qui, inject\'e dans l'\'equation d'avant donne 
 \begin{eqnarray}
 \implies \dot{a}^2 &=& \frac{ 8 \pi G}{3} \frac{\rho_0 a_0^3}{a} + H_0^2 a_0^2 - \frac{ 4 \pi G }{3} \rho_0 a_0^2 \cr
 \implies \dot{a}^2 &=& \frac{8 \pi G }{3} \frac{\rho_0 a_0^3}{a} -  \frac{4 \pi G}{3} a_0^2 (\rho_0 -\rho_{c,0}) 
 \end{eqnarray}
 L'\'equation \`a r\'esoudre est donc 
 \begin{equation}
 \dot{a}^2 - \frac{8\pi G }{3} \frac{\rho_0 a_0^3}{a} =
 - \frac{4 \pi G}{3} a_0^2 (\rho_0 - \rho_{c,0})= \text{ const },
 \end{equation} 
 qui est clairement de la forme $E_{ein} +U =E_{tot} \text{ const } $. La solution exacte de cette \'equation est donn\'ee par : 
 \begin{equation}
 t =\pm  \displaystyle \int \frac{da}{\sqrt{\frac{ 8 \pi G}{3} \frac{\rho_0 a_0^3}{a} -
 \frac{4 \pi G }{3} a_0^2 (\rho_0 -\rho_{c,0})}}.
 \end{equation}
 Essayons de comprendre qualitativement les solutions en utilisant l'analogie avec la m\'ecanique classique. 
 
%       \begin{figure}[h]
% %\begin{minipage}{14pc}
% \includegraphics[width=15pc]{t9.png}
% \caption{Le potentiel en fonction de $a$.}
% %  \includegraphics[width=15pc]{t4.png}
% %  \caption{Diagramme de Hubble (pour
% % les supernovae) de 2005 [2]}
% \end{figure}
 \begin{figure}[h]
\begin{minipage}{14pc}
\includegraphics[width=15pc]{t9.png}
\caption{Le potentiel en fonction de $a$.}
\end{minipage}\hspace{3pc}%
% \begin{minipage}{14pc}
% \includegraphics[width=15pc]{t13.png}
% \caption{}
% \end{minipage}\hspace{3pc}%
\end{figure}

%  
Il faut distinguer les cas suivants: 
\begin{enumerate}
	\item[i)] $\rho_0 =\rho_{c,0}$ , i.e. $E_{tot}=0:$ \\

	Dans ce cas la solution est simple et on l'a d\'ej\`a trouv\'e avant : $a \sim t^{2/3}$, \\$\implies $ expansion infinie. 
	\item[ii)] $\rho_0 > \rho_{c,0} $, i.e. $E_{tot} <0$:\\
	$a_{\max} = \frac{2a_0 \rho_0}{\rho_0 - \rho_{c,0}}$ 
	
% 	  \begin{figure}[h]
% %\begin{minipage}{14pc}
% \includegraphics[width=15pc]{t10.png}
% \caption{}
%   \includegraphics[width=15pc]{t11.png}
%   \caption{}
% \end{figure}
	
	\begin{figure}[h]
\begin{minipage}{14pc}
\includegraphics[width=15pc]{t10.png}
\caption{}
\end{minipage}\hspace{3pc}%
\begin{minipage}{14pc}
\includegraphics[width=15pc]{t11.png}
\caption{}
\end{minipage}\hspace{3pc}%
\end{figure}

$\implies $ collapse de l'Univers. 
\item[iii)] $\rho_0 < \rho_{c,0}$ i.e. $E_{tot} >0:$ \\
$ \implies$ expansion infinie. 


%   \begin{figure}[h]
% %\begin{minipage}{14pc}
% \includegraphics[width=15pc]{t12.png}
% \caption{}
%   \includegraphics[width=15pc]{t13.png}
%   \caption{}
% \end{figure}
% 


\begin{figure}[h]
\begin{minipage}{14pc}
\includegraphics[width=15pc]{t12.png}
\caption{}
\end{minipage}\hspace{3pc}%
\begin{minipage}{14pc}
\includegraphics[width=15pc]{t13.png}
\caption{}
\end{minipage}\hspace{3pc}%
\end{figure}



En conclusion, le futur de l'Univers d\'epend de son contenu. S'il y a beaucoup de mati\`ere, l'Univers va s'effondrer ("Big Crunch"), autrement il y aura une expansion   infinie. Les observations astronomiques actuelles favorisent une expansion infinie de l'Univers.
\end{enumerate}
 
 
  
\chapter{L'acc\'el\'eration de l'Univers et mod\`eles de gravit\'e}\label{chapitre4}
\minitoc


Ce chapitre est consacr\'e \`a l'acc\'el\'eration de l'Univers, et aux mod\`eles physiques qui
permettent de l'expliquer. Apr\`es avoir rappel\'e les \'el\'ements de la cosmologie dont nous 
aurons besoin par la suite, nous d\'etaillerons les principales observations qui d\'emontrent que
l'Univers est actuellement dans une phase d'expansion acc\'el\'er\'ee. Nous pr\'esenterons ensuite
un panorama des mod\`eles ayant pour objectif d'expliquer cette acc\'el\'eration. Nous verrons
que certains mod\`eles se basent sur la Relativit\'e G\'en\'erale et la pr\'esence d'\'energie noire,
tandis que d'autres proposent un changement plus radical de paradigme, en modifiant les
lois de la gravit\'e.
\section{Le mod\`ele cosmologique standard}

\subsection{M\'etrique de FLRW }
Le mod\`ele cosmologique standard se base sur le fait que l'Univers autour de nous
appara\^it comme \'etant isotrope \`a grandes \'echelles, tant au niveau de la r\'epartition des
galaxies que du fond diffus cosmologique. Si l'on suppose de plus que notre Terre n'occupe
pas de place privil\'egi\'ee dans l'Univers, on en arrive au principe cosmologique qui fait
l'hypoth\`ese que l'Univers est spatialement homog\`ene et isotrope \`a grandes \'echelles.
La mod\'elisation math\'ematique de cette hypoth\`ese nous am\`ene \`a d\'ecrire l'espace-temps
par une m\'etrique de Friedmann-Lema\^itre-Robertson-Walker (FLRW) $ g_{\mu \nu} $ dont la forme
g\'en\'erale \cite{215, 277, 282} s'\'ecrit

\begin{eqnarray}
 ds^2 \equiv g_{ \mu \nu } dx^\mu dy^\nu = - dt^2  + a^2(t)[ d\chi^2 + f^2_K(\chi)(d\theta^2 + sin^2 \theta d\phi^2 ) ] \label{gene1} 
\end{eqnarray}

o\`u $ \chi $ est la coordonn\'ee radiale. La constante $K$ d\'ecrit la g\'eom\'etrie de la section spatiale
de l'espace-temps : l'espace est ferm\'e pour $ K > 0$, plat pour $K = 0$ et enfin ouvert pour
$K < 0$. La fonction $  f_K $ , qui est telle que la surface d'une sph\`ere de rayon $ \chi $ est donn\'ee par
    $ S(\chi ) = 4 \pi f_K^2(\chi) $, prend la forme 
   
   \begin{eqnarray}
   \qquad
   f_K(\chi) =  \begin{cases}
   K^{- \frac{1}{2}} sin( \sqrt{ K} \chi )  &\text{si }  K > 0 \\    \chi &\text{si } K = 0 \\ (-K)^{- \frac{1}{2}} sinh( \sqrt{ -K} \chi )
   &\text{si } K < 0
\end{cases}
    \end{eqnarray}

    La fonction $a(t)$ s'appelle le facteur d'\'echelle et caract\'erise l'\'evolution de l'Univers. On
normalise $ a(t) $ de telle sorte qu'aujourd'hui $ a \equiv a_0 = 1 $ (nous noterons avec un indice 0
la valeur des quantit\'es cosmologiques \'evalu\'ees aujourd'hui). Sa d\'ependance dans le temps
est d\'etermin\'ee en r\'esolvant les \'equations de la dynamique.

\subsection{Relativit\'e G\'en\'erale et \'equations de Friedmann}
Pour aller plus loin, il nous faut choisir une th\'eorie de la gravitation \`a partir de 
laquelle nous pourrons d\'eriver les \'equations de la dynamique pour le facteur d'\'echelle. La
cosmologie standard est bas\'ee sur la Relativit\'e G\'en\'erale $(RG)$ ; cependant, dans
cette th\`ese, nous travaillerons \'egalement avec d'autres th\'eories alternatives de la gravit\'e.
%(gravit\'e massive, mod\`ele $DGP$, th\'eories scalaire-tenseur, etc.) 
D'une th\'eorie \`a l'autre, les
\'equations de la dynamique varient, modifiant ainsi la cosmologie et l'interpr\'etation que
l'on peut faire des observations. C'est pr\'ecis\'ement l\`a tout l'int\'er\^et des th\'eories de gravit\'e
modifi\'ee, qui peuvent permettre de rendre compte des observations d'une fa\c{c}on diff\'erente
de la RG. Pour l'instant, nous voudrions
pr\'esenter le mod\`ele cosmologique standard, qui fait l'hypoth\`ese que la gravit\'e est d\'ecrite
par la RG.

L'action de la RG s'\'ecrit

\begin{eqnarray}
 S = \frac{M_P^2}{2} \int d^4 x \sqrt{- g} R [g]  + S_{\text {mati\`ere} } [ g],
\end{eqnarray}
o\`u $ R[g]$ est le scalaire de Ricci et $ M_P $ est la masse de Planck qui est reli\'ee \`a la constante de
Newton $G$ \`a travers  la relation $ M_P^{ -2} = 8 \pi G $. La mati\`ere est suppos\'ee minimalement
coupl\'ee \`a la m\'etrique.

La variation de l'action de la RG par rapport \`a la m\'etrique m\`ene aux \'equations d'Einstein

\begin{eqnarray}
 G_\mu^\nu = 8 \pi G T_\mu^\nu, \label{gene4}
\end{eqnarray}
o\`u $ G_\mu^\nu \equiv R_\mu^\nu - \frac{1}{2} \delta_\mu^\nu R  $ est le tenseur d'Einstein et $T_\mu^\nu $ le tenseur \'energie-impulsion de la
mati\`ere
\begin{eqnarray}
 T_{\mu \nu }(x) \equiv - \frac{2}{\sqrt{- g} }\frac{ \delta }{ \delta g^{\mu \nu } (x) } S_{\text{mati\`ere} } [g].
\end{eqnarray}
Les identit\'es de Bianchi garantissent que la mati\`ere est conserv\'ee, c'est-\`a-dire que
\begin{eqnarray}
 \Delta_\nu T_\mu^\nu = 0.
\end{eqnarray}
L'hypoth\`ese d'homog\'en\'eit\'e et d'isotropie, ainsi que la forme ( \ref{gene1}) de la m\'etrique, 
impose au tenseur \'energie-impulsion de la mati\`ere d'\^etre celui d'un fluide parfait
    
\begin{eqnarray}
 T_\mu^\nu = diag( -\rho, P, P, P),
\end{eqnarray}
o\`u $ \rho $ est la densit\'e de la mati\`ere, et $P$ sa pression. Il est ensuite ais\'e d'obtenir, \`a partir
de l'expression ( \ref{gene1}) de la m\'etrique de $FLRW$ et des \'equations d'Einstein ( \ref{gene4}), les deux
\'equations ind\'ependantes dites de Friedmann
\begin{eqnarray}
 H^2 = \frac{8 \pi G \rho }{ 3 } - \frac{K}{ a^2 }, \label{gene5} \\ \frac{\ddot{a }}{a } = -\frac{4 \pi G }{3} ( \rho + 3 P), \label{gene5'}
\end{eqnarray}
o\`u $ H \equiv \dot{a} / a $ est le param\`etre de Hubble et un point repr\'esente une d\'eriv\'ee par rapport au
temps cosmique t. L'\'equation de conservation de la mati\`ere prend quant \`a elle la forme
\begin{eqnarray}
 \dot{ \rho} + 3H(\rho + P) = 0 \label{gene6}
\end{eqnarray}
Il est usuel de r\'e\'ecrire l'\'equation ( \ref{gene5}) sous la forme r\'eduite
\begin{eqnarray}
 \sum_{\substack{i}} \Omega_i + \Omega_K = 1
\end{eqnarray}
o\`u l'indice $i$ correspond aux diff\'erents types de mati\`ere (radiation, mati\`ere noire, \'energie
noire, etc.) et o\`u l'on a d\'efini
\begin{eqnarray}
 \Omega_i \equiv \frac{8 \pi G \rho_i}{3 H^2},  \\ \Omega_K \equiv - \frac{K}{a^2 H^2}
\end{eqnarray}

Les observations \cite{177} montrent que la courbure spatiale est tr\`es proche de z\'ero, 
conform\'ement aux pr\'edictions g\'en\'eriques des mod\`eles d'inflation (cf. par exemple \cite{215}
; nous consid\'ererons donc par la suite dans ce m\'emoire que l'Univers est plat et fixerons
par cons\'equent $ K = 0 $.
 
 \subsection{Equation d'\'etat de la mati\`ere et \'evolution de l'Univers
%\'equation d'\'etat
}

La description la plus simple de la mati\`ere consiste \`a supposer que la densit\'e et la
pression de la mati\`ere suivent une \'equation d'\'etat de la forme
\begin{eqnarray}
  P = \omega \rho   \label{gene14}
\end{eqnarray}
o\`u $ \omega $ est suppos\'e constant. La mati\`ere sans pression (mati\`ere noire froide ) est d\'ecrite
par $ \omega = 0 $ tandis que le rayonnement a une \'equation d'\'etat  $ \omega = 1/3 $. Dans le cas d'une
constante cosmologique, $ \omega = -1 $.

Une fois muni de l'\'equation d'\'etat ( \ref{gene14}), il est ais\'e d'int\'egrer les \'equations de 
Friedmann ( \ref{gene5}) et (\ref{gene5'}) et l'\'equation de conservation de la mati\`ere ( \ref{gene6}) ; pour $ \omega \neq  -1 $, on
obtient
\begin{eqnarray}
 H = \frac{2}{ 3(1 +  \omega )(t - \bar{t}) } \label{gene15}  \\ a(t) \propto  (t- \bar{t})^{ \frac{2}{ 3(1+ \omega)}} \label{gene19} \\
 \rho \propto a^{ -3( 1 + \omega)} \label{gene20}
\end{eqnarray}
o\`u $\bar{t}$ correspond au d\'ebut de l'\`ere consid\'er\'ee. Pour $ \omega  = -1$, l'Univers est en acc\'el\'eration
et correspond \`a un espace de de Sitter tel que
\begin{eqnarray}
 H = cste  \\ a(t) \propto e^{H t}  \\ \rho = cste.
\end{eqnarray}
En utilisant la d\'ependance de $ \rho $ en a, on peut exprimer $H$ en fonction du contenu en
mati\`ere de l'Univers, gr\^ace \`a l'expression
\begin{eqnarray}
 H^2 = H_0^2 \Biggl( \sum_{\substack{i}} \Omega_{i,0} a^{-3(1 + \omega_i)} \Biggl) = 
 H_0^2 \Biggl( \sum_{\substack{i}} \Omega_{i,0} (1+z)^{3(1 + \omega_i)} \Biggl), \label{gene21}
\end{eqnarray}
o\`u l'on a introduit le redshift $  z \equiv  a^{-1} - 1 $. En int\'egrant cette \'equation, on peut obtenir
l'\'evolution du facteur d'\'echelle $ a(t)$.

{\bf Des origines \`a nos jours}

Selon le mod\`ele cosmologique standard, l'histoire de l'Univers peut se d\'ecomposer en
une succession d'\'etapes (encore appel\'ees \`eres), que nous r\'esumons ci-dessous.

La nature de la premi\`ere phase de l'\'evolution de l'Univers, entre l'origine de l'Univers et
le Big Bang chaud, n'est pas encore bien \'etablie, car il est tr\`es difficile de tester les mod\`eles
cosmologiques \`a des \'epoques aussi recul\'ees. Le mod\`ele de l'inflation est cependant devenu
le paradigme standard. Il s'agit d'une phase d'expansion acc\'el\'er\'ee durant laquelle $  \omega \simeq -1 $.
Cette phase d'acc\'el\'eration permet de donner une explication naturelle au probl\`eme de la
platitude (le fait que l'espace apparaisse plat \`a l'heure actuelle, ce qui ne peut \^etre le cas que
si l'espace \'etait extr\^emement plat aux \'echelles de Planck) et au probl\`eme de l'horizon (le fait
que des zones du ciel causalement ind\'ependantes aient une temp\'erature comparable \`a  $ 10^{-5}$
pr\`es). On consid\`ere g\'en\'eralement que cette phase d'inflation est domin\'ee par un champ
scalaire (ou plusieurs champs scalaires pour les mod\`eles multi-champs) appel\'e inflaton, en
r\'egime de roulement lent (cf. par exemple \cite{215}). Les perturbations quantiques
de ce champ sont la source des perturbations de la mati\`ere que l'on peut observer dans les
grandes structures et le fond diffus cosmologique (CMB). \`A l'issue de l'inflation, l'inflaton
se d\'esint\`egre en un grand nombre de particules : c'est la phase de r\'echauffement.

La suite de l'histoire de l'Univers est bien comprise, et est d\'ecrite par le mod\`ele du Big
Bang chaud :

- la premi\`ere phase qui suit le Big Bang chaud est l'\`ere de radiation, domin\'ee par
le rayonnement (photons et neutrinos). Cette phase dure jusqu'\`a l'\'egalit\'e mati\`ere-rayonnement, 
c'est-\`a-dire jusqu'au moment o\`u la densit\'e de la mati\`ere commence \`a
dominer la densit\'e d'\'energie du rayonnement (le redshift correspondant est $z_{eq} + 1  \equiv a_{eq}^{-1} \simeq 3600 $

- L'\`ere de mati\`ere est domin\'ee par la mati\`ere noire (CDM). C'est le moment o\`u
se forment les grandes structures. C'est \'egalement durant l'\`ere de mati\`ere qu'a
lieu le d\'ecouplage entre les photons et les baryons (le redshift correspondant est
  $z_{dec} + 1  \equiv a_{dec}^{-1} \simeq 1100 $).
 Cela est d\^u \`a la recombinaison des protons et des 
 \'electrons libres qui s'assemblent pour former des atomes neutres. Apr\`es le d\'ecouplage,
les photons peuvent se propager librement dans l'Univers ; l'observation du CMB
correspond \`a l'observation de ces photons \'emis au moment du d\'ecouplage.

- Les observations r\'ecentes du CMB, des supernovae et des oscillations baryons-photons
(BAO) indiquent que l'Univers est entr\'e tardivement (aux alentours de \cite{75}   $ z_c \simeq 0,67 $
pour   $ \Omega_m = 0,3 $ ) dans une nouvelle phase d'expansion acc\'el\'er\'ee, telle que  $  \omega \simeq -1  $.

Cette derni\`ere observation, \`a savoir que l'Univers est actuellement dans une phase 
d'expansion acc\'el\'er\'ee, constitue la motivation premi\`ere des th\'eories de gravit\'e modifi\'ee que
nous \'etudierons dans ce m\'emoire de th\`ese.
% Nous chercherons notamment \`a r\'epondre aux
% question suivantes : quelle peut \^etre la source de cette acc\'el\'eration, et comment contraindre
% exp\'erimentalement les propri\'et\'es physiques de cette source ?
% Avant de nous tourner vers ces questions , 
Nous voudrions
consacrer la prochaine section \`a la mod\'elisation de la source d'\'energie responsable de
cette acc\'el\'eration et que l'on appelle commun\'ement \'energie noire, ainsi qu'aux preuves
exp\'erimentales de son existence.

\section{Mod\'elisation et preuves exp\'erimentales de l'existence de
l'\'energie noire}
\subsection{Mod\'elisation de l'\'energie noire}
La source de l'acc\'el\'eration de l'Univers observ\'ee aujourd'hui est mal connue ; c'est
pourquoi on lui donne souvent le nom d'\'energie noire, traduisant ainsi le caract\`ere effectif
de cette description. On peut mod\'eliser cette \'energie noire par le tenseur \'energie impulsion
d'un fluide parfait
\begin{eqnarray}
 T_\mu^{de\, \nu} = diag( -\rho_{de}, P_{de}, P_{de}, P_{de}), \label{gene22}
\end{eqnarray}

la forme ci-dessus \'etant impos\'ee par l'hypoth\`ese d'homog\'en\'eit\'e et d'isotropie de l'Univers
qui est \`a la base de notre \'etude. On peut d\'efinir l'\'equation d'\'etat
\begin{eqnarray}
 \omega_{de} \equiv P_{de}/\rho_{de}
\end{eqnarray}
qui doit \^etre telle que
\begin{eqnarray}
 \omega_{de}<-1/3
\end{eqnarray}
pour que la phase domin\'ee par l'\'energie noire soit une phase d'acc\'el\'eration. La densit\'e
r\'eduite d'\'energie noire est mesur\'ee par
\begin{eqnarray}
 \Omega_{de} \equiv \frac{8 \pi G \rho_{ de}}{3 H^2} \label{gene25}
\end{eqnarray}
Les fonctions $ \Omega_{de}(a) $  et $ \omega_{de}(a) $ sont reli\'ees l'une \`a l'autre au travers de l'\'equation (\ref{gene20}),
qui dans le cas de l'\'energie noire, prend la forme
\begin{eqnarray}
 \Omega_{de} = \Omega_{de,0}a^{-3(\omega_{de} +1)}
\end{eqnarray}
Nous verrons \`a la Section\ref{emilo}  diff\'erents mod\`eles visant \`a d\'ecrire la nature de cette
\'energie noire. Il pourra s'agir d'une source de mati\`ere (champ scalaire par exemple), 
responsable de l'acc\'el\'eration. Il est \'egalement possible que les lois de la gravit\'e ne soient pas
celles de la RG, auquel cas l'interpr\'etation de l'\'energie noire comme une \'energie n'est pas
\`a prendre au sens litt\'eral ; cependant, comme les cons\'equences de ces modifications de la
gravit\'e ne peuvent pas \^etre distingu\'ees des effets qu'auraient une source d'\'energie de la
forme ( \ref{gene22}), nous d\'ecrirons la source de l'acc\'el\'eration de l'Univers, telle qu'elle soit,
par le tenseur \'energie impulsion (\ref{gene22}), et appellerons le ph\'enom\`ene responsable {\bf  \'energie 
noire}.

D'un point de vue pratique, on peut d\'ecrire en premi\`ere approximation l'impact de
cette \'energie noire par une constante cosmologique $ \Lambda$ telle que   $ \rho_\Lambda = - P_\Lambda = \Lambda / ( 8 \pi G)$ . Les
observations permettent d'\'evaluer
\begin{eqnarray}
 \Omega_\Lambda = \frac{\Lambda}{ 3 H_0^2} \simeq 0,7.
\end{eqnarray}
Nous voudrions \`a pr\'esent d\'ecrire plus en d\'etail les m\'ethodes astrophysiques et 
cosmologiques qui prouvent l'existence de l'\'energie noire. La plupart des valeurs num\'eriques qui
seront donn\'ees correspondent au cas d'une constante cosmologique (mod\`ele dit $ \Lambda CDM $),
qui est la param\'etrisation la plus simple de l'\'energie noire. Il faut bien s\^ur garder \`a l'esprit
qu'une param\'etrisation plus fine peut se r\'ev\'eler plus r\'ealiste. Le r\'esultat principal reste
cependant le m\^eme, quelque soit le cadre dans lequel on l'interpr\`ete : une composante
d'\'energie noire est n\'ecessaire pour expliquer les observations que sont les Supernovae de
type Ia, l'\^age apparent de l'Univers, la position des pics acoustiques du $CMB$ et la signature
des oscillations baryons-photons dans les grandes structures.


% 
%  \section{Mod\'elisation et preuves exp\'erimentales de l'existence de
% l'\'energie noire}
% \subsection{Mod\'elisation de l'\'energie noire}
% La source de l'acc\'el\'eration de l'Univers observ\'ee aujourd'hui est mal connue ; c'est
% pourquoi on lui donne souvent le nom d'\'energie noire, traduisant ainsi le caract\`ere effectif
% de cette description. On peut mod\'eliser cette \'energie noire par le tenseur \'energie impulsion
% d'un fluide parfait
% \begin{eqnarray}
%  T_\mu^{de\, \nu} = diag( -\rho_{de}, P_{de}, P_{de}, P_{de}), \label{gene22}
% \end{eqnarray}
% 
% la forme ci-dessus \'etant impos\'ee par l'hypoth\`ese d'homog\'en\'eit\'e et d'isotropie de l'Univers
% qui est \`a la base de notre \'etude. On peut d\'efinir l'\'equation d'\'etat
% \begin{eqnarray}
%  \omega_{de} \equiv P_{de}/\rho_{de}
% \end{eqnarray}
% qui doit \^etre telle que
% \begin{eqnarray}
%  \omega_{de}<-1/3
% \end{eqnarray}
% pour que la phase domin\'ee par l'\'energie noire soit une phase d'acc\'el\'eration. La densit\'e
% r\'eduite d'\'energie noire est mesur\'ee par
% \begin{eqnarray}
%  \Omega_{de} \equiv \frac{8 \pi G \rho_{ de}}{3 H^2} \label{gene25}
% \end{eqnarray}
% Les fonctions $ \Omega_{de}(a) $  et $ \omega_{de}(a) $ sont reli\'ees l'une \`a l'autre au travers de l'\'equation (\ref{gene20}),
% qui dans le cas de l'\'energie noire, prend la forme
% \begin{eqnarray}
%  \Omega_{de} = \Omega_{de,0}a^{-3(\omega_{de} +1)}
% \end{eqnarray}
% Nous verrons \`a la Section\ref{emilo}  diff\'erents mod\`eles visant \`a d\'ecrire la nature de cette
% \'energie noire. Il pourra s'agir d'une source de mati\`ere (champ scalaire par exemple), 
% responsable de l'acc\'el\'eration. Il est \'egalement possible que les lois de la gravit\'e ne soient pas
% celles de la RG, auquel cas l'interpr\'etation de l'\'energie noire comme une \'energie n'est pas
% \`a prendre au sens litt\'eral ; cependant, comme les cons\'equences de ces modifications de la
% gravit\'e ne peuvent pas \^etre distingu\'ees des effets qu'auraient une source d'\'energie de la
% forme ( \ref{gene22}), nous d\'ecrirons la source de l'acc\'el\'eration de l'Univers, telle qu'elle soit,
% par le tenseur \'energie impulsion (\ref{gene22}), et appellerons le ph\'enom\`ene responsable {\bf  \'energie 
% noire}.
% 
% D'un point de vue pratique, on peut d\'ecrire en premi\`ere approximation l'impact de
% cette \'energie noire par une constante cosmologique $ \Lambda$ telle que   $ \rho_\Lambda = - P_\Lambda = \Lambda / ( 8 \pi G)$ . Les
% observations permettent d'\'evaluer
% \begin{eqnarray}
%  \Omega_\Lambda = \frac{\Lambda}{ 3 H_0^2} \simeq 0,7.
% \end{eqnarray}
% Nous voudrions \`a pr\'esent d\'ecrire plus en d\'etail les m\'ethodes astrophysiques et 
% cosmologiques qui prouvent l'existence de l'\'energie noire. La plupart des valeurs num\'eriques qui
% seront donn\'ees correspondent au cas d'une constante cosmologique (mod\`ele dit $ \Lambda CDM $),
% qui est la param\'etrisation la plus simple de l'\'energie noire. Il faut bien s\^ur garder \`a l'esprit
% qu'une param\'etrisation plus fine peut se r\'ev\'eler plus r\'ealiste. Le r\'esultat principal reste
% cependant le m\^eme, quelque soit le cadre dans lequel on l'interpr\`ete : une composante
% d'\'energie noire est n\'ecessaire pour expliquer les observations que sont les Supernovae de
% type Ia, l'\^age apparent de l'Univers, la position des pics acoustiques du $CMB$ et la signature
% des oscillations baryons-photons dans les grandes structures.

\subsection{Supernovae de type Ia}
La preuve la plus directe que notre Univers est actuellement en expansion acc\'el\'er\'ee
est donn\'ee par la mesure de la distance des luminosit\'es des supernovae de type $Ia$ (SN Ia). La
distance luminosit\'e d'un objet astrophysique de luminosit\'e intrins\`eque $L_{\text{source}}$ et situ\'e \`a
un redshift $z$ est d\'efinie de telle sorte que le flux observ\'e $ \Phi_{obs}$ est donn\'e par
\begin{eqnarray}
 \Phi_{obs} = \frac{L_{source}}{ 4 \pi D_L^2} \label{gene28}
\end{eqnarray}
On peut montrer \cite{267} que dans un Univers (plat) en expansion, $D_L$ est donn\'ee par
\begin{eqnarray}
 D_L = \frac{1}{H_0} (1 + z) \int_0^z \frac{dz'}{\sqrt{ \sum_{i}  \Omega_{i,0}(1+z')^{3(1 + \omega_i) }}} \label{gene29}
\end{eqnarray}
o\`u les $ \Omega_{i,0} $ sont les densit\'es r\'eduites de chaque composante de mati\`ere ($y$ compris l'\'energie
noire).

Pour des objets proches ($ z  \ll 1 $), on voit que la formule ( \ref{gene29}) se r\'esume \`a $ z \sim H_0 D_L $.
D'autre part, on peut montrer que le redshift est reli\'e \`a la vitesse de r\'ecession des galaxies
au travers de $ z \equiv  a^{-1} - 1 \sim v $ ; on obtient donc que pour des petits redshifts $ v \sim H_0 D_L $.
On retrouve l\`a la loi de Hubble qui permet de mesurer $H_0$ \`a partir de l'observation de la
vitesse de r\'ecession des galaxies. Pour des objets plus lointains, on voit que l'expression
de la distance luminosit\'e d\'epend du contenu en \'energie de l'Univers, au travers de la
racine carr\'ee de l'\'equation ( \ref{gene29}). La mesure de la fonction $ D_L(z) $ permet donc de mesurer
le r\^ole jou\'e par les diff\'erentes composantes de mati\`ere $ \Omega_{i,0} $ . Cependant, pour mesurer la
distance luminosit\'e d'objets au travers de la d\'efinition ( \ref{gene28}), encore faut-il conna\^itre leurs
luminosit\'es intrins\`eques.

C'est pr\'ecis\'ement le cas des  $ SN Ia$, qui peuvent \^etre observ\'ees lorsqu'une naine blanche
d'un syst\`eme binaire a accr\'et\'e tellement de mati\`ere en provenance de son compagnon que
sa masse a atteint la masse de Chandrasekhar ; elle s'effondre alors et explose en supernova.
La courbe de lumi\`ere des $ SN Ia $ est quasi-identique d'une supernova \`a l'autre : ce sont des
chandelles standard. En 1998, deux \'equipes ont publi\'e ind\'ependamment le diagramme de
Hubble $ D_L(z) $ de deux catalogues de $ SN Ia $ : le groupe Supernova Cosmology Project (SCP)
\cite{213} a ainsi mesur\'e la distance luminosit\'e de 42 $SN Ia$ de redshift $ z \in [0.18; 0.8]$, tandis
que les membres de la High-Z Supernova Search Team (HSST) \cite{224} ont identifi\'e 24 $SN Ia$
proches et 14 $SN Ia$ dans l'intervalle $ z \in [0.16; 0.62]$ . Ces observations (ainsi que d'autres
mesures qui ont eu lieu depuis lors, parmi lesquelles on peut citer \cite{26, 171, 195, 225} ont
permis d'\'ecarter la possibilit\'e que l'Univers soit plat et constitu\'e uniquement de mati\`ere
 $( \Omega_m = 1)$. Il est donc n\'ecessaire d'inclure une composante d'\'energie noire, comme on peut
le voir sur la Fig.\ref{figgene1} .

% \begin{figure}[h]
% \includegraphics[width=30pc]{figgene1.png}
% \caption{\label{figgene1} Diagramme de Hubble (distance luminosit\'e $ \log [H_0 D_L(z) ] $ en fonction du 
% red-shift $z$) pour un Univers plat pour un certain nombres de SN Ia observ\'ees. Les points
% noirs proviennent du catalogue “Gold” de Riess et al. $[225]$, tandis que les points rouges
% proviennent de mesures effectu\'ees par le Hubble Space Telescope $(HST)$. On voit tr\`es 
% clairement que le cas $ \Omega_m = 0.31$, $ \Omega_\Lambda = 0.69$ correspond mieux aux observations que le cas d'un
% univers sans constante cosmologique $ \Omega_m = 1$, $ \Omega_\Lambda = 0$ . La figure provient de la r\'ef\'erence
% $[71]$. 
%  }
% \end{figure}

\begin{figure}[h]
%\begin{minipage}{14pc}
\includegraphics[width=30pc]{figgene1.png}
\caption{\label{figgene1} Diagramme de Hubble (distance luminosit\'e $ \log [H_0 D_L(z) ] $ en fonction du 
red-shift $z$) pour un Univers plat pour un certain nombres de SN Ia observ\'ees. Les points
noirs proviennent du catalogue ''Gold'' de Riess et al. \cite{225}, tandis que les points rouges
proviennent de mesures effectu\'ees par le Hubble Space Telescope $(HST)$. On voit tr\`es 
clairement que le cas $ \Omega_m = 0.31$, $ \Omega_\Lambda = 0.69$ correspond mieux aux observations que le cas d'un
univers sans constante cosmologique $ \Omega_m = 1$, $ \Omega_\Lambda = 0$ . La figure provient de la r\'ef\'erence
\cite{71}.}
%\end{minipage}\hspace{3pc}%
% \begin{minipage}{14pc}
% \includegraphics[width=15pc]{t4.png}
%  \caption{Diagramme de Hubble (pour
% les supernovae) de $2005$ \cite{2}}
%  \end{minipage}\hspace{3pc}%
\end{figure}



% 
% Utilis\'ees toutes seules, les donn\'ees des SN Ia ne peuvent pas contraindre tr\`es 
% pr\'ecis\'ement $ \Omega_\Lambda $ ou $ \Omega_m $, mais seulement une combinaison des deux param\`etres; cela est d\^u \`a la
% forme elliptique des contraintes (cf. Fig. \ref{figgene2}). Nous verrons qu'il est possible d'am\'eliorer
% consid\'erablement la pr\'ecision des mesures en combinant ces mesures de distance luminosit\'e
% avec d'autres donn\'ees (CMB, BAO,...).
% 
%  \begin{figure}[h]
% \includegraphics[width=20pc]{figgene2.png}
% \caption{\label{figgene2} Contours des niveaux de confiance \`a $ 68.3 \% $, $ 95.5 \% $ et $ 99.7 \% $ pour les param\`etres
% $ (\Omega_m, \Omega_\Lambda )$, issues du diagramme de Hubble obtenu par le Supernova Legacy Survey (SNLS)
% (lignes pleines) et des BAO mesur\'ees par le Sloan Digital Sky Survey (SDSS) (lignes 
% pointill\'ees). Les contraintes conjointes sont repr\'esent\'ees par les lignes en pointill\'es larges. Le
% cas $ (\Omega_m, \Omega_\Lambda ) = (1, 0)$ est tr\`es clairement exclu. Les valeurs des param\`etres $ (\Omega_m, \Omega_\Lambda )$ 
% favoris\'ees par les donn\'ees des SN Ia sont dans une ellipse orient\'ee selon une droite d'\'equation
%   $  \Omega_\Lambda \backsimeq  \Omega_m + 0.4 $ . La figure ci-dessus est issue de \cite{26}.
%   }
% \end{figure}
 
 \subsection{ \^Age de l'Univers }
 Au moment du Big Bang chaud (qui correspond \`a la fin de l'inflation dans les mod\`eles
inflationnaires) que l'on prendra par convention \`a $ t = 0$, on doit avoir  $ a(t = 0) \geqslant 0 $. On en
d\'eduit donc que le temps $t_0$ \'ecoul\'e depuis le Big Bang est born\'e par
\begin{eqnarray}
 t_0 = \int_{\tau_0}^{\tau_1} dt = \int_{a(t=0)}^{a_0 = 1} \frac{dt}{ da} da \leqslant \int_{0}^{ 1} \frac{dt}{ da} da =
 \int_{0}^{ 1} \frac{da}{ aH}   \label{gene30}
\end{eqnarray}
Si on utilise l'\'equation ( \ref{gene21}) et l'on exprime l'int\'egrale en fonction du redshift $z$ on obtient
\begin{eqnarray}
 t_0 \leqslant \int_{0}^{\infty } \frac{dz}{(1 + z)H(z) } = \frac{1}{H_0}
 \int_0^{\infty} \frac{dz}{ (1 + z) \sqrt{ \sum_{i}  \Omega_{i,0}(1+z)^{3(1 + \omega_i) }}}, \label{gene31}
\end{eqnarray}
o\`u l'indice $i$ correspond \`a la radiation, la mati\`ere noire et \`a l'\'energie noire, que l'on 
mod\'elisera par une constante cosmologique. On peut n\'egliger dans l'int\'egrale ci-dessous la
contribution de la radiation, car l'\`ere de radiation est tr\`es courte compar\'ee aux \`eres 
suivantes. On voit donc que l'\^age de l'Univers d\'epend des param\`etres $(\Omega_m, \Omega_{de} )$. Dans le cas
d'un univers domin\'e par la mati\`ere tel que $\Omega_m = 1$, on peut calculer explicitement 
l'int\'egrale de l'\'equation (\ref{gene31}) et obtenir la borne $ t_0 \leqslant 2 /(3H_0)$. En utilisant la valeur de la
constante de Hubble trouv\'ee par l'\'equipe du Hubble Key Project \cite{126} $ H_0 = 72 \pm 8 km.s^{-1} Mpc^{-1} $,
 on trouve que $t_0 = 8 − 10 $ milliards d'ann\'ees. Cette valeur de l'\^age de l'Univers
doit \^etre compar\'ee \`a l'\^age des plus vieux objets stellaires connus, qui est d'environ $ 11 − 13$
milliards d'ann\'ees \cite{143, 163, 223}. {\bf  \`A l'\'evidence, le cas d'un Univers domin\'e uniquement par
 la mati\`ere est donc en contradiction avec l'\^age de ces objets.} Cette difficult\'e peut \^etre
r\'esolue en supposant la pr\'esence d'\'energie noire. Ainsi, si l'on fait l'hypoth\`ese que 
l'\'energie noire prend la forme d'une constante cosmologique telle que $(\Omega_m, \Omega_{\Lambda} ) = (0.3, 0.7)$, on
trouve que  $ t_0 \backsimeq 13.1$ milliards d'ann\'ees, {\bf ce qui n'est plus contradictoire avec l'\^age des plus
vieux objets connus dans l'Univers. Ceci est donc un \'el\'ement suppl\'ementaire en faveur de
la pr\'esence d'\'energie noire dans l'Univers.}
 
 \subsection{Fond diffus cosmologique}
 
 Une preuve ind\'ependante de l'existence d'\'energie noire dans l'Univers peut \^etre obtenue
\`a partir de la position des pics acoustiques du fond diffus cosmologique (CMB). En effet,
on peut montrer \cite{215} que la position dans l'espace des multip\^oles du n-i\`eme pic est donn\'ee
dans le cas de perturbations primordiales adiabatiques par
\begin{eqnarray}
  l_{(n)} = n \pi \frac{D_A(z_{LSS})}{r_s (z_{LSS}) }, \;\;\;\; n = 1, \; 2, \; ... \label{gene32}
\end{eqnarray}
o\`u nous avons introduit l'horizon sonique co-mobile
\begin{eqnarray}
 r_s(z) \equiv \int_z^{\infty } \frac{C_s(z')}{H(z') }dz', \label{gene33}
\end{eqnarray}
et le diam\`etre angulaire co-mobile
\begin{eqnarray}
 D_A = \frac{1}{H_0} \frac{1}{1+z} \int_0^z \frac{dz'}{\sqrt{ \sum_{i}  \Omega_{i,0}(1+z')^{3(1 + \omega_i) }}} \label{gene34}
\end{eqnarray}
Dans les expressions ci-dessus, $z_{LSS} \simeq  1100 $ est le redshift de la surface de derni\`ere 
diffusion  et $c_s$ est la vitesse du son du fluide baryon-photon dont les oscillations avant la
recombinaison sont \`a l'origine des pics du CMB. Le rayon sonique $ r_s(z{LSS})$ d\'epend de
la quantit\'e de baryons dans l'Univers $ \Omega_b $ et est ind\'ependant de l'\'energie noire car cette
derni\`ere ne joue aucun r\^ole \`a des temps aussi anciens ; la hauteur des pics du CMB et en
particulier la diff\'erence entre la hauteur des pics pairs et impairs permet de mesurer $ \Omega_b $
(et donc $ r_s (z_{LSS} )$ ). On peut alors utiliser la mesure de la position des pics (\ref{gene32}) pour
contraindre l'\'energie noire, au travers de la d\'ependance du diam\`etre angulaire $ D_A(z_{LSS} )$
dans la g\'eom\'etrie de l'Univers.

Le spectre du CMB d\'epend \'egalement de l'\'energie noire au travers de l'effet 
SachsWolfe int\'egr\'e (ISW), qui permet de prendre en compte le fait que les photons du CMB ont
travers\'e sur leur chemin jusqu'\`a nous des puits de potentiel en \'evolution. L'\'evolution des
potentiels gravitationnels d\'ependant du contenu en mati\`ere de l'Univers, la mesure de cet
effet permet de contraindre la densit\'e d'\'energie noire, ainsi que son \'equation d'\'etat $ \omega_{de} $.

L'analyse des donn\'ees prises pendant $5$ ans \cite{177} par le satellite $WMAP$ a permis de
mesurer $ \Omega_\Lambda = 0.742 \pm 0.030 $ dans le cas d'une constante cosmologique. Dans le cas o\`u
l'\'equation d'\'etat de l'\'energie noire est \'egalement mesur\'ee, l'\'equipe de WMAP a obtenu
l'intervalle de confiance \`a $95 \%$ :$ −1.37 < 1 + \omega_{de} < 0.32 $.
 
 \subsection{Oscillations Acoustiques des Baryons (BAO)}
 Les oscillations du fluide baryons-photons qui ont lieu avant la recombinaison ne sont
pas seulement observables dans le spectre du CMB, mais \'egalement dans les grandes 
structures. En effet, consid\'erons une sur-densit\'e \`a $ t = 0$; sous l'effet de la pression de la radiation,
cette sur-densit\'e s'\'etend \`a la vitesse du son du plasma, sous la forme d'une onde sph\'erique.

Au moment de la recombinaison, les densit\'es aux points situ\'es \`a une distance $ r_s (z_{LSS} )$
de la sur-densit\'e initiale sont donc corr\'el\'ees. De nos jours, cette corr\'elation est bien s\^ur
brouill\'ee, car les ondes sonores de diff\'erentes sur-densit\'es se sont superpos\'ees depuis ; il est
cependant possible de d\'etecter statistiquement la signature de l'\'echelle co-mobile $ r_s (z_{LSS} )$,
caract\'eristique des BAO, au travers de la mesure de la fonction de corr\'elation des galaxies.
En raison de l'expansion de l'Univers, cette \'echelle co-mobile (fixe depuis la recombinaison,
et qui vaut environ $ r_s (z_{LSS} ) \backsimeq 150 Mpc) $ est vue sous un angle
\begin{eqnarray}
 \theta_S = \frac{ r_s (z_{LSS} ) }{ D_A(z_{LSS} ) } \label{gene35}
\end{eqnarray}
o\`u le diam\`etre angulaire co-mobile $ D_A(z_{LSS} )$ a \'et\'e d\'efini \`a l'\'equation (\ref{gene34}). La premi\`ere
d\'etection de ce ph\'enom\`ene a eu lieu en 2005 par l'\'equipe du SDSS  \cite{115}; on voit sur
la figure \ref{figgene3} la signature tr\`es nette du rayon sonique dans la fonction de corr\'elation des
galaxies 8 du catalogue.

% \begin{figure}[h]
% \includegraphics[width=35pc]{figgene3.png}
% \caption{\label{figgene3} Fonction de corr\'elation des galaxies du catalogue du SDSS. Les courbes verte
% (en haut), rouge et bleue (en bas, avec un pic) correspondent respectivement aux mod\`eles
%  $ \Omega_m h^2 =  0.12, 0.13, 0.14$, avec $ \Omega_b h^2 = 0.024 $. La courbe du bas (sans pic) correspond \`a un
% mod\`ele de mati\`ere noire sans baryon $ ( \Omega_m h^2 = 0.105) $. On voit tr\`es clairement la signature
% des BAO \`a $ r \backsimeq 105h^{-1}Mpc \backsimeq 150Mpc $. Figure issue de l'article \cite{115}.
%  }
% 
% \end{figure}

\begin{figure}[h]
\includegraphics[width=25pc]{figgene3.png}
\caption{\label{figgene3} Fonction de corr\'elation des galaxies du catalogue du SDSS. Les courbes verte
(en haut), rouge et bleue (en bas, avec un pic) correspondent respectivement aux mod\`eles
 $ \Omega_m h^2 =  0.12, 0.13, 0.14$, avec $ \Omega_b h^2 = 0.024 $. La courbe du bas (sans pic) correspond \`a un
mod\`ele de mati\`ere noire sans baryon $ ( \Omega_m h^2 = 0.105) $. On voit tr\`es clairement la signature
des BAO \`a $ r \backsimeq 105h^{-1}Mpc \backsimeq 150Mpc $. Figure issue de l'article  \cite{115}.
 }
\end{figure}

Il est donc possible de tester le contenu en mati\`ere de l'Univers par l'\'etude des BAO,
et notamment la densit\'e d'\'energie noire $ \Omega_{de} $ et son \'equation d'\'etat $ \omega_{de} $ . En particulier,
l'observation des BAO donne des informations tr\`es pr\'ecieuses lorsqu'elles sont combin\'ees
avec d'autres observations, comme on peut le voir sur la figure \ref{figgene4}.
 
 \subsection{Mesures combin\'ees}
 Comme nous l'avons d\'ej\`a signal\'e ci-dessus, les m\'ethodes \'evoqu\'ees (SN Ia, CMB, BAO)
sont compl\'ementaires. En premier lieu, ces mesures provenant de processus physiques tr\`es 
diff\'erents et ind\'ependants, il est impressionnant qu'elles puissent mener \`a des mesures 
coh\'erentes : cela confirme d'une fa\c{c}on spectaculaire la robustesse du mod\`ele cosmologique
standard. De plus, les ellipses de contraintes associ\'ees \`a chacune de ces m\'ethodes ne sont
pas orient\'ees dans la m\^eme direction, comme on peut le voir sur la figure \ref{figgene4}. En combinant
ces mesures, on peut donc diminuer la d\'eg\'en\'erescence entre les diff\'erents param\`etres. En
proc\'edant de la sorte, Percival et al. \cite{212} ont mesur\'e, dans le cas d'un mod\`ele $ \Lambda CDM $,
$ (\Omega_m , \Omega_\Lambda ) = (0.288 \pm 0.018, 0.712  \pm 0.018)$.

L'existence d'une phase d'acc\'el\'eration r\'ecente de l'Univers est donc bien \'etablie. Nous
savons que l'on peut mod\'eliser la source de cette acc\'el\'eration par le tenseur \'energie-
impulsion d'un fluide parfait, que nous avons nomm\'e \'energie noire. Nous n'avons cependant
pas encore abord\'e la question de la nature physique de cette \'energie noire. C'est ce vers
quoi il nous faut \`a pr\'esent nous tourner.



	 \begin{figure}[h]
%\begin{minipage}{14pc}
\includegraphics[width=25pc]{figgene4.png}
\caption{\label{figgene4} Combinaison des contraintes sur les param\`etres $( \Omega_m, \Omega_\Lambda )$ du mod\`ele $ \Lambda CDM$
issues de la mesure de la distance luminosit\'e des SN Ia (Union supernovae \cite{195}), du
CMB (WMAP5) et des BAO (SDSS - release 7). Les ellipses des contraintes associ\'ees des
trois types de mesures sont orient\'ees d'une fa\c{c}on tr\`es diff\'erente : l'association des trois
observations permet donc de mesurer les param\`etres $( \Omega_m, \Omega_\Lambda )$  beaucoup plus pr\'ecis\'ement
qu'avec chacune des m\'ethodes s\'epar\'ement. La figure provient de \cite{212}.
  }
%\end{minipage}\hspace{3pc}%
% \begin{minipage}{14pc}
% \includegraphics[width=15pc]{t4.png}
%  \caption{Diagramme de Hubble (pour
% les supernovae) de $2005$ \cite{2}}
%  \end{minipage}\hspace{3pc}%
\end{figure}


\newpage
 
 \section{\'Energie noire ou gravit\'e modifi\'ee ?} \label{emilo}
\subsection{Le probl\`eme de la constante cosmologique}
Nous avons d\'ej\`a vu que le mod\`ele le plus simple d'\'energie noire est simplement une
constante cosmologique $ \Lambda $ telle que $ \rho_\Lambda = -P_\Lambda =  \Lambda /(8 \pi G)$ et $ \Omega_\Lambda \backsimeq 0,7 $. Le probl\`eme
de ce mod\`ele est qu'il n'explicite pas vraiment l'origine physique de cette constante. La
source d'\'energie qui pourrait correspondre d'une fa\c{c}on a priori naturelle \`a une constante
cosmologique est l'\'energie du vide. Malheureusement, les ordres de grandeur des deux
ph\'enom\`enes n'ont rien en commun. La constante cosmologique est telle que la densit\'e correspondant vaut
\begin{eqnarray}
 \rho_\Lambda = 10^{-47}GeV   \label{gene36}
\end{eqnarray}
On peut d'autre part \'evaluer la densit\'e d'\'energie du vide, qui correspond \`a la somme sur
tous les modes, de l'\'energie de point z\'ero  $ \omega_k/2 $. Pour un champ scalaire, on obtient :
\begin{eqnarray}
 \rho_{vac} = \frac{1}{2} \int_0^{k_{max} } \frac{d^3k}{(2 \pi)^3} \sqrt{k^2 + m^2} \backsimeq  \frac{k^4_{max}}{ 16 \pi^2} \label{gene37}
\end{eqnarray}
o\`u $ k^4_{max} $ est le  cutoff
de la th\'eorie des champs que l'on consid\`ere et $m$ la masse du champ,
que l'on supposera  $ \ll k_{max} $ . Si l'on prend pour cutoff l'\'echelle de grande unification $ k_{max} =
E_{GUT} = 10^{16} GeV $, on obtient
\begin{eqnarray}
 \rho_{vac} \backsimeq 10^{62 } GeV^4 \ggg  \rho_\Lambda.  \label{gene38}
\end{eqnarray}

Le mod\`ele de la constante cosmologique est \'egalement insatisfaisant du point de vue de
l'explication de la valeur de sa densit\'e d'\'energie actuelle. Ce probl\`eme est connu sous le
nom de probl\`eme de la co\"{i}ncidence : pourquoi la densit\'e d'\'energie noire est-elle de l'ordre
de la densit\'e de la mati\`ere noire, pr\'ecis\'ement aujourd'hui, alors que les deux ph\'enom\`enes
ont a priori des origines tr\`es diff\'erentes. De plus, cette densit\'e, tr\`es faible, doit \^etre 
ajust\'ee tr\`es pr\'ecis\'ement : c'est ce qui est appel\'e le probl\`eme du fine tuning de la constante
cosmologique.

Ces difficult\'es d'ordre th\'eorique (d'un point de vue ph\'enom\'enologique, le mod\`ele $ \Lambda CDM $
est pour l'instant en tr\`es bon accord avec les observations) ont pouss\'e la communaut\'e des
physiciens des particules et des cosmologistes \`a se pencher plus en d\'etail sur les possibles
mod\`eles d'\'energie noire, au-del\`a du mod\`ele de la constante cosmologique. Nous allons 
pr\'esenter un panorama de ces mod\`eles (on pourra \'egalement consulter avec profit les articles
\cite{75, 211, 265}). Deux grandes cat\'egories de mod\`eles peuvent \^etre distingu\'ees : les mod\`eles
qui supposent que la RG d\'ecrit correctement les lois de la gravitation, et les mod\`eles de
gravit\'e modifi\'ee.

\subsection{Mod\`eles restant dans le cadre de la Relativit\'e G\'en\'erale}
La plupart des mod\`eles de cette cat\'egorie consistent \`a prendre litt\'eralement 
l'interpr\'etation du fluide parfait (\ref{gene22}) en tant qu'\'energie, et \`a postuler l'existence d'un ou de
plusieurs nouveaux champs responsables de l'acc\'el\'eration r\'ecente de l'Univers.

Les mod\`eles les plus simples de cette classe sont les mod\`eles de quintessence \cite{221, 284},
dans lesquelles l'\'energie noire est d\'ecrite par un champ scalaire en r\'egime de roulement
lent, d'une mani\`ere tout \`a fait semblable \`a l'inflaton. L'action de ce champ scalaire $ \varphi $ s'\'ecrit
\begin{eqnarray}
 S = \int d^4x\sqrt{-g} \Biggl( -\frac{1}{2}g^{ \mu \nu} \partial_{\mu \varphi} \partial_{ \nu \varphi} - V(\varphi) \Biggr), \label{gene39}
\end{eqnarray}
o\`u $V$ est le potentiel du champ. \`a partir de cette action, il est facile de calculer le tenseur
\'energie-impulsion du champ scalaire dans un univers homog\`ene et isotrope ; on peut alors
montrer que l'\'equation d'\'etat prend la forme
\begin{eqnarray}
 \omega_\varphi = \frac{\frac{1}{2}\dot{\varphi} - V(\varphi) }{\frac{1}{2}\dot{\varphi} + V(\varphi) }. \label{gene40}
\end{eqnarray}
On peut d\'efinir les param\`etres de roulements lents par
\begin{eqnarray}
 \epsilon \equiv  \frac{1}{16 \pi G} \Biggl( \frac{V_{, \varphi}}{V}  \Biggr)^2, \;\;\;\;
 \eta = \frac{1}{ 8 \pi G} \Biggl(  \frac{V_{,\varphi \varphi}}{ V}    \Biggr), \label{gene41}
\end{eqnarray}
o\`u $ V{,\varphi}$ est la d\'eriv\'ee du potentiel par rapport au champ scalaire. Dans la limite de roulement
lent
\begin{eqnarray}
 \epsilon \ll 1, \;\;\;\; \eta \ll 1, \label{bene42}
\end{eqnarray}
le champ scalaire peut jouer le r\^ole d'\'energie noire, avec une \'equation d'\'etat  $ \omega_\varphi \backsimeq -1 + 2 \epsilon / 3 $ .

Le potentiel $ V( \varphi)M^4 e^{- \lambda \varphi} $ est un exemple int\'eressant de potentiel, car on peut 
montrer que le r\'egime $ \omega \backsimeq -1 $ est un attracteur, c'est-\`a-dire que pour une grande plage de
conditions initiales, le champ $ \varphi $ entrera dans le r\'egime de roulement lent si l'on attend 
suffisamment longtemps. Ce m\'ecanisme d'attraction ne r\'esoud pas le probl\`eme du fine-tuning,
mais permet n\'eanmoins de donner une explication naturelle au fait que l'\'energie noire ait
une \'equation d'\'etat constante ou quasi-constante aujourd'hui.

De nombreux autres mod\`eles existent, parmi lesquels on peut citer les mod\`eles de 
tachyon \cite{241, 242, 243}, d'\'energie noire fantomale \cite{57, 59} ou encore de K-essence \cite{23, 24, 69}.

Une autre possibilit\'e pour expliquer l'acc\'el\'eration apparente de l'Univers sans modifier
les lois de la gravitation consiste \`a trouver un m\'ecanisme qui modifie la fa\c{c}on dont nous
percevons les photons qui se propagent depuis les SN Ia et autres ph\'enom\`enes observables.
C'est par exemple le cas lorsque les photons sont coupl\'es \`a un axion, et peuvent osciller
lors de la propagation dans le champ magn\'etique extra-galactique \cite{78, 99, 102}. Les objets
astrophysiques nous apparaissent alors plus lointains qu'ils ne le sont en r\'ealit\'e.

\subsection{Mod\`eles de gravit\'e modifi\'ee}
Une autre approche de la question de l'\'energie noire consiste \`a questionner la validit\'e
m\^eme de la RG. Il existe ainsi de nombreux mod\`eles de gravit\'e modifi\'ee, dans lesquels les
lois de la gravit\'e sont diff\'erentes de la RG.

Les mod\`eles les plus simples de modifications sont sans doute les modifications scalaires
de la gravit\'e dans lesquelles un champ scalaire additionnel est coupl\'e non-minimalement \`a
la m\'etrique. Un exemple de tels
mod\`eles est donn\'e par la classe des th\'eories scalaire-tenseur \cite{41, 206, 276} dont l'action
prend la forme
\begin{eqnarray}
 S = \frac{1}{16 \pi G} \int d^4x \sqrt{-g} \Biggl(  F(\varphi )R -( \varphi)g^{\mu \eta} \partial_{ \mu \varphi
 \partial_{ \nu \varphi}} - 2U(\varphi)   \Biggr) + S_m [ \psi_m; g_{ \mu \nu}]. \label{gene43}
\end{eqnarray}


Cette action appara\^it comme une g\'en\'eralisation directe des mod\`eles de quintessence d\'ecrits
\`a l'\'equation ( \ref{gene39}), la diff\'erence \'etant que le champ scalaire des th\'eories scalaire-tenseur est
coupl\'e non-minimalement \`a la m\'etrique au travers de la fonction $F(\varphi )$. Il est possible de
coupler le champ scalaire \`a d'autres invariants de courbure que le scalaire de Ricci; on peut
par exemple penser aux mod\`eles dans lesquels le champ scalaire n'est pas coupl\'e \`a $R$ mais
au terme de Gauss-Bonnet, $ L_{GB} = R^2 - 4R_{ \mu \nu}R^{\mu \nu } + R_{ \mu \nu \rho \sigma} R^{\mu \nu \rho \sigma }$. Nous n'\'etudierons pas
plus avant ce types de mod\`eles, mais l'on pourra se r\'ef\'erer aux articles \cite{9, 10, 175, 176, 183}
pour plus de d\'etails. Notons enfin que deux mod\`eles de modifications scalaires dont l'action
poss\`ede plus de deux de d\'eriv\'ees du champ scalaire ont r\'ecemment \'et\'e propos\'es. Il s'agit
du mod\`ele de Galil\'eon introduit par A. Nicolis, R. Rattazzi et E. Trincherini \cite{203} (voir
\'egalement \cite{94, 97} et \cite{ 249}) et des mod\`eles de k-Mouflage pr\'esent\'es par E. Babichev, C.
Deffayet  dans l'article \cite{27}.

%%%%%%%%%%%%%%%%%%emile%%%%%%
% Il est \'egalement possible de modifier les lois de la gravit\'e en postulant l'existence de
% dimensions spatiales suppl\'ementaires. Le mod\`ele DGP \cite{111} propos\'e par Dvali, Gabadadze
% et Porrati en 2000 est sans doute le plus c\'el\`ebre des mod\`eles comportant des dimensions
% suppl\'ementaires et modifiant la gravit\'e \`a grandes distances. Notons aussi
% que la gravit\'e est 4-dimensionnelle \`a courtes distances et 5-dimensionnelle \`a grandes
% distances. La ph\'enom\'enologie de ce mod\`ele est tr\`es riche, notamment du point de vue
% cosmologique \cite{92}.
% 
% Vue de notre espace \`a 4 dimensions, cette th\'eorie \`a 5 dimensions appara\^it comme une
% th\'eorie o\`u la gravit\'e n'est plus port\'ee par un unique graviton sans masse, mais par une
% infinit\'e de gravitons massifs. Cette id\'ee a par la suite \'et\'e \'etendue \`a un ensemble plus large
% de mod\`eles, dans lesquelles la gravit\'e trouve sa source dans une r\'esonance de gravitons
% massifs ; ces mod\`eles sont appel\'es mod\`eles de d\'egravitation \cite{91, 112}. Le mod\`ele DGP et les
% th\'eories de d\'egravitation sont donc reli\'es \`a la classe des th\'eories de gravit\'e massive, dans
% lesquelles le graviton poss\`ede une masse. 
%%%%%%%%%%%%%%%emile%%%%%%%%%%%%%%

Pour \^etre int\'eressants d'un point de vue ph\'enom\'enologique, les mod\`eles de gravit\'e
modifi\'ee doivent \^etre nettement diff\'erents de la Relativit\'e G\'en\'erale, tout au moins aux
\'echelles cosmologiques. Cependant, nous savons que la gravit\'e dans le syst\`eme solaire est
tr\`es proche de la RG. Il faut donc un m\'ecanisme qui s\'epare ces deux r\'egimes. Pour l'instant,
deux m\'ecanismes de ce type sont connus : celui de cam\'el\'eon et le m\'ecanisme de Vainshtein.
Dans les deux cas, la gravit\'e est modifi\'ee par un champ scalaire qui est libre de se propager
\`a grandes distances mais qui ne joue pas de r\^ole au sein ou au voisinage d'objets massifs.

%Nous consacrerons une partie importante de ce m\'emoire au m\'ecanisme de Vainshtein,
%dont nous montrerons, pour la premi\`ere fois, la validit\'e dans le cadre des th\'eories de gravit\'e
%massive (Partie II). Nous l'utiliserons \'egalement pour construire la classe des th\'eories de
%k-Mouflage (Partie III).
Enfin, il est important d'avoir \`a l'esprit que tout mod\`ele de gravit\'e modifi\'ee r\'ealiste
doit \^etre capable de satisfaire de fortes contraintes exp\'erimentales, tant au niveau local
qu'\`a des \'echelles cosmologiques. 

\chapter{Conclusion g\'en\'erale et perspectives} \label{chapitre5}
\section{Conclusion}\label{conclusion}
\section{perspectives}\label{perspective}
% \label{chapit






\begin{thebibliography}{90}


%%%%%%%%%%%%%%%%%%%%%%%%%%%%%%%%%%%%DEBUT CHAPITRE2%%%%%%%%%%%%%%%%%%%%%%%%%%%%%%%%%%%%%%%%%%%%%%%%%
\bibitem{215} Peter, P., and Uzan, J.-P., Cosmologie primordiale. \'editions Belin, 2005.
\bibitem{277} Wald, R. M., General Relativity. Chicago University Press, 1984. Chicago, Usa :
Univ. Pr. ( 1984) 491p.
\bibitem{282} Weinberg, S., Gravitation and cosmology. John Wiley and Sons, 1972.
\bibitem{171} Knop, R. A., Aldering, G., Amanullah, R., Astier, P., Blanc, G., Burns, M. S., Conley,
A., Deustua, S. E., Doi, M., Ellis, R., Fabbro, S., Folatelli, G., Fruchter, A. S.,
Garavini, G., Garmond, S., Garton, K., Gibbons, R., Goldhaber, G., Goobar, A.,
Groom, D. E., Hardin, D., Hook, I., Howell, D. A., Kim, A. G., Lee, B. C., Lidman,
C., Mendez, J., Nobili, S., Nugent, P. E., Pain, R., Panagia, N., Pennypacker, C. R.,
Perlmutter, S., Quimby, R., Raux, J., Regnault, N., Ruiz-Lapuente, P., Sainton, G.,
Schaefer, B., Schahmaneche, K., Smith, E., Spadafora, A. L., Stanishev, V., Sullivan,
M., Walton, N. A., Wang, L., Wood-Vasey, W. M., and Yasuda, N., New constraints
on omegam, omegal, and w from an independent set of eleven high-redshift supernovae
observed with hst, Astrophys.J. 598 (2003) 102, [astro-ph/0309368].
%\bibitem{215} Peter, P., and Uzan, J.-P., Cosmologie primordiale. \'editions Belin, 2005.
\bibitem{75} Copeland, E. J., Sami, M., and Tsujikawa, S., Dynamics of dark energy,
Int.J.Mod.Phys.D 15 (2006) 1753–1936, [hep-th/0603057].
\bibitem{267} Uzan, J.-P., Tests of general relativity on astrophysical scales, arXiv:0908.2243.
\bibitem{213} Supernova Cosmology Project Collaboration, Perlmutter, S., et al., Measure-
ments of Omega and Lambda from 42 High-Redshift Supernovae, Astrophys. J. 517
(1999) 565–586, [astro-ph/9812133].
\bibitem{224} Riess, A. G., Filippenko, A. V., Challis, P., Clocchiattia, A., Diercks, A., Garnavich,
P. M., Gilliland, R. L., Hogan, C. J., Jha, S., Kirshner, R. P., Leibundgut, B., Phillips,
M. M., Reiss, D., Schmidt, B. P., Schommer, R. A., Smith, R. C., Spyromilio, J.,
Stubbs, C., Suntzeff, N. B., and Tonry, J., Observational evidence from supernovae
for an accelerating universe and a cosmological constant, Astron.J. 116 (1998) 1009–
1038, [astro-ph/9805201].
\bibitem{26} Astier, P., Guy, J., Regnault, N., Pain, R., Aubourg, E., Balam, D., Basa, S., Carl-
berg, R., Fabbro, S., Fouchez, D., Hook, I., Howell, D., Lafoux, H., Neill, J., Palanque-
Delabrouille, N., Perrett, K., Pritchet, C., Rich, J., Sullivan, M., Taillet, R., Aldering,
G., Antilogus, P., Arsenijevic, V., Balland, C., Baumont, S., Bronder, J., Courtois,
H., Ellis, R., Filiol, M., Goncalves, A., Goobar, A., Guide, D., Hardin, D., Lusset,
V., Lidman, C., McMahon, R., Mouchet, M., Mourao, A., Perlmutter, S., Ripoche,
P., Tao, C., and Walton, N., The supernova legacy survey : Measurement of omegam,
omegalambda and w from the first year data set, Astron.Astrophys. 447 (2006) 31–48,
[astro-ph/0510447].

\bibitem{171} Knop, R. A., Aldering, G., Amanullah, R., Astier, P., Blanc, G., Burns, M. S., Conley,
A., Deustua, S. E., Doi, M., Ellis, R., Fabbro, S., Folatelli, G., Fruchter, A. S.,
Garavini, G., Garmond, S., Garton, K., Gibbons, R., Goldhaber, G., Goobar, A.,
Groom, D. E., Hardin, D., Hook, I., Howell, D. A., Kim, A. G., Lee, B. C., Lidman,
C., Mendez, J., Nobili, S., Nugent, P. E., Pain, R., Panagia, N., Pennypacker, C. R.,
Perlmutter, S., Quimby, R., Raux, J., Regnault, N., Ruiz-Lapuente, P., Sainton, G.,
Schaefer, B., Schahmaneche, K., Smith, E., Spadafora, A. L., Stanishev, V., Sullivan,
M., Walton, N. A., Wang, L., Wood-Vasey, W. M., and Yasuda, N., New constraints
on omegam, omegal, and w from an independent set of eleven high-redshift supernovae
observed with hst, Astrophys.J. 598 (2003) 102, [astro-ph/0309368].

\bibitem{195} M.Kowalski, D.Rubin, G.Aldering, R.J.Agostinho, A.Amadon, R.Amanullah,
C.Balland, Barbary, K., G.Blanc, P.J.Challis, A.Conley, N.V.Connolly,
R.Covarrubias, K.S.Dawson, S.E.Deustua, R.Ellis, S.Fabbro, V.Fadeyev,
X.Fan, B.Farris, G.Folatelli, B.L.Frye, G.Garavini, E.L.Gates, L.Germany,
G.Goldhaber, B.Goldman, A.Goobar, D.E.Groom, J.Haissinski, D.Hardin,
I.Hook, S.Kent, A.G.Kim, R.A.Knop, C.Lidman, E.V.Linder, J.Mendez, J.Meyers,
G.J.Miller, M.Moniez, A.M.Mourao, H.Newberg, S.Nobili, P.E.Nugent, R.Pain,
O.Perdereau, S.Perlmutter, M.M.Phillips, V.Prasad, R.Quimby, N.Regnault, J.Rich,
E.P.Rubenstein, P.Ruiz-Lapuente, F.D.Santos, B.E.Schaefer, R.A.Schommer,
R.C.Smith, A.M.Soderberg, A.L.Spadafora, L.-G.Strolger, Strovink, M.,
N.B.Suntzeff, N.Suzuki, R.C.Thomas, N.A.Walton, L.Wang, and W.M.Wood-
Vasey, Improved cosmological constraints from new, old and combined supernova
datasets, Astrophys.J. 686 (2008) 749–778, [arXiv:0804.4142].

\bibitem{225} Riess, A. G., Strolger, L.-G., Tonry, J., Casertano, S., Ferguson, H. C., Mobasher,
B., Challis, P., Filippenko, A. V., Jha, S., Li, W., Chornock, R., Kirshner, R. P.,
Leibundgut, B., Dickinson, M., Livio, M., Giavalisco, M., Steidel, C. C., Benitez,
N., and Tsvetanov, Z., Type ia supernova discoveries at z>1 from the hubble space
telescope : Evidence for past deceleration and constraints on dark energy evolution,
Astrophys.J. 607 (2004) 665–687, [astro-ph/0402512].

\bibitem{71} Choudhury, T. R., and Padmanabhan, T., Cosmological parameters from supernova
observations : A critical comparison of three data sets, Astron.Astrophys. 429 (2005)
807, [astro-ph/0311622].

\bibitem{126} Freedman, W. L., Madore, B. F., Gibson, B. K., Ferrarese, L., Kelson, D. D., Sakai, S.,
Mould, J. R., R. C. Kennicutt, J., Ford, H. C., Graham, J. A., Huchra, J. P., Hughes,
S. M. G., Illingworth, G. D., Macri, L. M., and Stetson, P. B., Final results from the
hubble space telescope key project to measure the hubble constant, Astrophys.J. 553
(2001) 47–72, [astro-ph/0012376].

\bibitem{143} Hansen, B. M. S., Brewer, J., Fahlman, G. G., Gibson, B. K., Ibata, R., Limongi,
M., Rich, R. M., Richer, H. B., Shara, M. M., and Stetson, P. B., The white dwarf
cooling sequence of the globular cluster messier 4, Astrophys.J. 574 (2002) L155–
L158, [astro-ph/0205087].

\bibitem{163} Jimenez, R., Thejll, P., Jorgensen, U. G., MacDonald, J., and Pagel, B.,
Ages of globular clusters : a new approach, mnras 282 (1996) 926–942,
[arXiv :astro-ph/9602132].

\bibitem{223} Richer, H. B., Brewer, J., Fahlman, G. G., Gibson, B. K., Hansen, B. M., Ibata,
R., Kalirai, J. S., Limongi, M., Rich, R. M., Saviane, I., Shara, M. M., and Stetson,
P. B., The lower main sequence and mass function of the globular cluster messier 4,
Astrophys.J. 574 (2002) L151–L154, [astro-ph/0205086].

\bibitem{177} Komatsu, E., Dunkley, J., Nolta, M. R., Bennett, C. L., Gold, B., Hinshaw, G.,
Jarosik, N., Larson, D., Limon, M., Page, L., Spergel, D. N., Halpern, M., Hill, R. S.,
Kogut, A., Meyer, S. S., Tucker, G. S., Weiland, J. L., Wollack, E., and Wright, E. L.,
Five-year wilkinson microwave anisotropy probe (wmap) observations : Cosmological
interpretation, Astrophys.J.Suppl. 180 (2009) 330–376, [arXiv:0803.0547].

\bibitem{115} Eisenstein, D. J., Zehavi, I., Hogg, D. W., Scoccimarro, R., Blanton, M. R., Nichol,
R. C., Scranton, R., Seo, H., Tegmark, M., Zheng, Z., Anderson, S., Annis, J., Bahcall,
N., Brinkmann, J., Burles, S., Castander, F. J., Connolly, A., Csabai, I., Doi, M.,
Fukugita, M., Frieman, J. A., Glazebrook, K., Gunn, J. E., Hendry, J. S., Hennessy,
G., Ivezic, Z., Kent, S., Knapp, G. R., Lin, H., Loh, Y., Lupton, R. H., Margon,
B., McKay, T., Meiksin, A., Munn, J. A., Pope, A., Richmond, M., Schlegel, D.,
Schneider, D., Shimasaku, K., Stoughton, C., Strauss, M., SubbaRao, M., Szalay,
A. S., Szapudi, I., Tucker, D., Yanny, B., and York, D., Detection of the baryon
acoustic peak in the large-scale correlation function of sdss luminous red galaxies,
Astrophys.J. 633 (2005) 560–574, [astro-ph/0501171].

\bibitem{212} Percival, W. J., Reid, B. A., Eisenstein, D. J., Bahcall, N. A., Budavari, T., Fukugita,
M., Gunn, J. E., Ivezic, Z., Knapp, G. R., Kron, R. G., Loveday, J., Lupton, R. H.,
McKay, T. A., Meiksin, A., Nichol, R. C., Pope, A. C., Schlegel, D. J., Schneider,
D. P., Spergel, D. N., Stoughton, C., Strauss, M. A., Szalay, A. S., Tegmark, M.,
Weinberg, D. H., York, D. G., and Zehavi, I., Baryon acoustic oscillations in the
sloan digital sky survey data release 7 galaxy sample, arXiv:0907.1660.

\bibitem{195} M.Kowalski, D.Rubin, G.Aldering, R.J.Agostinho, A.Amadon, R.Amanullah,
C.Balland, Barbary, K., G.Blanc, P.J.Challis, A.Conley, N.V.Connolly,
R.Covarrubias, K.S.Dawson, S.E.Deustua, R.Ellis, S.Fabbro, V.Fadeyev,
X.Fan, B.Farris, G.Folatelli, B.L.Frye, G.Garavini, E.L.Gates, L.Germany,
G.Goldhaber, B.Goldman, A.Goobar, D.E.Groom, J.Haissinski, D.Hardin,
I.Hook, S.Kent, A.G.Kim, R.A.Knop, C.Lidman, E.V.Linder, J.Mendez, J.Meyers,
G.J.Miller, M.Moniez, A.M.Mourao, H.Newberg, S.Nobili, P.E.Nugent, R.Pain,
O.Perdereau, S.Perlmutter, M.M.Phillips, V.Prasad, R.Quimby, N.Regnault, J.Rich,
E.P.Rubenstein, P.Ruiz-Lapuente, F.D.Santos, B.E.Schaefer, R.A.Schommer,
R.C.Smith, A.M.Soderberg, A.L.Spadafora, L.-G.Strolger, Strovink, M.,
N.B.Suntzeff, N.Suzuki, R.C.Thomas, N.A.Walton, L.Wang, and W.M.Wood-
Vasey, Improved cosmological constraints from new, old and combined supernova
datasets, Astrophys.J. 686 (2008) 749–778, [arXiv:0804.4142].

\bibitem{211} Peebles, P. J. E., and Ratra, B., The cosmological constant and dark energy, Rev.
Mod. Phys. 75 (2003) 559–606, [astro-ph/0207347].

\bibitem{265} Uzan, J.-P., The acceleration of the universe and the physics behind it, Gen.Rel.Grav.
39 (2007) 307–342, [astro-ph/0605313].

\bibitem{221} Ratra, B., and Peebles, P. J. E., Cosmological Consequences of a Rolling Homoge-
neous Scalar Field, Phys. Rev. D37 (1988) 3406.

\bibitem{284} Wetterich, C., Cosmology and the Fate of Dilatation Symmetry, Nucl. Phys. B302
(1988) 668.

\bibitem{241} Sen, A., Supersymmetric world-volume action for non-bps d-branes, JHEP 9910
(1999) 008, [hep-th/9909062].
\bibitem{242} Sen, A., Rolling Tachyon, JHEP 04 (2002) 048, [hep-th/0203211].
\bibitem{243} Sen, A., Tachyon matter, JHEP 07 (2002) 065, [hep-th/0203265].

\bibitem{57} Caldwell, R., A phantom menace ? cosmological consequences of a dark energy
component with super-negative equation of state, Phys.Lett. B545 (2002) 23–29,
[astro-ph/9908168].

\bibitem{59} Caldwell, R. R., Kamionkowski, M., and Weinberg, N. N., Phantom energy and cos-
mic doomsday, Phys.Rev.Lett. 91 (2003) 071301, [astro-ph/0302506].

\bibitem{23} Armendariz-Picon, C., Mukhanov, V., and Steinhardt, P. J., A dynamical solution
to the problem of a small cosmological constant and late-time cosmic acceleration,
Phys.Rev.Lett. 85 (2000) 4438–4441, [astro-ph/0004134].
\bibitem{24} Armendariz-Picon, C., Mukhanov, V., and Steinhardt, P. J., Essentials of k-essence,
Phys.Rev.D 63 (2001) 103510, [astro-ph/0006373].

\bibitem{69} Chiba, T., Okabe, T., and Yamaguchi, M., Kinetically driven quintessence,
Phys.Rev.D 62 (2000) 023511, [astro-ph/9912463].

\bibitem{78} Csaki, C., Kaloper, N., and Terning, J., Dimming supernovae without cosmic accele-
ration, Phys.Rev.Lett. 88 (2002) 161302, [hep-ph/0111311].
\bibitem{41} Bergmann, P. G., Comments on the scalar tensor theory, Int. J. Theor. Phys. 1 (1968)
25–36.
\bibitem{99} Deffayet, C., Harari, D., Uzan, J.-P., and Zaldarriaga, M., Dimming of supernovae by
photon-pseudoscalar conversion and the intergalactic plasma, Phys.Rev. D66 (2002)
043517, [hep-ph/0112118].

\bibitem{102} Deffayet, C., and Uzan, J.-P., Photon mixing in universes with large extra-dimensions,
Phys.Rev. D62 (2000) 063507, [hep-ph/0002129].

\bibitem{206} Nordtvedt, Kenneth, J., PostNewtonian metric for a general class of scalar tensor
gravitational theories and observational consequences, Astrophys. J. 161 (1970) 1059–
1067.

\bibitem{276} Wagoner, R. V., Scalar tensor theory and gravitational waves, Phys. Rev. D1 (1970)
3209–3216.

\bibitem{9} Amendola, L., Charmousis, C., and Davis, S. C., Constraints on gauss-bonnet gravity
in dark energy cosmologies, JCAP 0612 (2006) 020, [hep-th/0506137].
\bibitem{10} Amendola, L., Charmousis, C., and Davis, S. C., Solar system constraints on gauss-
bonnet mediated dark energy, JCAP 0710 (2007) 004, [arXiv:0704.0175].

\bibitem{175} Koivisto, T., and Mota, D. F., Cosmology and astrophysical constraints of gauss-
bonnet dark energy, Phys.Lett.B 644 (2007) 104–108, [astro-ph/0606078].
\bibitem{176} Koivisto, T., and Mota, D. F., Gauss-bonnet quintessence : Background evolution,
large scale structure and cosmological constraints, Phys.Rev.D 75 (2007) 023518,
[hep-th/0609155].

\bibitem{183} Li, B., Barrow, J. D., and Mota, D. F., The cosmology of modified gauss-bonnet
gravity, Phys.Rev.D 76 (2007) 044027, [arXiv:0705.3795].

\bibitem{203} Nicolis, A., Rattazzi, R., and Trincherini, E., The galileon as a local modification of
gravity, arXiv:0811.2197.

\bibitem{94} Deffayet, C., Deser, S., and Esposito-Farese, G., Generalized galileons : All scalar
models whose curved background extensions maintain second-order field equations
and stress tensors, Phys. Rev. D 80 (2009) 064015, [arXiv:0906.1967].

\bibitem{97} Deffayet, C., Esposito-Farese, G., and Vikman, A., Covariant galileon, Phys. Rev. D
79 (2009) 084003, [arXiv:0901.1314].

\bibitem{249} Silva, F. P., and Koyama, K., Self-accelerating universe in galileon cosmology,
arXiv:0909.4538.

\bibitem{27} Babichev, E., Deffayet, C., and Ziour, R., k-mouflage gravity, International Journal
of Modern Physics D (2009) [arXiv:0905.2943].

\bibitem{111} Dvali, G., Gabadadze, G., and Porrati, M., 4d gravity on a brane in 5d minkowski
space, Phys.Lett.B 485 (2000) 208–214, [hep-th/0005016].

\bibitem{92} Deffayet, C., Cosmology on a brane in minkowski bulk, Phys.Lett.B 502 (2001) 199–
208, [hep-th/0010186].

\bibitem{91} de Rham, C., Hofmann, S., Khoury, J., and Tolley, A. J., Cascading gravity and
degravitation, JCAP 0802 (2008) 011, [arXiv:0712.2821].

\bibitem{112} Dvali, G., Hofmann, S., and Khoury, J., Degravitation of the cosmological constant
and graviton width, Phys.Rev.D 76 (2007) 084006, [hep-th/0703027].




\end{thebibliography}
\end{document}
